\section{Overview of measure theory}
\paragraph{Notation}
\begin{itemize}
	\item $S$ is a set.
	\item $\mathcal{A}$ is algebra of subsets of $S$
	\begin{enumerate}
		\item $S\in \mathcal{A}$
		\item $$E\in \mathcal{A} \Rightarrow E^C\in \mathcal{A}$$, where $E^C = S \setminus E$
		\item $$E_1, E_2 \in \mathcal{A} \Rightarrow E_1 \cup E_2 \in \mathcal{A}$$
		meaning
		$$E_1, E_2 \in \mathcal{A} \Rightarrow E_1 \cap E_2 \in \mathcal{A}$$
	\end{enumerate}
\item $\mathcal{F}$ is a $\sigma$-algebra if lest item works for countable union.
\item $E\Delta F = E\setminus F \cup F \setminus E$ 
\end{itemize}

\begin{definition}
	A measurable space is a pair $\left\{S,\mathcal{F}\right\}$.
\end{definition}

\begin{prop}
	If we have $\qty(\mathcal{F}_i)_{i\in I}$, then $\bigcap_{i\in I} \mathcal{F} $ is also a $\sigma$-algebra.
\end{prop}

\begin{definition}
	Let $C$ be a collection of subsets of $S$. $\sigma(C)$ is a smallest $\sigma$-algebra containing $C$ ($\sigma$-algebra generated by $C$).
	
	It is easy to construct one
	$$I = \left\{ \mathcal{F} :\mathcal{F} \supset C \right\}$$
	and then
	$$\sigma(C) = \bigcap_{\mathcal{F}\in I} \mathcal{F}$$
\end{definition}

\begin{definition}
	Let $\left\{S,\mathcal{F}\right\}$ be a topological space. $\mathcal{B}(X)$ (Borel $\sigma$-algebra) is defined as $\sigma$-algebra generated by open sets. We denote
	$\mathcal{B} = \mathcal{B}(\mathbb{R})$.
\end{definition}
\paragraph{Exercise}
$$\pi(\mathbb{R}) = \left\{ (-\infty, x], \: x\in \mathbb{R} \right\}$$
Show that $\sigma(\pi(\mathbb{R})) = B$

\begin{definition}
	Additive set function on a collection of sets $\mathcal{F}$ is
	$$\mu: \mathcal{F} \to [0, \infty)$$
	$$\forall E,F \in \mathcal{F} \: E\cap F = \emptyset \quad \mu(E\cup F) = \mu(E) +\mu(F) $$
	
	We say $\mu$ is $\sigma$-additive if same holds of countable infinite sets
	$$\forall \left\{ E_i \right\}_{i=1}^\infty \: E_i \cap E_j = \emptyset \quad \mu\qty(E\cup F) = \sum_{i=1}^\infty \mu(E_i)  $$
	
\end{definition}


\begin{definition}
	A triple $\qty(S, \mathcal{F}, \mu)$ is a measure space if $\mathcal{F}$ is a $\sigma$-algebra on $S$ and $\mu$ is $\sigma$-additive on $\mathcal{F}$.
	
\end{definition}


\begin{definition}
 $\qty(S, \mathcal{F}, \mu)$ is finite if $\mu(S) < \infty$
 
  $\qty(S, \mathcal{F}, \mu)$ is $\sigma$-finite if
	$$\exists \left\{ E_i, \: \mu(E_i)<\infty \right\}_{i=1}^\infty \quad S=\bigcup_{i=1}^\infty E_i$$
\end{definition}
\begin{definition}
If $\mu(S) =1$, $\qty(S, \mathcal{F}, \mu)$ is probability space.
\end{definition}

\begin{definition}
	$E$ is null if $\mu(E)=0$.
\end{definition}
\begin{definition}
	$\phi$ is said to be true almost everywhere with respect of $\mu$ if 
	$$\mu(\left\{ X : \phi(X) = \text{False} \right\}) = 0$$
\end{definition}

\subsection{Results from measure theory}
\begin{definition}
	A collection of sets $\mathcal{D}$ is called a $\pi$-system if $E,F \in D \: \Rightarrow \: E\cap F \in \mathcal{D}$
\end{definition}
\begin{theorem}[Uniqness]
	Let $\mathcal{D}$ be a $\pi$-system generating a $\sigma$-algebra $\mathcal{F}$. Let $\mu_1$ and $\mu_2$ be two finite measures on $\mathcal{F}$ which agree on $\mathcal{D}$. Then $\mu_1=\mu_2$.
	
	\begin{coll}
		$(S,\mathcal{F}, P_1)$, $(S,\mathcal{F}, P_2)$ probability spaces, $P1=P2$ on $\pi$-system $\mathcal{D}$, then $P_1=P_2$.
	\end{coll}
\end{theorem}

\begin{theorem}[Carath\'{e}odory's extension theorem]
	Let $\mathcal{A}$ be an algebra of sets. $\mu_0: \mathcal{A} \to \mathbb{R}^+$ $\sigma$-additive set function on $\mathcal{A}$. Then exists unique extension $\bar{\mu} : \sigma(\mathcal{A}) \to \mathbb{R}^+ $ such that $\bar{mu} = \mu_0$.
\end{theorem}

\paragraph{Homework}
Lebesgue on $\mathbb{R}$.  $\mathcal{A} = \left\{ \text{open set} \right\}$. If we have
$$O = \bigcup_{i=1}^\infty (a_i, b_i)$$
then
$$\mu_0(O) = \sum_{i=1}^\infty b_i-a_i$$

Check that $\mu_0$ is well defined and $\sigma$-additive.

\begin{lemma}
	
	$(S,\mathcal{F}, \mu)$ measure space. $A,B \in \mathcal{F}$, then
	$$\mu(A\cup B) \leq \mu(A) + \mu(B) $$
	$$\mu\qty(\bigcup_{i=1}^\infty F_i) \leq \sum_{i=1}^\infty \mu(F_i) $$
	
	If $\mu(S) < \infty$
	$$\mu(A\cup B) = \mu(A) + \mu(B) - \mu(A\cap B)$$
	
	From that we get inclusion-exclusion:
	$$\mu\qty(\bigcup_{i=1}^n A_i) = \sum_{i=1}^n \mu(A)_i - \sum_{i\neq j}\mu(A_i\cap A_j) + \dots + (-1)^{n-1} \mu\qty(\bigcap_{i=1}^n A_i)$$
\end{lemma}
\paragraph{Exercise}
Proof the lemma

\begin{lemma}
	If $F_n \subseteq F_{n+1}$ then
	$$\mu\qty(\bigcup_{i=1}^\infty F_i) = \lim_{n\to \infty} \mu(F_n)$$
	
	
	If $\mu(S) < \infty$ and $F_n \supseteq F_{n+1}$ then
	$$\mu\qty(\bigcap_{i=1}^\infty F_i) = \lim_{n\to \infty} \mu(F_n)$$
\end{lemma}
\begin{proof}
	Assume $\mu(S) < \infty$.  Define 
	$F_\infty = \bigcup_{i=1}^\infty F_i$.
	Let $G_n = F_{n}\setminus F_{n+1}$. Then
	$$F_\infty = \bigcup_{i=1}^\infty G_i$$
	Meaning 
	$$\mu(F_\infty) =\sum_{i=1}^\infty G_i$$
	$$\mu(F_n) =\sum_{k=1}^n G_k$$
	
	Since measure is finite, the tail of series tends to $0$, thus 
	$$\mu(F_\infty) - \mu(F_n) = \sum_{k=n}^\infty G_k \to 0$$
	
	Then we can take complements and get the second statement.
\end{proof}

\paragraph{Exercise}
Proof unconditionally