\documentclass[]{article}
\usepackage{amsmath}
\usepackage{amsfonts}
\usepackage{amssymb}
\usepackage{hyperref}
\usepackage{gensymb}
\usepackage{graphicx}
\usepackage{svg}
\usepackage{bbding}
\usepackage{mathtools}
\usepackage{centernot} % not parallel, etc.
\usepackage{lmodern}
\usepackage{morewrites}
\usepackage{xcolor,sectsty} % colorful sections
\usepackage[left=10mm, top=10mm, right=10mm, bottom=20mm, nohead]{geometry}
%\usepackage{bigints}
\usepackage{dsfont} %mathbb 1
\usepackage{esint} % beatiful integrals
\usepackage[arrowdel]{physics}
\usepackage{amsthm} % theorems

\usepackage[T1]{fontenc}
% Nicer default font (+ math font) than Computer Modern for most use cases
% \usepackage{mathpazo} % problems with greek vectors
\usepackage[utf8x]{inputenc} % Allow utf-8 characters in the tex document
% Prevent overflowing lines due to hard-to-break entities
\sloppy 
% Colors for the hyperref package
\definecolor{urlcolor}{rgb}{0,.145,.698}
\definecolor{linkcolor}{rgb}{.71,0.21,0.01}
\definecolor{citecolor}{rgb}{.12,.54,.11}
% Setup hyperref package
\hypersetup{
	breaklinks=true,  % so long urls are correctly broken across lines
	colorlinks=true,
	urlcolor=urlcolor,
	linkcolor=linkcolor,
	citecolor=citecolor,
}


\DeclareFontFamily{OMX}{lmex}{}
\DeclareFontShape{OMX}{lmex}{m}{n}{<-> lmex10}{}


%colors of sections
\definecolor{secfont}{RGB}{46,116,181}
\definecolor{subfont}{RGB}{146,23,57}
\definecolor{parfont}{RGB}{19,127,43}
\definecolor{subparfont}{RGB}{7,11,100}

\subsectionfont{\color{subfont}}
\sectionfont{\color{secfont}}
\paragraphfont{\color{parfont}}
\subparagraphfont{\color{subparfont}}


% declare a new theorem style
\newtheoremstyle{bluestyle}%
{3pt}% Space above
{3pt}% Space below 
{}% Body font
{}% Indent amount
{\bfseries\color{blue}}% Theorem head font
{.}% Punctuation after theorem head
{.5em}% Space after theorem head
{}% Theorem head spec (can be left empty, meaning ‘normal’)
% declare a new theorem style
\newtheoremstyle{redstyle}{3pt}{3pt}{}{}{\bfseries\color{red}}{.}{.5em}{}
\newtheoremstyle{olivestyle}{3pt}{3pt}{}{}{\bfseries\color{olive}}{.}{.5em}{}
\newtheoremstyle{orangestyle}{3pt}{3pt}{}{}{\bfseries\color{orange}}{.}{.5em}{}
\newtheoremstyle{magentastyle}{3pt}{3pt}{}{}{\bfseries\color{magenta}}{.}{.5em}{}

\theoremstyle{bluestyle}
\newtheorem{theorem}{Theorem}[section]
\theoremstyle{redstyle}
\newtheorem{definition}{Definition}[section]
\theoremstyle{magentastyle}
\newtheorem{coll}{Collary}[theorem]
\theoremstyle{olivestyle}
\newtheorem{lemma}{Lemma}[section]
\theoremstyle{olivestyle}
\newtheorem{prop}[theorem]{Proposition}

%\usepackage{babel}[english]
%opening
\title{106349 - Advanced probability}
\author{Nick Crawford}
% njc860@gmail.com
% Amado 707
% https://sites.google.com/site/njcrawfordacademic/
% David Williams Probability with Martingales
% Rick Derret Probability Theory and examples


\parindent=0em
\begin{document}


\maketitle

\begin{abstract}

\end{abstract}

%\tableofcontents
\section{Introduction. Summary of course through an example. Branching process}
We have an individual that gives a birth to a random number of offsprings -- random variable $X$.
$X$ define a distribution, i.e., $P: \mathbb{Z}^+ \to [0,1]$, i.e., $P(X=k) \in [0,1]$, and $\sum_{k=0}^\infty P(X=k)  = 1$.

\begin{definition}
	$f_X(\theta) = \sum_{k=0}^\infty \theta^k P(X=k)$ -- moment-generating function.
\end{definition}

 The series is absolutely convergent for $\theta\in [-1,1]$ since $k$ sums to $1$. For $\theta \in (-1,1)$, $f_x$ is analytic, thus we can differentiate it term-by-term:
$$f'_X(\theta) = \sum_{k\geq1} \theta^{k-1} P(X=k)$$
Since, $f_X$ is analytic, knowing it means knowing $P(X=k)$ and vice versa.

Note that $f_X(0) = P(X=0)$ and $f_X(1)=1$. Also
$$f'_X(1) = \sum_{k\geq 0 }^\infty k P(X=k) = \mathbb{E} X = \mu$$
$$\lim_{\theta \to 1} \frac{f_X(1)-f_X(\theta)}{1-\theta}=\lim_{\theta \to 1} \frac{1-f_X(\theta)}{1-\theta}$$

Note also that $f_X$ is convex, since second derivative is positive.
\paragraph{Size of $n^{th}$ generation}
Let $\qty(X_r^{(n)})_{n,r=1^\infty}$, where $n$ is generation and $r$ is offspring number (index) in $n^{th}$ generation.

Assume $X_r^{(n)}$ are i.i.d.  (independent, identically distributed) random variables.

Identically distributed means $$P(X_n^r = k) = P(X=k)$$.

Independence means $$P\qty(\forall i<J \: X_{r_i}^{n_i} =k) = \prod_{i=1}^J P\qty( X_{r_i}^{n_i} =k)$$.

Define $z_1=X_1^1$. $z_2 = \sum_{r=1}^{z_1} X_r^2$ an so on:
$$z_{n+1} = \sum_{r=1}^{z_n} X_{r}^{n}$$

We want to study asymptotics of $z_n$.

Given $U$ and $V$ taking values in $ \mathbb{Z}^+$,
$$\mathbb{E}] \qty[ U|V=k] = \sum_{j=0}^\infty j P(U=j | V=k)$$, where $$P(U=j|V=k) = \frac{P(U=j, V=k)}{P(V=k)}$$

\paragraph{} If $U$, $V$ are independent, $P(U=j|V=k) = P(U=j)$ and thus $\mathbb{E}\qty[ U|V=k] = \mathbb{E} U$.
\begin{definition}
Define random variable $\mathbb{E}\qty[U|V]$ such that
$$\mathbb{E}\qty[U|V] = \mathbb{E}\qty[U|V=k] $$
if $V=k$.
\end{definition}

\begin{definition}[Tower property]
$$\mathbb{E} \qty\big[\mathbb{E}\qty[U|V]] = \mathbb{E}U$$

Define $$f_{n} = \sum_{k=0}^\infty \sum_{k=0}^\infty \theta^{k} P(z_n=k) = \mathbb{E} \theta^{z_n}$$. 

\end{definition}
\begin{theorem}
$$f_{n+1} (\theta) = f_n (f_X(\theta))$$
or
$$f_{n} (\theta) = \underbrace{f\circ f\circ \dots \circ f}_{n \text{ times}}(\theta))$$
\begin{proof}
	
	Use tower property with $U^{z_{n+1}} $ and $V=\theta^{z_n}$.
	By tower property
	$$\mathbb{E} \qty[\theta^{z_{n+1}}] = \mathbb{E} \qty\big[\mathbb{E} \qty[\theta^{z_{n+1}}| \theta^{z_{n}}]]$$
	$$ \mathbb{E} \qty\big[\mathbb{E} \qty[\theta^{z_{n+1}}| \theta^{z_{n}}]] = \sum_{k=0}^\infty P(z_n=k) \mathbb{E} \qty[\theta^{z_{n+1}}| \theta^{z_{n}}=k]$$
What is $\mathbb{E} \qty[\theta^{z_{n+1}}| \theta^{z_{n}}=k]$?
$$\mathbb{E} \qty[\theta^{z_{n+1}}| \theta^{z_{n}}=k] =\mathbb{E} \qty[\theta^{\sum_{j=1}^{k} X_j^{n+1}}| \theta^{z_{n}}=k] \stackrel{\text{independence}}{=}\mathbb{E} \qty[\theta^{\sum_{j=1}^{z_n} X_j^{n+1}}]\stackrel{\text{independence}}{=} \prod_{j=1}^k \mathbb{E} \qty[\theta^{ X_j^{n+1}}] \stackrel{\text{i.d.}}{=} (f_X(\theta))^k $$
Thus
$$ \mathbb{E} \qty\big[\mathbb{E} \qty[\theta^{z_{n+1}}| \theta^{z_{n}}]] = \sum_{k=0}^\infty P(z_n=k) (f_X(\theta))^k = f_n(f(\theta))$$

Also we can say
$$\mathbb{E} \qty[\theta^{z_{n+1}}|z_n] = \qty(f_X(\theta))^{z_n}$$
\end{proof}
\end{theorem}

\paragraph{Study of $z_n$}
What is $\pi_n= P(z_n=0) = f_n(0) = f(\pi_{n-1})$, probability that population is extinguished. Since $z_{n-1} =0 \Rightarrow z_n=0$, i.e. $\pi_n$ is non-decreasing.

Let $P\qty(z_n=0 \text{ for some n}) = \pi$.

We hope that $\left\{ z_n=0 \right\}$ such that
$$\bigcup_n \left\{ z_n=0 \right\} = \left\{ z_n=0 \text{ for some n}\right\}$$
i.e., $\pi = \lim_{n\to \infty} \pi_n$. We call $\pi$ the extinction probability.

\begin{theorem}
	If $\mu=\mathbb{E} > 1$ then $\pi$ is a unique root of $\pi=f(\pi)$ and $\pi \in [0,1)$. If $\mu\leq 1$, $\pi=1$.
	
	If we look at $f(\pi)$ and $\pi$, they intersect in $1$, and they can intersect in two points since $f(x)$ is convex. There is second intersection iff $f'(1) = \mu > 1$.
\end{theorem}
\paragraph{Construction of $X_n^r$}
Construct set $\Omega$, $f_{n,r}: \Omega \to \mathbb{Z}^+$ and $\mathcal{F}$ a collection of subsets of $\Omega$ with $P: \mathcal{F} \to [0,1]$.

Let $\Omega = \mathbb{Z}^+ \times \mathbb{Z}^+$, $\mathcal{F} = \left\{ 0,1 \right\}^\Omega$. 

The problem is when we have infinitely number of variables.



\paragraph{Example}
Example of not well-behaved triple $(\Omega, \mathcal{F}, P)$. $\Omega =\mathbb{N}$. Now $\mathcal{F} = \left\{  C\subset \mathbb{N} : C \text{ has density} \right\}$.

$C$ has density means 
$$\frac{\abs{C \cap \mathbb{N}}}{n} \stackrel{n\to \infty}{\to} \rho(C)$$

However, for $C(m) = \left\{  1,2,\dots, m \right\}$, $\forall m \quad \rho(c_m)$, and 
$$\rho\qty(\bigcup C_m ) = 1$$

Thus $(\mathbb{N}, \mathcal{F}, \rho)$ is not a good probability space, since it doesn't fulfills this
$\pi_n\to \pi$ property. Note we can define other probabilities on naturals, for example
$$P\qty(\left\{ i \right\}) = 2^{-i}$$


\paragraph{Asymptotics of $z_i$ }
Assuming $\pi \in (0,1)$, what is behavior of $z_n$?
\begin{definition}
	$z_n$ is a Markov chain if 
	$$P(z_{n+1}=j|z_i=k_i \quad \forall i\leq n) = P(z_{n+1} = j | z_n=k_n)$$
	We can use to compute expectation:
	$$\mathbb{E}[z_{n+1}|z_i=k_i \quad \forall i<n] = E\qty[z_{n+1}|z_n=k_n]$$
\end{definition}

Then, since $E\qty[\sum_{i=1}^J X_i^n] = J\mu$
$$E\qty[z_{n+1}|z_n] = \mu z_n$$

Let $M_n = \frac{z_n}{\mu^n}$ then $\mathbb{E}[M_n] = 1$. Also
$$\mathbb{E}[M_{n+1}|z_0,\dots, z_n] = M_n$$
This is a definition of martingale with respect to $z_0, \dots, z_n$.

Let $\qty(\Omega, \mathcal{F}, P)$ we say $S$ happens almost surely (a.s.) if $$P\qty(\left\{ w\in \Omega : S \text{ is true for w} \right\}) = 1$$
\begin{theorem}[Martingale convergence theorem]
	If $M_n$ is a positive martingale then $\lim_{n\to \infty} M_N=M_\infty$ exists a.s. and 
	\begin{itemize}
		\item $\mu \leq 1$. $M_\infty =0 $ a.s. That means $\mathbb{E}M_\infty = 0$ but $\mathbb{E} M_ = 1$, i.e., 
		$$\mathbb{E}\qty[\liminf_{n\to \infty} M_n] < [\liminf_{n\to \infty} \mathbb{E}\qty[M_n]$$ 
		\item $\mu>1$. If $M_\infty > 0$ with positive probability then $z_n \sim \mu^n M_\infty$.  
	\end{itemize}
\end{theorem}


\begin{lemma}[Fatou's lemma] 
	$$\mathbb{E}\qty[\liminf_{n\to \infty} M_n] \leq \liminf_{n\to \infty} \mathbb{E}\qty[M_n]$$ 
\end{lemma}

\begin{theorem}
	$$\mathbb{E} \qty[M_\infty] = 1 \iff \mu>1 \quad \text{ and } \mathbb{E}\qty[X \log(X)] < \infty$$
\end{theorem}

\section{Overview of measure theory}
\paragraph{Notation}
\begin{itemize}
	\item $S$ is a set.
	\item $\mathcal{A}$ is algebra of subsets of $S$
	\begin{enumerate}
		\item $S\in \mathcal{A}$
		\item $$E\in \mathcal{A} \Rightarrow E^C\in \mathcal{A}$$, where $E^C = S \setminus E$
		\item $$E_1, E_2 \in \mathcal{A} \Rightarrow E_1 \cup E_2 \in \mathcal{A}$$
		meaning
		$$E_1, E_2 \in \mathcal{A} \Rightarrow E_1 \cap E_2 \in \mathcal{A}$$
	\end{enumerate}
\item $\mathcal{F}$ is a $\sigma$-algebra if lest item works for countable union.
\item $E\Delta F = E\setminus F \cup F \setminus E$ 
\end{itemize}

\begin{definition}
	A measurable space is a pair $\left\{S,\mathcal{F}\right\}$.
\end{definition}

\begin{prop}
	If we have $\qty(\mathcal{F}_i)_{i\in I}$, then $\bigcap_{i\in I} \mathcal{F} $ is also a $\sigma$-algebra.
\end{prop}

\begin{definition}
	Let $C$ be a collection of subsets of $S$. $\sigma(C)$ is a smallest $\sigma$-algebra containing $C$ ($\sigma$-algebra generated by $C$).
	
	It is easy to construct one
	$$I = \left\{ \mathcal{F} :\mathcal{F} \supset C \right\}$$
	and then
	$$\sigma(C) = \bigcap_{\mathcal{F}\in I} \mathcal{F}$$
\end{definition}

\begin{definition}
	Let $\left\{S,\mathcal{F}\right\}$ be a topological space. $\mathcal{B}(X)$ (Borel $\sigma$-algebra) is defined as $\sigma$-algebra generated by open sets. We denote
	$\mathcal{B} = \mathcal{B}(\mathbb{R})$.
\end{definition}
\paragraph{Exercise}
$$\pi(\mathbb{R}) = \left\{ (-\infty, x], \: x\in \mathbb{R} \right\}$$
Show that $\sigma(\pi(\mathbb{R})) = B$

\begin{definition}
	Additive set function on a collection of sets $\mathcal{F}$ is
	$$\mu: \mathcal{F} \to [0, \infty)$$
	$$\forall E,F \in \mathcal{F} \: E\cap F = \emptyset \quad \mu(E\cup F) = \mu(E) +\mu(F) $$
	
	We say $\mu$ is $\sigma$-additive if same holds of countable infinite sets
	$$\forall \left\{ E_i \right\}_{i=1}^\infty \: E_i \cap E_j = \emptyset \quad \mu\qty(E\cup F) = \sum_{i=1}^\infty \mu(E_i)  $$
	
\end{definition}


\begin{definition}
	A triple $\qty(S, \mathcal{F}, \mu)$ is a measure space if $\mathcal{F}$ is a $\sigma$-algebra on $S$ and $\mu$ is $\sigma$-additive on $\mathcal{F}$.
	
\end{definition}


\begin{definition}
 $\qty(S, \mathcal{F}, \mu)$ is finite if $\mu(S) < \infty$
 
  $\qty(S, \mathcal{F}, \mu)$ is $\sigma$-finite if
	$$\exists \left\{ E_i, \: \mu(E_i)<\infty \right\}_{i=1}^\infty \quad S=\bigcup_{i=1}^\infty E_i$$
\end{definition}
\begin{definition}
If $\mu(S) =1$, $\qty(S, \mathcal{F}, \mu)$ is probability space.
\end{definition}

\begin{definition}
	$E$ is null if $\mu(E)=0$.
\end{definition}
\begin{definition}
	$\phi$ is said to be true almost everywhere with respect of $\mu$ if 
	$$\mu(\left\{ X : \phi(X) = \text{False} \right\}) = 0$$
\end{definition}

\subsection{Results from measure theory}
\begin{definition}
	A collection of sets $\mathcal{D}$ is called a $\pi$-system if $E,F \in D \: \Rightarrow \: E\cap F \in \mathcal{D}$
\end{definition}
\begin{theorem}[Uniqness]
	Let $\mathcal{D}$ be a $\pi$-system generating a $\sigma$-algebra $\mathcal{F}$. Let $\mu_1$ and $\mu_2$ be two finite measures on $\mathcal{F}$ which agree on $\mathcal{D}$. Then $\mu_1=\mu_2$.
	
	\begin{coll}
		$(S,\mathcal{F}, P_1)$, $(S,\mathcal{F}, P_2)$ probability spaces, $P1=P2$ on $\pi$-system $\mathcal{D}$, then $P_1=P_2$.
	\end{coll}
\end{theorem}

\begin{theorem}[Carath\'{e}odory's extension theorem]
	Let $\mathcal{A}$ be an algebra of sets. $\mu_0: \mathcal{A} \to \mathbb{R}^+$ $\sigma$-additive set function on $\mathcal{A}$. Then exists unique extension $\bar{\mu} : \sigma(\mathcal{A}) \to \mathbb{R}^+ $ such that $\bar{mu} = \mu_0$.
\end{theorem}

\paragraph{Homework}
Lebesgue on $\mathbb{R}$.  $\mathcal{A} = \left\{ \text{open set} \right\}$. If we have
$$O = \bigcup_{i=1}^\infty (a_i, b_i)$$
then
$$\mu_0(O) = \sum_{i=1}^\infty b_i-a_i$$

Check that $\mu_0$ is well defined and $\sigma$-additive.

\begin{lemma}
	
	$(S,\mathcal{F}, \mu)$ measure space. $A,B \in \mathcal{F}$, then
	$$\mu(A\cup B) \leq \mu(A) + \mu(B) $$
	$$\mu\qty(\bigcup_{i=1}^\infty F_i) \leq \sum_{i=1}^\infty \mu(F_i) $$
	
	If $\mu(S) < \infty$
	$$\mu(A\cup B) = \mu(A) + \mu(B) - \mu(A\cap B)$$
	
	From that we get inclusion-exclusion:
	$$\mu\qty(\bigcup_{i=1}^n A_i) = \sum_{i=1}^n \mu(A)_i - \sum_{i\neq j}\mu(A_i\cap A_j) + \dots + (-1)^{n-1} \mu\qty(\bigcap_{i=1}^n A_i)$$
\end{lemma}
\paragraph{Exercise}
Proof the lemma

\begin{lemma}
	If $F_n \subseteq F_{n+1}$ then
	$$\mu\qty(\bigcup_{i=1}^\infty F_i) = \lim_{n\to \infty} \mu(F_n)$$
	
	
	If $\mu(S) < \infty$ and $F_n \supseteq F_{n+1}$ then
	$$\mu\qty(\bigcap_{i=1}^\infty F_i) = \lim_{n\to \infty} \mu(F_n)$$
\end{lemma}
\begin{proof}
	Assume $\mu(S) < \infty$.  Define 
	$F_\infty = \bigcup_{i=1}^\infty F_i$.
	Let $G_n = F_{n}\setminus F_{n+1}$. Then
	$$F_\infty = \bigcup_{i=1}^\infty G_i$$
	Meaning 
	$$\mu(F_\infty) =\sum_{i=1}^\infty G_i$$
	$$\mu(F_n) =\sum_{k=1}^n G_k$$
	
	Since measure is finite, the tail of series tends to $0$, thus 
	$$\mu(F_\infty) - \mu(F_n) = \sum_{k=n}^\infty G_k \to 0$$
	
	Then we can take complements and get the second statement.
\end{proof}

\paragraph{Exercise}
Proof unconditionally
\section{Recasting measure theory as probability}
\begin{definition}
	A probability space is a $(\Omega, \mathcal{F}, P)$ is a measure space such that $P(\Omega) = 1$.
	
	We call $\omega \in \Omega$ an outcome. $E\in \mathcal{F}$ is an event. $P(E)$ is probability of the event.
\end{definition}

\paragraph{Example}
Tossing finite or infinite sequence of coins.


\subparagraph{Tossing 4 coins}
$$\Omega = \left\{ HHHH, HHHT, HHTH, \dots, TTTH, TTTT \right\}$$
$$\mathcal{F} = 2^\Omega$$
$$P(\omega \in \Omega) = \frac{1}{\abs{\Omega}}$$
	
\subparagraph{Tossing infinite number coins}
$$\Omega = \left\{ 0,1 \right\}^{\mathbb{N}}$$


$\Omega$ has a natural topology which is called a product topology. It is coarsest topology such that $\pi_i: \Omega \to \left\{ 0,1 \right\}$ $\pi_i(\omega) =\omega_i$ is continuous. 

Let $\mathcal{F} = \mathcal{B}(\Omega)$.

Smallest $\sigma$-algebra such that
$$\pi_i^{-1}(0) \subset \Omega \in \mathcal{F} $$
$$\pi_i^{-1}(1) \subset \Omega \in \mathcal{F} $$

$$\pi_i (\Omega, \mathcal{F} ) \to \qty(\left\{ 0,1 \right\}, \left\{ 0,1 \right\}^{\left\{ 0,1 \right\}})$$

Natural $\pi$-system $\mathcal{F}_n$ smallest $\sigma$-algebra making $\pi_1, \dots, \pi_n$ measurable.

Note that
\begin{prop}
	
	$$\bigcup_n \mathcal{F}_n \neq \mathcal{F} $$
	\begin{proof}
		
		Define $S_n(\omega) = \sum_{i=1}^n \omega_n$. 
		$$X_n = \frac{S_n(\omega)}{n}$$
		Define
		$$Y(\omega) = \limsup X_n(\omega)$$
		$$E = \left\{ \omega : Y(\omega) \geq \frac{1}{3} \right\}$$
		$$E \in \mathcal{F} \setminus \bigcup_n \mathcal{F}_n$$
	\end{proof}
\end{prop}


\paragraph{What $\mathcal{F}_n$ looks like?}
For example, $\mathcal{F}_2$ has 4 outcomes, deciding only first two tosses.
\paragraph{Note} If we take $(\Omega_4, \mathcal{F}^{(4)}, P_4)$, restricting to $(\Omega_3, \mathcal{F}^{(3)}, P_3)$
$$P_4\qty(\left\{ (0,0,0,\omega_4)\right\}) = P_3\qty(\left\{ (0,0,0)\right\})  $$


Thus we want $P_{fair}$ defined on $\Omega$ to  fulfill same property:
$$P_{fair}(E) = P_n(\tilde{E})$$
where $E \in \mathcal{F}_n$ and $\tilde{E} \in F^{(n)}$.

\begin{definition}
	$E\subset \mathcal{F}$ occures almost surely (a.s.) if $P(E)=1$.
\end{definition}


\begin{definition}[$\limsup$ and $\liminf$]
 Let	$\left\{ E_n \right\}$ be a sequence of events.
 $$\limsup E_n = \bigcap_{m}\bigcup_{n\geq m} E_n = \left\{ E_n \text{ occurs infenetely often (i.o.)} \right\} = \left\{ \omega \in \Omega  : \: \forall m \: \exists n(\omega)>m \quad \omega \in E_n(\omega) \right\}$$
 
 Alternatively, $(\Omega, \mathcal{F})$ and $\left\{ E_n \right\}$  there is a natural map 
 $$I: \Omega \to \left\{ 0,1\right\}^N$$
 $$\omega \mapsto \left\{ 1_{E_n} (\omega) \right\}$$
 
 where $$1_{E}(\omega) = \begin{cases}
 0 & \omega \notin E\\
 1 & \omega \in E\\
 \end{cases}$$
 
 Now
 
 $$\liminf E_n = \bigcup_{m}\bigcap_{n\geq m} E_n = \left\{ E_n \text{ occurs eventually} \right\} = \left\{ \omega \in \Omega  : \: \exists m(\omega) \: \forall n\geq m(\omega) \quad \omega \in E_n(\omega) \right\}$$
\end{definition}

\paragraph{Remark}
Since everything is countable, if $E_n \in \mathcal{F}$, then $\limsup E_n,\liminf E_n\in \mathcal{F}$ 

We can write
$$\left\{ \frac{S_n}{n} \to \frac{1}{2} \right\} = \left\{ \limsup \frac{S_n}{n} \leq \frac{1}{2} \right\} \cap \left\{ \liminf \frac{S_n}{n} \geq \frac{1}{2} \right\}$$.

Choose $q\in \mathbb{Q}^+$ and take a look at $$\left\{ \liminf \frac{S_n}{n} > q \right\} = \liminf E_n(q)$$
where $E_n = \left\{ \omega  : \frac{S_n}{n} > q \right\}$.

In addition
$$\left\{ \limsup \frac{S_n}{n} < q \right\} = \liminf F_n(q)$$
where $F_n = \left\{ \omega  : \frac{S_n}{n} < q \right\}$.

Therefore
$\left\{ \liminf \frac{S_n}{n} > q \right\} \in \mathcal{F}$.


Finally,
$$\left\{ \liminf \frac{S_n}{n} \geq \alpha \right\} = \bigcap_{q<\alpha} \left\{ \liminf \frac{S_n}{n} > q \right\} $$



\begin{lemma}[Fatou's lemma]
	$$P\qty[\liminf_{n\to \infty} E_n] \leq \liminf_{n\to \infty} p\qty(E_n)$$
	
	\begin{proof}
		$$\liminf_{n\to \infty} E_n = \bigcup_m \bigcap_{n\geq m} E_n $$
		
		Sets $F_m= \bigcap_{n\geq m} E_n $ are increasing and $F_n \subseteq E_n$, thus
		$$P\qty[\liminf_{n\to \infty} E_n] = \lim_{n\to \infty} P(F_n) \leq \liminf_{n\to \infty} P(E_n)  $$
	\end{proof} 
\end{lemma}


\begin{lemma}[Fatou's lemma] \label{fatou}
	$$P\qty[\limsup_{n\to \infty} E_n] \geq \limsup_{n\to \infty} p\qty(E_n)$$
	
	\begin{proof}
		Note that $\qty(\limsup E_n)^C = \liminf E_n^C$, thus this is straightforward form previous lemma.
	\end{proof} 
\end{lemma}

\begin{lemma}[First Borel-Cantelli lemma] \label{bc1}
	Let $\left\{ E_n\right\} \subseteq \mathcal{F}$ be a sequence of events s.t. $\sum_n P(E_n) < \infty$, then
	$$P(E_n \text{ happens i.o.}) =0$$
	
	\begin{proof}
		$$P(E_n \text{ i.o.}) = P\qty(\bigcap_m \bigcup_{n\geq m} E_n) \leq  P\qty( \bigcup_{n\geq m} E_n) \leq \sum_{n=m}^\infty P(E_n) \stackrel{m\to \infty}{\to} 0$$
		Since $P(E_n \text{ i.o.})$ is independent on $m$, it got to be $0$.
	\end{proof}
\end{lemma}

\paragraph{Example}
Fix $\epsilon>0$. Look at $P\qty(\abs{\frac{S_n(\omega)}{n} -\frac{1}{2}}>\epsilon)$.

\paragraph{Claim} $$P\qty(\abs{\frac{S_n(\omega)}{n} -\frac{1}{2}}>\epsilon) \leq \frac{12}{\epsilon^4} \frac{1}{n^2}$$

By \ref{bc1} $P\qty(\abs{\frac{S_n(\omega)}{n} -\frac{1}{2}}>\epsilon \text{ i.o.}) = 0$ thus
$$\left\{ \frac{S_n}{n} \to \frac{1}{2} \right\} = \bigcap_{\epsilon>0} \left\{ \abs{\frac{S_n(\omega)}{n} -\frac{1}{2}}<\epsilon \text{ eventually}  \right\} = 1$$
		
		



\begin{definition}
	Let $\qty(S,\mathcal{F})$, $\qty(\Omega, \mathcal{B})$ be measurable spaces.
	$$\phi: S \to \Omega$$
	
	$\phi$ is $\qty\big((\mathcal{F}, \mathcal{B}))$-measurable if 
	$\forall B\in \mathcal{B} \quad \phi^{-1}(B) \in \mathcal{F} $.
\end{definition}

\paragraph{Remark}
$\mathcal{C}$ is collection of sets in $\Omega$. $\phi^{-1}(\mathcal{C}) = \left\{ \phi^{-1}(C) : C \in \mathcal{C} \right\}$.
\begin{itemize}
	\item $$\phi^{-1}\qty(\bigcap_{i\in I} B_i) =\bigcap \phi^{-1} (B_i)$$
	\item $$\phi^{-1}\qty(\bigcup_{i\in I} B_i) =\bigcup \phi^{-1} (B_i)$$
	\item $$\phi^{-1}\qty(B^C) =\qty[\phi^{-1} (B)]^c$$
\end{itemize}

\begin{lemma}
	Let $\sigma(\mathcal{C}) = \mathcal{B}$. $\phi$ is measurable iff $\phi^{-1}(\mathcal{C}) \subseteq \mathcal{F}$.
	
	\begin{coll}
		$\Omega = \mathbb{R}$, $\mathcal{B}(\mathbb{R})$ then $\phi$ is measurable iff
		$$\forall x \: \phi^{-1}((-\infty, x]) \subseteq \mathcal{F}$$
	\end{coll}
\end{lemma}
\begin{lemma}
	Let $\qty(S,\mathcal{F})$, $\qty(T, \mathcal{T})$, $\qty(\Omega, \mathcal{B})$ be measurable spaces. Let $\phi_1: S \to T$ and $\phi_2 T \to \Omega$ measurable. Then $\phi_2 \circ \phi_1$ is measurable.
	\begin{proof}
		Let $B\in \mathcal{B}$. Then $\phi_2^{-1}(B) \in \mathcal{T}$, and thus $\phi_1^{-1}\qty(\phi_2^{-1}(B)) \in \mathcal{F}$, meaning $\qty(\phi_2 \circ \phi_1)^{-1}\qty(B) \in \mathcal{F}$.
	\end{proof}
\end{lemma}
\begin{lemma}
	$\Omega = \mathbb{R}$. Then $\left\{ \phi| \phi \text{ is } \mathcal{F}, \mathcal{B} \text{-measurable} \right\}$ is an algebra over $\mathbb{R}$.
	\begin{proof}
		Using previous lemma and the fact $+$ is continuous, and thus measurable, we define $\Psi(s) = \qty(\phi_1(s), \phi_2(s))$.
		
		$\Psi$ is measurable. Take a look at
		$$\Psi^{-1}\big((-\infty, x_1] \cross (-\infty, x_2] \big) = \left\{  s: \phi_1(s) \in (-\infty, x_1] ,\: \phi_2(s) \in (-\infty, x_2]  \right\}$$
		
	\end{proof}
\end{lemma}

\paragraph{Notation} 
$$\phi: (S,\mathcal{F}) \to (\Omega, \mathcal{B})$$
We write $\phi\in \mathcal{F}$ for $\phi$ is $\mathcal{F},\mathcal{B}$ measurable.
\paragraph{Constructions preserved by measurability}
\begin{prop}
	If $\left\{ \phi_n \right\}_{n=1}^\infty$ measurable maps $(S,\mathcal{F}) \to (\Omega, \mathcal{B})$, then $\liminf \phi_n$, $\limsup \phi_n$, $\inf \phi_n$, $\sup_n$ are also measurable.
	\begin{proof}
		For example, fpr infimum, we need to show that
		$$\left\{s|\: \inf\limits_n \phi_n(s) \leq c  \right\} \in \mathcal{F}$$
		or alternatively,
		$$\left\{s|\: \inf\limits_n \phi_n(s) > c  \right\} \in \mathcal{F}$$
		which is just countable intersection:
		$$\bigcap_n \left\{ s: \phi_n(s) > c \right\}$$
		
		Same for $\limsup$, which is just infimum of supremum:
		$$\limsup \phi_n = \inf\limits_m \qty(\sup\limits_{n\geq m} \phi_n)$$
	\end{proof}
\end{prop}

\paragraph{Recall}
$$S_n = \text{number of 1's until n}$$
We can view $s_n$ as a composition of projection and sum:
$$\omega \mapsto \qty\big(\pi_1(\omega),\dots \pi_n(\omega)) \mapsto \sum_{i=1}^n \pi_i(\omega)$$
Both are continuous (projection from the definition of product topology) and thus measurable, and so is $\frac{S_n}{n}$.
\section{Random variables}
\begin{definition}
	Let $\qty(\Omega, \mathcal{F}, P)$ be a probability space. 
	$X: \Omega \to (S, \mathcal{S}) $ measurable is called a random variable. 
\end{definition}

\paragraph{Basic constructions with random variables}
\begin{definition}
	Given a probability space $\qty(\Omega, \mathcal{F}, P)$  and measurable $ (S, \mathcal{S})$, $X$ induces measure $\mathcal{L}_X$ on $ (S, \mathcal{S})$ via
	$$\mathcal{L}_X(E) = P(X \in E)$$

$\mathcal{L}_X$ is called marginal distribution of $X$ or law of $X$.
\end{definition}
\begin{prop}
	$\mathcal{L}_X$ is countably additive set function.
\end{prop}

If $ (S, \mathcal{S})$ is $\mathbb{R}, \mathcal{B}$. By uniqueness theorem, $\mathcal{L}_X$ if defined by
$$F_X(x) = \mathcal{L}_X\big((-\infty, x]\big) = P\big(X\in (-\infty, x]\big)$$ 

$\mathcal{L}_X \mapsto F_X$ is 1-1 (onto). $F_X$ is cumulative distribution function (CDF) or distribution function of $X$.

We ask the question: what are set properties distinguish $F_X$? 

\begin{prop}[Properties of CDF]
\begin{enumerate}
	\item $F_X$ is non-decreasing
	\item $F_X$ is right continuous
	\item $F_X(-\infty) = 0$ 
\end{enumerate}
\begin{proof}
	
	\begin{enumerate}
		\item If $x<y$, $(-\infty, x] \subseteq (-\infty, y]$, thus, from monotonicity of measure $F_X(x)\leq F_X(y)$  
		\item we want to show
		$$\lim_{x\downarrow x_0} F(x) = F(x_0)$$
		
		Since if $E_n \downarrow E$, then $\mu(E_n) \to \mu(E)$
		
		We can look on sequence $\left\{ x_n\right\}$:
		$$\omega \in \bigcap_n \left\{ X \in (-\infty, x_n) \right\} \Rightarrow \forall n \quad X(\omega) \leq x_n \Rightarrow X(\omega) \leq x \Rightarrow \bigcap_n \left\{ X \in (-\infty, x_n) \right\} \subseteq \left\{ X \in (-\infty, x) \right\}$$
		The other direction is obvious.
		\item Since 
		$$\bigcap_x X^{-1}\big((-\infty,x]\big) = \emptyset$$
	\end{enumerate}
\end{proof}
\end{prop}

\subsection{Independence}
$(\Omega, \mathcal{F}, P)$ probability space.
Let $\left\{ \mathcal{J}_i \right\}_{i\in I}$ be a collection of sub-$\sigma$-algebras of $\mathcal{F}$.
\begin{definition}[Independence of $\sigma$-algebras]
	Say $\left\{ \mathcal{J}_i \right\}_{i\in I}$ are independent if
	$$\forall i_1, \dots i_k \: \forall j \quad G_{ij} \in  \mathcal{J}_i \quad P\qty(\bigcap_{j=1}^k G_{ij}) = \prod_{j=1}^k P(G_{ij})$$
\end{definition}
\begin{definition}[Independence of random variables]
	Say $\left\{ X_i \right\}_{i\in I}$ are independent if $\sigma(X_i)$ are independent.
\end{definition}
\begin{definition}[Independence of sets]
Say $\left\{ E_i \right\}_{i\in I}$ are independent if random variables $\mathds{1}_{E_i}$ are independent.
\end{definition}
\begin{lemma}[Checking independence]
	Let $\mathcal{F}_1$, $\mathcal{F}_2$ be $\sigma$-algebras, $\mathcal{A}_1$, $\mathcal{A}_2$ $\pi$-systems such that $\sigma(\mathcal{A_1})=\mathcal{F}_i$.
	
	Then $\mathcal{F}_1$, $\mathcal{F}_2$ are independent iff 
	$$\forall A_1\in \mathcal{A}_1, \: A_2\in \mathcal{A}_2 \quad P(A_1\cap A_2) P(A_1)P(A_2)$$ 
	\begin{proof}
		Given $E \in \mathcal{F}$ let $P_E(A) = P(A\cap E)$, a measure on $(\Omega, \mathcal{F})$.
		
		Given $A_1\in \mathcal{A}_1$ consider $\eval{P_{A_1}}_{\mathcal{F}_2}$.
			
		$$\forall \: A_2\in \mathcal{A}_2 \: P_{A_1}(A_2) = P(A_1)P(A_2)$$
		
		Thus $P_{A_1}$ and $P(A_2)\times P$ are measures on $\mathcal{F}_2$ agreeing on $\mathcal{A}_2$.
		
		$$\forall A_1 \in \mathcal{A}_1 , E_2\in \mathcal{F}_2  \quad P(A_1\cap E_2) = P(A_1)P(E_2)$$
		
		Next iterate argument argument for  $\mathcal{F}_1 $
		$$\forall \: E_2\in \mathcal{F}_2 \: P_{E_2}= P(E_2)P$$
		
		By uniqueness 
		$$\forall E_i \in \mathcal{F}_i \quad P(E_1\cap E_2) = P(E_1)P(E_2)$$  
	\end{proof}

\begin{coll}
	To check $X_1,\dots, X_k$ are independent random variables it suffices 
	$$\forall x\in \mathbb{R}^k  \quad P(X_i\leq x_i) = \prod_{i=1}^k P(X_1\leq x_i)$$ 
\end{coll}
\end{lemma}

\begin{lemma}[Second Borel-Cantelli lemma]
	 \label{bc2}
	 If $\sum_i P(E_i) =\infty$ and $\left\{ E_i \right\}_{i=1}^\infty $ are independent, then 
	 
	 $$P(E_n \text{ i.o.}) =0$$
	 \begin{proof}
	 	$$\left\{ E_i  \text{ i.o.} \right\}^C =\left\{ E_i^C  \text{ eventually} \right\} $$
	 	It's enough to show
	 	$$P\qty(\bigcap_{i\geq n} E_i^C) = 0 $$
	 	or, by truncating
	 	$$P\qty(\bigcap_{i\geq n}^k E_i^C) = 0 $$
	 	
	 	By independence
	 	$$P\qty(\bigcap_{i\geq n} E_i^C) \prod_{i=n}^k P(E_i^C) = \prod_{i=n}^k \qty[1-P(E_i)] \leq e^{-\sum_{i=n}^k P(E_i)}$$
	 	
	 	Since the sum tends to infinity, the exponent tends to $0$.
	 \end{proof}
\end{lemma}

\paragraph{Example}
Let $\left\{ X_i \right\}_{i=1}^\infty$ be i.i.d. $Exp(1)$ random variables, i.e. 
$$P(X_i>x) = e^{-x}$$

We are interested in growth rate of $X_n \leq f(n)$.

If $f(n) = \alpha \log(n)$
$$P(X_n > f(n)) = e^{-f(n)} = n^{-\alpha}$$
Thus, from Lemmas \ref{bc1}, \ref{bc2}
$$P(X_n \geq \alpha \log(n) \text{ i.o.}) = \begin{cases}
0 & \alpha > 1\\
1 & \alpha \leq 1
\end{cases}$$
Define $L = \limsup_{n\to \infty} \frac{X_n}{\log(n)} $, then
$$P(L\geq 1) = P(X_n \geq \log(n) \text{ i.o.}) = 1$$

Finally, if we look at $E = \bigcup_{k=1}^\infty \left\{ L \geq 1+ \frac{1}{k} \right\}$,
$$P(E)  \leq \sum_{k=1}^\infty P\qty( L \geq 1+ \frac{1}{k}) \leq \sum_{k=1}^\infty P\qty( X_n \geq \qty(1+ \frac{1}{2k})\log(n)) = 0 $$
thus $P(L\leq 1)=1$.

\paragraph{Method of generation of i.i.d. uniform $[a,b]$ variables}

We write $\omega = \sum_{i=1}^\infty \frac{\omega_i}{2^i}$
$$\begin{cases}
w^{(1)} = \omega_1\omega_3\omega_6\omega_{10}\omega_{15}\dots\\
w^{(2)} = \omega_2\omega_5\omega_{9}\omega_{14}\dots\\
w^{(3)} = \omega_4\omega_8\omega_{13}\omega_{19}\dots\\
\vdots
\end{cases}$$

\begin{theorem}[Kolmogorov's zero–one law]
	
	$$\mathcal{T}_n = \sigma (X_n, X_{n+1}, \dots)$$
	$$\mathcal{T} = \bigcap_n \mathcal{T}_n$$
	
	
	Suppose $\left\{ X_n\right\}_{n=1}^\infty$ are independent, then $\forall A\in \mathcal{T}$, $P(A) \in \left\{ 0,1\right\}$.
	
	\begin{proof}
		We show that $P(A)=P(A)^2$.
		
		We show that $\mathcal{T}$ is independent of itself, $\forall A,B \in \mathcal{T} \quad P(A\cup B) = P(A)\cdot P(B)$.
		
		$\mathcal{T} \subset \mathcal{T}_1$. Consider $\mathcal{I}_{l,n} = \sigma (X_l, X_{l+1}, \dots, X_n)$
		
		For $n<k$, take $A\in\mathcal{I}_{l,n} $, $B\in\mathcal{I}_{k,m} $, then $A$, $B$ are independent.
		
		Let $\Pi_n = \bigcup_{m>n+1} \mathcal{I}_{n+1,m}$. $\Pi_n$ is $\pi$-system, and $\sigma(\Pi_n) = \mathcal{T}_n$.
		
		By lemma \ref{ind_lemma}, $\mathcal{I}_{l,n} $ is independent on $\mathcal{T}_n$. Thus $\forall A \in \mathcal{I}_{l,n}$, $B \in \mathcal{T}_n$,
		$$P(A\cap B) = P(A) \cdot P(B)$$
		
		$\mathcal{T}_1 = \sigma\qty(\bigcup_n \mathcal{I}_{1,n})$, and thus $\mathcal{T}_1$ is independent from $\mathcal{T}$ and since $\mathcal{T} \subseteq \mathcal{T}_1$, thus $\mathcal{T}$ is independent of itself and 
		$$\forall A\in \mathcal{T} \quad P(A) = P(A\cap A) = P(A)^2 $$
		
	\end{proof}
	\begin{coll}
		If $\left\{ X_n\right\}_{n=1}^\infty$ i.i.d. $s_n = \sum_{i=1}^n X_i$ then $\forall c\in \mathbb{R} \quad P(\limsup \frac{S_n}{n}\geq c) \in \left\{ 0,1 \right\} $.
	\end{coll}
\end{theorem}


\begin{prop}
Let $\left\{ X_n\right\}_{n=1}^\infty$ i.i.d. $s_n = \sum_{i=1}^n X_i$ then $ P(\limsup \frac{S_n}{n} \text{ exists}) \in \left\{ 0,1 \right\} $.

If $ P(\limsup \frac{S_n}{n} \text{ exists}) = 1 $, then 
$\exists c \in [-\infty, \infty]$ such that $P(\limsup \frac{S_n}{n} = c) = 1$.
\end{prop}

\section{Integration theory}
\begin{definition}[Notation]
	$(S, \mathcal{F}, \mu)$. Given $f: S\to \mathbb{R}$ measurable. We define
	$$\mu(f) = \int\limits_S f(S) \dd{\mu(S)}$$
	If $A\in \mathcal{F}$:
	$$\mu(f_{j}A) =  \mu(f\cdot \mathds{1}_A) = \int_Af(S) \dd{\mu(S)} = \int_A f \dd{\mu} $$
\end{definition}
\paragraph{Desirable properties of integral}
\begin{enumerate}
	\item Linearity $$\int \alpha f + g \dd{\mu} = \alpha \int f \dd{\mu} + \int g \dd{\mu}$$
	\item Positivity: $$f>0 \Rightarrow \int f \dd{\mu}>0$$
	\item $$\int \mathds{1}_A \dd{\mu} = \mu(A)$$
\end{enumerate}

Classes of functions we are going to consider:
\begin{enumerate}
	\item $$\mathcal{S} = \left\{ f(x) = \sum_{k=1}^m a_k \mathds{1}_{A_k} \: a_k>0, A_k \in \mathcal{F} \right\}$$ 
	\item $$\mathcal{P} = \left\{\text{positive measurable functions} \right\}$$ 
	\item $$\mathcal{I} = \left\{ f(x) = g(x)-h(x) | \: g,h\in \mathcal{P}, \int g \dd{\mu}  \text{ or } \int h \dd{\mu}  \text{ finite.} \right\} $$ 
\end{enumerate}

\begin{definition}
	For $\phi \in \mathcal{S}$ let $\mu_0(\phi) = \int_S \phi(S) \dd{\mu_0}  =\sum_{k=1}^m a_k \mu(A_k)$.
\end{definition}

\begin{definition}
For $f \in \mathcal{P}$ let $\mu(\phi) = \sup\limits_{\substack{\phi \in \mathcal{S} \\ \phi\leq f}} \mu_0(\phi)$.
\end{definition}

\begin{prop}
	If $f=g$ $\mu$ a.e., then $\mu(f)=\mu(g)$.
\end{prop}

\begin{lemma}
	If $\mu(f) = 0$ then $f=0$ a.e.
\end{lemma}
\begin{lemma}
	
	$$\mu\qty(\qty(\min\left\{f,k \right\}) \cdot  \mathds{1}_{\frac{1}{k}\leq f}) \to \mu(f)$$
	\begin{proof}
		Given $\phi \in S^+$, $\phi \leq f$, for $k$ large enough, $\phi \leq \qty(\min\left\{f,k \right\}) \cdot  \mathds{1}_{\frac{1}{k}\leq f}$.  Since $\phi = \sum_{l=1}^m a_l \mathds{1}_{A_l} $ $\phi \leq f\cdot  \mathds{1}_{\frac{1}{k}\leq f}$ and for $k$ large enough $\phi \leq \min\left\{f,k \right\}$. Thus 
		$$\phi \leq \qty(\min\left\{f,k \right\}) \cdot  \mathds{1}_{\frac{1}{k}\leq f}$$
		and
		$$\mu_0(\phi) \leq \mu\qty(\qty(\min\left\{f,k \right\}) \cdot  \mathds{1}_{\frac{1}{k}\leq f})$$
		Taking the limit
		$$\mu_0(\phi) \leq \lim_{k\to \infty} \mu\qty(\qty(\min\left\{f,k \right\}) \cdot  \mathds{1}_{\frac{1}{k}\leq f})$$
		By taking supremum over $\phi$
		$$\mu(f) \leq \lim_{k\to \infty} \mu\qty(\qty(\min\left\{f,k \right\}) \cdot  \mathds{1}_{\frac{1}{k}\leq f})$$
		Other direction is trivial and thus
		$$\mu\qty(\qty(\min\left\{f,k \right\}) \cdot  \mathds{1}_{\frac{1}{k}\leq f}) \to \mu(f)$$
	\end{proof}
\end{lemma}
\begin{theorem}[Monotone convergence theorem] \label{mct}
	Let $0<f_n$ and $f_n\uparrow f$, then $\mu(f_n) \uparrow \mu(f)$
	
	\begin{proof}
		From lemma we can assume $f$ is bonded and 
		$$\exists \epsilon \quad \left\{ f>0\right\}=\left\{ f>\epsilon\right\}$$
		
		If $\mu(f>\epsilon) = \infty$, then $\mu(f) = \infty$. Also
		$$\left\{ f_n > \frac{\epsilon}{2} \right\} \uparrow \left\{ f \geq \frac{\epsilon}{2} \right\} \Rightarrow \mu\qty(f_n > \frac{\epsilon}{2}) \to \mu\qty(f\geq)\frac{\epsilon}{2})$$
		Thus $\mu(f_n>\epsilon) \to \infty$
		
		If $\mu(f>\epsilon) < \infty$, given $\delta>0$, let
		$$C_n = \left\{ \abs{f-f_n} > \delta, f>\epsilon \right\}$$
		Then $C_n \downarrow \emptyset$, thus $\forall \delta>0$ $\mu(C_n) \to 0$.
		
		Given $S \ni \phi \leq f$, $$\phi = \phi \mathds{1}_{f>\epsilon} = (\phi-\delta) \mathds{1}_{f>\epsilon} + \delta \mathds{1}_{f>\epsilon}$$
		Let $\phi_n = (\phi-\delta) \mathds{1}_{f>\epsilon}\mathds{1}_{\abs{f-f_n}<\delta}$. Obviously $\phi_n\leq f_n$.
		
		We claim
		$$\exists C>0 \quad \abs{\mu_0(\phi_n) - \mu_0(\phi)} \leq C\cdot \qty(\delta + \mu(C_n))$$
		since
		$$\mu_0(\phi_n) = \mu_0(\phi) - \delta\mu\qty(f>\epsilon, \abs{f-f_n}>\delta) - \delta \mu(f>\epsilon)$$
		$$\abs{\mu_0(\phi_n) - \mu_0(\phi)} \leq \delta \mu(f>\epsilon) + M\mu(C_n)$$
		for $M\geq f$.
		
		Since $\delta$ is arbitrary, $\lim_{n\to \infty} C\cdot \qty(\delta + \mu(C_n)) = 0$, thus
		$$\mu_0(\phi) \leq \lim_{n\to \infty} \mu_0(f_n) $$
		
		Optimizing over $\phi$ we get
		$$\mu(f) \leq \lim_{n\to \infty} \mu(f_n)$$
	\end{proof}
\begin{coll}
	$f,g\in \mathcal{P}$ and $a\geq 0$ then
	$$\mu(af+g) = a\mu(f)+\mu(g)$$
	\begin{proof}
		Taking staircase functions $\alpha^{(r)}_f$ and $\alpha^{(r)}_g$. Then $a\alpha^{(r)}_f+ \alpha^{(r)}_g \uparrow af+g$ and
		$$\mu\qty(a\alpha^{(r)}_f+ \alpha^{(r)}_g) = a\mu\qty(\alpha^{(r)}_f) + \mu\qty(\alpha^{(r)}_g)$$
		By \ref{mct}
		$$\lim_{r\to \infty} \mu\qty(a\alpha^{(r)}_f+ \alpha^{(r)}_g) = \lim_{r\to \infty}  a\mu\qty(\alpha^{(r)}_f) + \mu\qty(\alpha^{(r)}_g)$$
		$$\mu(af+g) = a\mu(f)+\mu(g)$$
	\end{proof}
\end{coll}
\begin{coll}
$$\mu\qty(\liminf f_n) \leq \liminf \mu(f_n)$$
\begin{proof}
	$$\liminf f_n = \sup\limits_k \qty[\underbrace{\inf\limits_{n\geq k} f_n}_{g_k}] $$
	$$g_k \uparrow \liminf f_n$$
	From \ref{mct}
	$$\mu(\liminf f_n) = \lim_{k} \mu(g_k) \leq \liminf_k \mu(f_k)$$
\end{proof}
\end{coll}
\end{theorem}
\input{lect9.tex}
\begin{definition}
	Let $f=g-h$ a.s. such that $g,h\geq 0$, and at most one of $\int g\dd{\mu}, \int h\dd{\mu}$ is infinite. Then define
	$$\int f\dd{\mu} = \int g\dd{\mu} -\int h\dd{\mu}$$
\end{definition}
\begin{prop}
$\int f\dd{\mu}$ is well defined.
\begin{proof}
	If $g_1-h_1=f=g_2-f_2$ a.s. it is true that
	$$g_1-g_2+h_2-h_1 = 0$$
	$$g_1+h_2=g_2+h_1$$
	$$\int g_1 \dd{\mu} + \int h_2 \dd{\mu} = \int g_2\dd{\mu} + \int h_1\dd{\mu}$$
	Since maximum one term on each side is infinite we can move the other one to the second side, getting the 
	$$\int g_1 \dd{\mu} -\int g_2\dd{\mu}  = \int h_1\dd{\mu} - \int h_2 \dd{\mu} $$
	as required
\end{proof}
\end{prop}

\begin{definition}
	$$f^\pm(\omega) = \max \left\{ \pm f(\omega), 0 \right\}$$
\end{definition}

\begin{definition}
	We say $f\in L^1(\mu)$ if $\exists g,h$ such that $f=g-h$ $\int g \dd{\mu} + \int h\dd{\mu} <\infty$.
	
	For $f \in L^1(\mu)$, $\int f \dd{\mu} = \int f^+ \dd{\mu} - \int f^- \dd{\mu}$.
	
	$\abs{f} = f^++f^-$ and $f\in L^1(\mu) \iff \int \abs{f}\dd{\mu} <\infty$.
\end{definition}

\begin{lemma}
	$L^1(\mu)$ is a vector space.
	\begin{proof}
		$f,g \in L^1(\mu)$ thus, since $\abs{f+g} \leq \abs{f}+\abs{g}$, $f+g\in L^1(\mu)$.
		
		$$\int f+g \dd{\mu} = \int f^++g^+ \dd{\mu} - \int f^- + g^- \dd{\mu} = \int f\dd{\mu} + \int g\dd{\mu} $$
	\end{proof}
\end{lemma}

If $\norm{f}_1 = \int f \dd{\mu} $ then $\norm{}_1$ is a norm on $L^1(\mu)$

Further $L^1(\mu)$ is complete, i.e., each Cauchy sequence converges.

\begin{lemma}[Reverse Fatou's Lemma]
	 Let $\left\{  f_n\right\}$ be a sequence of functions such that $0\leq f_n\leq g$ such that $\int g\dd{\mu} <\infty$. Then
	 $$\int \limsup f_n \dd{\mu}\geq \limsup \int f_n \dd{\mu}$$
	 \begin{proof}
	 	Let $h_n = g-f_n$. By \ref{fatou} $$\int \liminf h_n \dd{\mu} \leq \liminf \int h_n \dd{\mu}$$
	 	Using the fact $\liminf h_n = g -\limsup f_n$ we get the result.
	 \end{proof}
\end{lemma}
\begin{theorem}[Lebesgue's dominated convergence theorem ]
	Let $f_n$ be a sequence such that $\abs{f_n}\leq g$ and $g\in L^1(\mu)$ and $f_n\to f$ then
$$\int f_n \to \int f$$
and 
$$\int \abs{f_n-f} \to 0 $$
\begin{proof}
	We first proof that $\int f_n \to \int f$.
	
	
	Sine $\abs{f_n} < g$, $g\pm f_n \geq 0$. Applying \ref{fatou} to $g\pm f_n $:
	$$\int \liminf f_n \leq \liminf \int f_n$$
	$$\int f \leq \liminf \int f_n$$
	and
	$$\int \liminf (-f_n) \leq \liminf \int (-f_n)$$
	$$\int \limsup f_n \geq \limsup \int f_n$$
	$$\int f \geq \limsup \int f_n \geq \liminf \int f_n \geq \int f$$
	
	We have $h_n = \abs{f_n-f}$, and $h_n \stackrel{a.s.}{\to} 0$ and $h_n \leq 2g$ so by first statement
	$$0 = \lim_{n\to \infty} \int h_n$$
	
\end{proof}
\end{theorem}

\subsection{Integration on probability spaces and integration}
\begin{definition}[Expectation]
	Let $(\Omega, \mathcal{F}, P)$ be a probability space and $X: \Omega \to \mathbb{R}$ be a random variable.
	$$\mathbb{E}\qty[X] =  \int X(\omega) \dd{P(\omega)}$$
\end{definition}
\begin{theorem}[Bounded convergence theorem] \label{bct}
	Let $X_n \to X$ a.s. and $\abs{X}_n \leq C$. Then $\mathbb{E} \qty[\abs{X}_n-X] to 0$.
	\begin{proof}[independent of DCT]
		Define $E_{\epsilon} = \left\{ \omega : \abs{X_n(\omega) - X(\omega)} < \epsilon \right\}$.
		
		$$\mathbb{E} \qty[\abs{X_n(\omega) -X}] = \mathbb{E} \qty[\abs{X_n - X} \mathds{1}_{E_\epsilon}]+\mathbb{E} \qty[\abs{X_n - X} \mathds{1}_{E_\epsilon^C}]$$
		Since $\abs{X_n - X} \mathds{1}_{E_\epsilon} \leq \epsilon \mathds{1}_{E_\epsilon}$ and $\abs{X_n - X} \mathds{1}_{E_\epsilon^C} \leq 2C \mathds{1}_{E_\epsilon^C}$:
		
		$$\mathbb{E} \qty[\abs{X_n(\omega) -X}] \leq \epsilon \mathds{1}_{E_\epsilon} + 2C \mathds{1}_{E_\epsilon^C} \leq \epsilon + 2c P\qty((E_\epsilon^n)^C) $$
		
		For some $m>n$
		$$\quad (E_\epsilon^n)^C = \left\{ \omega : \abs{X_n - X} >\epsilon \right\} \subseteq  \left\{ \omega : \abs{X_m - X} >\epsilon \right\}$$
		
		$$ \bigcap_n F_{n,\epsilon} = \left\{ \omega : \limsup \abs{X_n -X} => \epsilon \right\}$$
		By continuity of measure
		$$\lim_{n\to \infty} P(F_{n,\epsilon}) =0$$
		$$\lim_{n\to \infty} P\qty((E_\epsilon^n)^C)  =0$$
	\end{proof}
\end{theorem}

\begin{definition}
	Let $A\in \mathcal{F}$. 
	$$\int_A f \dd{\mu} = \mu(f;A) = \mu(f\mathds{1}_A)$$
	
	We can look on it as constructing a new measure space:
	$\qty(S\cap A = A, \mathcal{F}_A, \eval{\mu}_A)$
	
	We claim that 
	$$\eval{\mu}_A (f)= \mu (f\mathds{1}_A)$$
\end{definition}

\begin{prop}[The standard machine]
	\begin{enumerate}
		\item Check for $\mathds{1}_E$.
		\item Check for simple functions (use linearity)
		\item Use MCT to check positive functions.
		\item Use linearity to extend to $L^1$. 
	\end{enumerate}
\end{prop}
\begin{prop}
	If $h,g: \mathbb{R}\to \mathbb{R}$ are Borel-measurable and $X$,$Y$ are independent, then $h(X)$, $g(Y)$ are independent.
	\begin{proof}
		$h(X)$, $g(Y)$ are measurable, and $\sigma$-algebras generated by then are sub-$\sigma$-algebras of original ones, and thus they're independent.
	\end{proof}
\end{prop}
\begin{prop}
	Suppose $X,Y \in L^1$ are independent and in $L_1$, then $X\cdot Y \in L_1$ and
	$$\mathbb{E}\qty[XY] = \mathbb{E}\qty[X] \cdot \mathbb{E}\qty[Y]$$
	
	\begin{proof}
		$X=X^+-X_-$ and $Y=Y^+-Y^-$, by linearity its enough to check for $X^{\pm}$, $Y^\pm$.
		
		So we can assume $X,Y>0$. Denote $X_N = \max\left\{ X, N \right\}$, $Y_N = \max\left\{ X, N \right\}$, from \ref{mct} if the claim holds for $X_N$, $Y_N$, then it holds for $X$, $Y$.
		
		Since now $X$, $Y$ are bounded, we can find simple functions $\alpha^{(r)}(X) \to X$, $\alpha^{(r)}(Y) \to Y$. 
		
		From \ref{bct} the identity would hold for bounded functions if it holds for simple functions. From linearity it's enough to show for indicators:
		$$\mathbb{E} \mathds{1}_E(X) \mathds{1}_F(Y)  = P(X \in E, Y\in F) = P(X\in E)P(Y\in F) = \mathbb{E} \mathds{1}_E(X) \cdot \mathbb{E}  \mathds{1}_F(Y)$$
	\end{proof}
\end{prop}

\begin{prop}
	Let $X: (\Omega, \mathcal{F}) \to \qty(\mathbb{R}, \mathcal{B})$. Let $\mu_X$ be a law of $X$ on $\mathbb{R}$:
	$$\mu_X(B) = P(X\in B)$$
	let $h: \mathbb{R} \to \mathbb{R}$, then $\mathbb{E}\qty[h(X)] = \int_{\mathbb{R}} h(x) \mu_X(\dd{x})$
	\begin{proof}
		Use the Standard machine.
	\end{proof}
 	\begin{proof}
	\end{proof}
\end{prop}

\begin{definition}
	Say $\nu \triangleleft \mu$ if $\mu(A) = 0 \Rightarrow \nu(A) = 0$. We say $\mu$ is absolutely continuous with respect to $\mu$.
\end{definition}

\begin{theorem}
	$S, \mathcal{F}$ is nice. $\nu  \triangleleft \mu \iff \exists f\in L^1(\mu)$ and $f\geq 0$, $\int f \dd{\mu} = 1$ such that $\nu(A) = \int_A f \dd{\mu}$. We call $f = \pdv{\nu}{\mu} $ a Radon-Nikodym derivative.
\end{theorem}

$P(X=\vdot|Z=_j$ is probability measure on $\left\{ x_1, \dots, x_m \right\}$ for $j$ fixed. $\mathbb{E}\qty[X|z=z_j] = \sum_i x_i P(X=x_i|z=z_j$ is a conditional expectation. note that it is random variable: $\mathbb{E}[X|Z] = \sum_j \mathbb{E}[X|Z=z_j$]. 

Properties
\begin{enumerate}
\item $\mathbb{E}\qty[X|Z] \in \sigma(Z)$
\item $\forall A \in \sigma(Z)$
$$\mathbb{E}\qty\bigg[\mathds{1}_A \cdot \mathbb{E}[X|Z] ] = \mathbb{E}\qty[\mathds{1}_A X]$$

It's enough to check for $A=\left\{ z_j \right\}$:
$$\mathbb{E}\qty\bigg[\mathds{1}_A \mathbb{E}[X|Z]] =  P(Z=z_j)\mathbb{E}\qty\bigg[\mathbb{E}[X|z_j]] =  \sum_i x_i P(X=x_i, Z=z_j) = \mathbb{E}\qty[\mathds{1}_A X]$$


\end{enumerate}
\begin{definition} 
We say that $Y$ is a conditional expectation of $Z$ given $\mathcal{J}$ if $Y\in\mathcal{J}$, $Y\in L^1(P)$ and $\forall A \in \mathcal{J}$ $\mathbb{E}[Y\mathds{1}_A] = \mathbb{E}[X\mathds{1}_A]$
\end{definition}

\begin{lemma}
If conditional expectation $Y$ exists, it is unique up to sets of measure $0$. Henceforth call $Y$ by $\mathbb{E}[X|\mathcal{J}]$
\begin{proof}
Let $Y$, $Y'$ two conditional expectations. We'll show that $Y>Y'$ $P$ a.s. then by symmetry $Y=Y'$ $P$ a.s.

Let $A_n = \left\{ Y>Y' + \frac{1}{n} \right\}$, $A_n \in \mathcal{J}$.
\end{proof}
\end{lemma}


\paragraph{$L^2(P)$ remark}
For $X,Y\in L^2(P)$  define $\langle X, Y \rangle = \mathbb{E}[XY]$  and $\norm{X} = \sqrt{\langle X, X \rangle} = \sqrt{\mathbb{E}[X^2]}$.
Parallelogram law:
$$\norm{X+Y}^2 + \norm{X-Y}^2 = 2\norm{X}^2 + 2\norm{Y}^22$$
\begin{theorem}
	Let $K\subseteq L^2(P)$ be a closed convex subset, $X \in  L^2(P)$ then $\exists ! Y\in K$ such that $\inf \left\{ \norm{X-Z}: z\in K \right\} = \norm{X-Y}$, call $Y=P_K(X)$.
	\begin{proof}
		Let $Y_n \in K$ such that $\norm{X-Y_n} \to \Delta = \inf \left\{ \norm{X-Z}: z\in K \right\} $
		
		We claim $Y_n$ is Cauchy.
		$$\norm{X-Y_n}^2 + \norm{X-Y_m}^2 = 2\norm{X - \frac{Y_n+Y_m}{2}}^2 + \frac{1}{2} \norm{Y_n-Y_m}^2$$
	Since $\frac{Y_n+Y_m}{2} \in K$ by convexity, thus
	$$ 2\norm{X - \frac{Y_n+Y_m}{2}}^2 + \frac{1}{2} \norm{Y_n-Y_m}^2 \geq 2\Delta^2 + \frac{1}{2} \norm{Y_n-Y_m}^2$$
and 
	$$\norm{X-Y_n}^2 + \norm{X-Y_m}^2 \to 2\Delta^2$$
	thus	$$ \frac{1}{2} \norm{Y_n-Y_m}^2 \to 0$$
i.e., $\left\{ Y_n \right\}$ is Cauchy and thus the limit exists and is in $K$.

If there are two different sequences, then from same identity, the distance between limits goes to $0$.               
	\end{proof}
\end{theorem}
\begin{theorem}
	If $X\in L^1$ and $\mathcal{J} \subseteq  \mathcal{F}$ then $\mathbb{E} [X|\mathcal{J} ]$ exists
\end{theorem}
\begin{definition}
	$P_K(X) = Y$, $P_K: L^2(P) \to V$. Then $P_K(X)$ is linear, contractive and self-adjoint.
\end{definition}

We want to use $P_K$ to define $\mathbb{E}\qty[X|\mathcal{J}]$.
\begin{itemize}
	\item For all $\mathcal{J} \subset \mathcal{F}$ containing all sets of measure $0$ $\left\{ Y, \mathbb{E} \qty[Y^2] \right\}$ forms a closed subspace of $L^2(P)$ and $L^2(\Omega, \mathcal{J}, P)$ is called complete measure space.
	 \item $L^2(\Omega, \mathcal{J}, P)$ is complete by Lemma.
	 \item $L^2(\Omega, \mathcal{J}, P) \subseteq L^2(\Omega, \mathcal{F}, P)$
\end{itemize}

\begin{definition}
	For $X \in L^2(\Omega, \mathcal{F}, P)$ and $\mathcal{J} \subseteq \mathcal{F}$ is complete then
	$$\mathbb{E} \qty[X | \mathcal{F}] = P_{L^2(\Omega,\mathcal{J},P)} (X)$$
\end{definition}

\begin{prop}
	$$\mathbb{E} \qty[\mathds{1}_A X] = \mathbb{E} \qty[\mathds{1}_A P \cdot (X)]$$
	\begin{proof}
		The statement is equvivalent to
		$$\left\langle \mathds{1}_A, X\right\rangle = \left\langle \mathds{1}_A, P_{L^2(\mathcal{J})}X \right\rangle$$
		However for $Y\in L^2(\Omega, \mathcal{J}, P)$, $P\cdot Y= Y$:
		$$\left\langle \mathds{1}_A, X\right\rangle = \left\langle P\cdot \mathds{1}_A, X \right\rangle=\left\langle  \mathds{1}_A,  P\cdot X  \right\rangle$$
	\end{proof}
\end{prop}

\begin{lemma}
	If $X\geq 0$ a.s. and $X\in L^2$ then $\mathbb{E} \qty[X | \mathcal{J}] \geq 0$ a.s.
	\begin{proof}
		Let $A_n = \left\{ \omega : \mathbb{E}[X|\mathcal{J}] <\frac{1}{n} \right\}$. We clame $P(A_n=0)$.
		$$0 \leq \mathbb{E}\qty[\mathds{1}_{A_n} X] =  \mathbb{E}\qty[\mathds{1}_{A_n} \mathbb{E}\qty[X|\mathcal{J}]] \leq -\frac{1}{n} \mathds{1}_{A_n} \Rightarrow P(A_n) = 0 $$
	\end{proof}
\end{lemma}

Note that to define conditional expectation $\mathbb{E}[X|\mathcal{J}]$ for $X\in L^1$, we can assume $X>0$, since we can define 
$$\mathbb{E}[X|\mathcal{J}] = \mathbb{E}[X^+|\mathcal{J}]  - \mathbb{E}[X^-|\mathcal{J}] $$

So, let $X_n=\max\left\{X, n\right\}$, then $X_n \in L^2(\mathcal{F})$ so $\mathbb{E}[X_n | \mathcal{J}]$ exists, and $X_n \uparrow X$. We want that (a.s.) $\mathbb{E}[X_n | \mathcal{J}] \uparrow_n $

Take a look at $X_n-X_m\geq 0$ for $n>m$:
$$\mathbb{E}\qty[X_n-X_m| \mathcal{J}] \geq 0$$
thus
$$\mathbb{E}\qty[X_n| \mathcal{J}] \geq \mathbb{E}\qty[X_m| \mathcal{J}]$$
so the sequence is increasing.

Let us define
$$\mathbb{E}[X | \mathcal{J}] = \lim_{n\to \infty} \mathbb{E}[X_n | \mathcal{J}]$$

Now $\forall A \in \mathcal{J}$, from monotone convergence,
$$\mathbb{E}\qty[\mathds{1}_AX_n] \to \mathbb{E}\qty[\mathds{1}_AX]$$
$$\mathbb{E}\qty[\mathds{1}_AX_n] = \mathbb{E}\qty[\mathds{1}_A\mathbb{E} \qty[X_n | \mathcal{J}]]$$
and also
$$\mathbb{E}\qty[\mathds{1}_A\mathbb{E} \qty[X_n | \mathcal{J}]] \to \mathbb{E}\qty[\mathds{1}_A\mathbb{E} \qty[X | \mathcal{J}]]$$
and thus
$$\mathbb{E}\qty[\mathds{1}_AX] = \mathbb{E}\qty[\mathds{1}_A\mathbb{E} \qty[X | \mathcal{J}]]$$
i.e., it is the conditional expectation.

\begin{prop}[Properties of conditional expectations]
	
	\begin{enumerate}
		\item If $Y = \mathbb{E}\qty[X| \mathcal{J}]$ a.s., then $\mathbb{E}[X] = \mathbb{E}[Y]$
		\item If $X\in \mathcal{J}$, $\mathbb{E}\qty[\mathbb{X|\mathcal{J}} ]= X$ a.s.
		\item $$\mathbb{E} \qty[aX+Y|\mathcal{J}] = a \mathbb{E} \qty[X|\mathcal{J}]+ \mathbb{E} \qty[Y|\mathcal{J}]$$
		\item If $X\geq 0$  then $ \mathbb{E}[X| \mathcal{J}] \geq 0$
		\item If $X_n\uparrow X $  then $ \mathbb{E}[X_n| \mathcal{J}] \uparrow \mathbb{E}[X| \mathcal{J}]$
		\item If $X_n \geq 0$ 
		$$ \mathbb{E}[\liminf X_n| \mathcal{J}]  \leq \liminf \mathbb{E}[X_n| \mathcal{J}] $$
		\item $\abs{X_n} \leq V(w)$ and $\mathbb{E}[V] < \infty $, $X_n \to X$ a.s., then 
		$$ \mathbb{E}[X_n| \mathcal{J}] \to \mathbb{E}[X| \mathcal{J}]$$
		\item 
		$$\mathbb{E} \qty[c(X) | \mathcal{J}] \geq c(\mathbb{E} \qty[X | \mathcal{J}])$$
		\item For $\mathcal{H} \subseteq \mathcal{J} \subset\mathcal{F}$
			$$\mathbb{E} \qty[\mathbb{E}[X| \mathcal{J}] | \mathcal{H}] = \mathbb{E}[X| \mathcal{H}]$$
		\item For $Z\in \mathcal{J}$:
		$$\mathbb{E}[ZX| \mathcal{J}]  =Z \mathbb{E}[X| \mathcal{J}] $$
		\item For $\mathcal{H}$ independent on $\sigma(X, \mathcal{J})$:
		$$\mathbb{E}[X| \sigma\qty(\mathcal{J}, \mathcal{H})] = \mathbb{E}[X| \mathcal{J}]$$
	\end{enumerate}
\end{prop}
\begin{prop}[Jensen's inequality]
	\begin{proof}
		So let
		$$S = \left\{ a,b | ax+b\leq c(x) \right\} $$
		Let $S'\subset S$ be a countable dense subset.
		$$\forall a,b\in S' \quad a\mathbb{E} [X|\mathcal{J}] +b \leq \mathbb{E} [c(X)|\mathcal{J}]$$
		
		Since for convex function
		$$c(x) = \sup\limits_{ax+b\leq c(x)} \left\{ ax+b \right\}$$
		Optimizing over $S'$ we get the Jensen inequality.
	\end{proof}
\end{prop}
\begin{prop}[H\"{o}lder's inequality]
	$$\abs{\mathbb{E} [X|\mathcal{J}]}^{p} \leq \mathbb{E} [X^p|\mathcal{J}]$$
	If $\frac{1}{p}+\frac{1}{q}=1$:
	$$\abs{\mathbb{E} [XY|\mathcal{J}]} \leq \qty(\mathbb{E} [\abs{X}^p|\mathcal{J}])^{\frac{1}{p}} + \qty(\mathbb{E} [\abs{Y}^q|\mathcal{J}])^{\frac{1}{q}}$$
\end{prop}
\begin{prop}[Minkowski's inequality]
$$\abs{\mathbb{E} [(X+Y)^p|\mathcal{J}]}^{\frac{1}{p}} \leq \abs{\mathbb{E} [X^p|\mathcal{J}]}^{\frac{1}{p}} + \abs{\mathbb{E} [Y^p|\mathcal{J}]}^{\frac{1}{p}}$$
\end{prop}


\paragraph{Example}
Let $X$, $Z$ be random variable.
$$P\qty((X,Z)\in A) = \int\limits_A f_{(X,Z)} (x,z) \dd{x} \dd{z}$$
where $f_{(X,Z)} $ is called the joint distribution

\begin{prop}
$$\mathbb{E}[h(x) | \sigma(Z)] = \frac{\int f(x,z) h(x) \dd{x}}{\int f(x,z) \dd{x}} \mathds{1}_{\int f(x,z) \dd{x} \neq 0} = \phi(z)$$

\begin{proof}
	We want to check that
$$\mathbb{E}[h(X); z\in \mathcal{B} ] =\mathbb{E}[\phi(z); z\in \mathcal{B}$$
$$\mathbb{E}[\phi(z); z\in \mathcal{B} ]=\mathbb{E}[\phi(z)\mathds{1}_{z\in \mathcal{B}}] = \int  \phi(z) \mathds{1}_{z\in \mathcal{B}} f(x,z) \dd{x} \dd{z} $$
$$\mathbb{E}[h(X); z\in \mathcal{B} ]  = \int \dd{z} \qty[\int f(x,z) h(x) \dd{x}] \mathds{1}_{z\in \mathcal{B}} = \int \qty[\frac{\int f(x,z) h(x) \dd{x}}{\int f(x,z) \dd{x}}]  \mathds{1}_{z\in \mathcal{B}} f(u,z)\dd{u}\dd{z} = \int  \phi(z) \mathds{1}_{z\in \mathcal{B}} f(x,z) \dd{x} \dd{z} $$
\end{proof}
\end{prop}

Suppose $\mathcal{H}, \mathcal{J} \subseteq \mathcal{F}$ $\sigma$-algebras. We want to regard 
$$X\in \mathcal{H}\mapsto \mathbb{E}\qty[X|\mathcal{J}]$$
as the expectation corresponding to some probability distribution.

$$P(A|\mathcal{J}) = \mathbb{E}\qty[\mathds{1}_A|\mathcal{J}]$$
So $A,\omega \mapsto P(A||\mathcal{J})(\omega)$ is random set function on $\mathcal{H}$. For $\omega$ fixed a.s. is this a probability measure? Generally, no.


\begin{definition}
Let $P(\cdot, \cdot): \mathcal{F}\times \Omega \mapsto [0,1]$
We say that $P$ is a regular conditional probability distribution for $\mathcal{J}$ if
\begin{enumerate}
	\item $\forall F\in \mathcal{F} $, $P(F, \cdot)$ is a version of $\mathbb{E}\qty[\mathds{1}_F|\mathcal{J}]$.
	\item a.e. $\omega\in \Omega $, $P(\cdot, \omega)$ is a probability measure on $\qty(\Omega, \mathcal{F})$.
\end{enumerate}
\end{definition}

\begin{prop}
	
	Let $X_1,\dots, X_k$ be independent random variables. Given $h \in \mathcal{B}(\mathbb{R}^k)$
	$$\gamma_n(x) = \mathbb{E} \qty[h(x,X_2, \dots X_k)]$$
	then
	$$\gamma(X_1) = \mathbb{E} \qty[h(X_1, \dots, X_k) | \sigma(X_1)]$$
	
	\begin{proof}
		If $C = \left\{ h\in \mathcal{B}(\mathbb{R}^k) \text{ s.t. identity holds } \right\}$ then we check that $C$ is monotone class.
		
		Note, for $A = A_1 \times \dots \times A_k$ then
		$$\mathbb{E} \qty[\mathds{1}_{A_1}\dots\mathds{1}_{A_k} | \sigma(X_1)] = \mathds{1}_{A_1} \mathbb{E} \qty[\mathds{1}_{A_2}\dots\mathds{1}_{A_k} | \sigma(X_1)] = \mathds{1}_{A_2} P(A_2\cap \dots \cap A_k) = \gamma_n(x)$$
	\end{proof}
\end{prop}
\paragraph{Example}
Let $\left\{ X_i\right\}_{i=1}^\infty$ i.i.d. random variables and define
$$S_n = \sum_{i=1}^n X_i$$
Let 
$$\mathcal{G}_n = \sigma(S_n, S_{n+1}, \dots ) = = \sigma(S_n, X_{n+1}, \dots )$$

What is
$$\mathbb{E}\qty[X_1|\mathcal{G}_n] = \mathbb{E}\qty[X_1|\sigma(\sigma(S_n), \sigma(X_{n+1}, \dots ))]$$
Since both $X_1$ and$S_n$ is independent on $ \sigma(X_{n+1}, \dots )$ we can rewrite as
$$\mathbb{E}\qty[X_1|\mathcal{G}_n] = \mathbb{E}\qty[X_1|\sigma(\sigma(S_n), \sigma(X_{n+1}, \dots ))]= \mathbb{E}\qty[X_1|\sigma(S_n)]$$

We claim that
$$\mathbb{E} [X_1 | S_n] = \frac{1}{n} S_n$$
since for $i \leq n$
$$\mathbb{E} [X_1 | S_n] = \mathbb{E} [X_i | S_n]$$
Since
$$\mathbb{E}\qty[\mathbb{E}\qty[X_i|S_n]; \mathds{1}_{S_n \in B}] = \mathbb{E}\qty[X_i; \mathds{1}_{S_n \in B}] = \int\limits_{\mathbb{R}^n} x_i \mathds{1}_{B}(x_1+x_2+\dots x_n) \prod_{i=1}^n \dd{\mu(x_i)} = \int\limits_{\mathbb{R}^n} x_1 \mathds{1}_{B}(x_1+x_2+\dots x_n) \prod_{i=1}^n \dd{\mu(x_i)} = \mathbb{E}\qty[X_1; \mathds{1}_{S_n \in B}] = \mathbb{E}\qty[\mathbb{E}\qty[X_1|S_n]; \mathds{1}_{S_n \in B}]$$

thus
$$S_n  =\mathbb{E}[S_n|S_n] = \sum_{k=1}^n \mathbb{E}[X_k|S_n] = n\mathbb{E}\qty[X_1|S_n]$$
\section{Martingales}
\begin{definition}[Filtration]
	$\left\{ \mathcal{F}_i \right\}_{i=0}^\infty$ sequence of $\sigma$-algebras is a filtration if $$\mathcal{F}_0 \subseteq \mathcal{F}_{1} \subseteq \dots \subseteq \mathcal{F}_{\infty} \subseteq \mathcal{F}$$
	where $\mathcal{F}_\infty = \sigma\qty(\bigcup_{i=1}^\infty \mathcal{F}_i)$
\end{definition}

\paragraph{Example}
Let $\left\{  W_i \right\}_{i=0}^\infty$ sequence of random variables (stochastic process). Let $\mathcal{F}_i = \sigma\qty(W_0, \dots W_i)$, then
$\left\{ \mathcal{F}_i \right\}_{i=0}^\infty$ is a filtration.
\begin{definition}[Martingale]
	A sequence $\left\{  X_n \right\}_{n=0}^\infty$ (sub-/super-) martingale with respect to $\left\{ \mathcal{F}_n \right\}_{n=0}^\infty$ if 
	\begin{enumerate}
		\item $X_n \in L^1(\mathcal{F}_n, P)$
		\item $\mathbb{E} \qty[X_{n+1} | \mathcal{F}_n] = X_n$ ($\geq$ for sub- and $\leq$ for super-).
	\end{enumerate}
\end{definition}

We may assume $X_n=0$ since if $X_n$ is martingale, so is $Y_n=X_n-X_0$.
\paragraph{Example} 
Let $X_i$ be i.i.d. and $X_i\geq 0$ with $\mathbb{E}[X_i] = 1$. 
$$M_n  = \prod X_i$$
Then
$$\mathbb{E}\qty[M_{n+1}|\mathcal{F}_n] = \mathbb{E} \qty[\prod_{1}^{n+1} X_i| \mathcal{F}_n] = \mathbb{E} \qty[X_{n+1}|\mathcal{F}_n] \prod X_i = \prod 1 \cdot \prod X_i = M_n$$
\paragraph{Example} 
Consider $\va{X}_i$ i.i.d. vectors in $\mathbb{R}^d$ with natural filtration $\mathcal{F}_n$ and
$$\va{S}_n = \sum \va{X}_i$$
$$\mathbb{E} \qty[\va{S}_{n+1} | \mathcal{F}_n] = \va{S}_n + \mathbb{E} \qty[\va{X}_{n+1} | \mathcal{F}_n]$$
Thus if $\mathbb{E} \qty[\va{X}_{n+1} | \mathcal{F}_n] = 0$, $\va{S}_n$ is martingale.

\paragraph{Example}
Let $d=2$ and $\va{X}_i$ be equiprobable out of $\pm \epsilon \vu{x}$, $\pm\epsilon \vu{y}$.

Given $f \in \mathcal{C}^3\qty(\mathbb{R}^2)$ consider $Z_n = f(S_n^\epsilon)$

\begin{definition}
	$f$ is (sub-/super-) harmonic if $Z_n$ is a (sub-/super-) martingale
\end{definition}

What happens for small $\epsilon$:
\begin{align*}
\mathbb{E} \qty[f(X_{n+1} | \mathcal{F}_n)] = \frac{1}{4} \qty(f(\epsilon \vu{x} + S_n)+f(-\epsilon \vu{x} + S_n)+f(\epsilon \vu{y} + S_n)+f(-\epsilon \vu{y} + S_n))  =\\= f(S_n) + \frac{1}{4}\qty\bigg(\qty\big[f( S_n+\epsilon \vu{x})-f(S_n)]+\qty\big[f(S_n-\epsilon \vu{x})-f(S_n)]+\qty\big[f(\epsilon \vu{y} + S_n)-f(S_n)]+\qty\big[f(-\epsilon \vu{y} + S_n)) -f(S_n)] 
\end{align*}

For Taylor expansion we get
$$f(\epsilon \vu{x}+ \va{S}_n) = \epsilon \pdv{f}{x} + \frac{\epsilon^2}{2} \pdv[2]{f}{x} + \order{\epsilon^3}$$
thus\begin{align*}
\mathbb{E} \qty[f(X_{n+1} | \mathcal{F}_n)] = f(S_n) + \frac{\epsilon^2}{4} \laplacian{f} + \order{\epsilon^3} 
\end{align*}

i.e., we get a condition on the Laplacian of $f$which decides whether $Z_n$ is martingale.

\paragraph{Information exposure martingale}
For some $\left\{ \mathcal{F}_n \right\}_{n=1}^\infty$ and $X \in L^1(\Omega, \mathcal{F}, P)$
\begin{itemize}
	\item $M_n = \mathbb{E} \qty[X|\mathcal{F}_n]$ is a martingale by tower property.
	\item From Jensen $Z_n = \phi(M_n)$ is submartingale if $\phi$ is convex.
\end{itemize}

We'll show that
$$M_n \stackrel{L_1 \text{ a.s.}}{\to} \mathbb{E} \qty[X|\mathcal{F}_\infty]$$
We are interested when $X = \mathbb{E} [X|\mathcal{F}_\infty]$.
\paragraph{Example}
Let $X_n$, $\mathcal{F}_n$ be martingale. $X_n-X_{n-1}$ is an amount we won in $n^{th}$ game.

\begin{definition}
	$C_n$ is predictable if $C_n\in \mathcal{F}_{n-1}$.
\end{definition}
\begin{definition}[Discrete stochastic integral]
	Discrete stochastic integral with respect to $X$ is 
	$$(C \circ X)_n = \sum_{j=1}^n C_j\qty(X_j-X_{j-1})$$
	i.e., total amount won at time $n$ won using strategy using gambling strategy $C$.
\end{definition}

\begin{prop}
	If $X_n-X_{n-1}$ is (super)martingale with respect to $\mathcal{F}_n$ and $C_n$ are bounded and positive (not necessarily uniformly) then so is $C\circ X$.
	\begin{proof}
		$$\mathbb{E} [(C\circ X)_n|\mathcal{F}_{n-1}] = \sum_{j=1}^n \mathbb{E} \qty[C_j(X_j-X_{j-1}) | \mathcal{F}_{n-1}] = (C\circ X)_{n-1} + C_n\mathbb{E} \qty[(X_n-X_{n-1}) | \mathcal{F}_{n-1}] $$
	\end{proof}
\end{prop}

So what is expected winning $\mathbb{E}\qty[(C\circ X)_n]$?
$$\mathbb{E}\qty[(C\circ X)_n] = \mathbb{E}\qty[\mathbb{E}\qty[(C\circ X)_n|\mathcal{F}_{n-1}]]  =  \mathbb{E}\qty[(C\circ X)_{n-1}] $$
by induction $\mathbb{E}\qty[(C\circ X)_n] = \mathbb{E}\qty[(C\circ X)_0] = 0$.

Similarly, if $X$ is a supermartingale and $C\geq 0$,
$$\mathbb{E}\qty[(C\circ X)_n]\leq 0$$


\begin{definition}
$T: \Omega \to \mathbb{N}$ is called stopping time with respect to $\mathcal{F}_n$ if $\forall n \left\{ \omega: T(\omega) \leq n \right\} \in \mathcal{F}_n$. Equivalently is $\left\{ T=n\right\} \in \mathcal{F}_n$.
\end{definition}
Note that $\left\{ T \geq n\right\} \in \mathcal{F}_{n-1}$, since its complementary of $\left\{ T \leq n-1\right\} $.
\begin{definition}
	Given a process $\left\{ X_n \right\}_n$ we say $X_n$ is adapted to $\left\{ \mathcal{F}_n \right\}_n$ if $\forall n \quad X_n\in \mathcal{F}_n$.
\end{definition}
\begin{definition}
Given an adapted process $\left\{ X_n \right\}_n$ and stopping time $T$the stopped process $$ X_n^{(T)} = X_{\min\left\{ T,n \right\}} $$
\end{definition}

\begin{lemma}
	If $X_n$ is (super-/sub-)martingale, so is $X_n^{(T)}$.
	\begin{proof}
		Let $C_n = \mathds{1}_{\left\{ T\geq n \right\}}$ predictable, then
		$$\qty(C \circ X)_n = \sum_{k\leq n} C_k\qty(X_k -X_{k-1}) = \sum_{1\leq k\leq n} \mathds{1}_{T\geq k} (X_k-X_{k-1}) = X_{\min\left\{ T,n \right\}} - X_0$$
		
		By we already know that $\qty(C \circ X)_n$ preserve martingale property.
	\end{proof}
\end{lemma}

\begin{theorem}
	Let $X_n$ be a supermartingale, then $\forall n$ $\mathbb{E} \qty[X_{\min\left\{ T,n \right\}}] \leq \mathbb{E} \qty[X_0]$.
\end{theorem}

Would the property survive under $n\to \infty$? No!

\paragraph{Example}
Let $X_n$ be a SRW on $\mathbb{Z}$ starting from $0$.
$$T_1 = \inf \left\{ n: X_n=1 \right\}$$

By theorem, $\mathbb{E} \qty[X_{\min\left\{ T,n \right\}}] = 0$. On the other hand, since $P(T<\infty) = 1$, $\mathbb{E} \qty[X_T]  =1$.

The problem is $T$ doesn't have expectation.

\begin{theorem}[Doob's optional sampling theorem]
	Let $T$ be a stopping time and let $X_n$ be a supermartingale. If one of the following holds
	\begin{enumerate}
		\item $T$ is bounded
		\item $X$ is bounded
		\item $\mathbb{E} < \infty $ and $\abs{X_n - X_{n-1}} \leq K$ $\forall n$ $P$ a.s. 
	\end{enumerate}
then $\mathbb{E} [X_T] \leq \mathbb{E}\qty[X_0]$.
\begin{proof}
	\begin{enumerate}
		\item $$\mathbb{E} \qty[X_0] \geq \mathbb{E} \qty[X_{\min\left\{ T,n \right\}}]$$
		Since
		$$X_{\min\left\{ T,n \right\}} \to X_T$$
		a.s. since $T$ is bounded, $\abs{X_{\min\left\{ T,n \right\}} } \leq \max \abs{X_k}$ by dominated convergence theorem, 
		$$\mathbb{E} \qty[X_{\min\left\{ T,n \right\}}] \to \mathbb{E}\qty[X_T]$$
		
		
		\item  Bounded convergence
		
		
		\item $$\abs{X_{\min\left\{ T,n \right\}}} \leq K\cdot\min\left\{ T,n \right\} \leq KT$$
			By DCT, 
			$$\mathbb{E} \qty[X_{\min\left\{ T,n \right\}}] \to \mathbb{E}\qty[X_T]$$
		
	\end{enumerate}
\end{proof}

\begin{coll}
	If $M$ is a martingale and $\abs{M_n-M_{n-1}} \leq K$ and $C$ is predictable and $\abs{C}\leq K$ and $\mathbb{E} \qty[T] <\infty$ then $\mathbb{E} \qty[C \circ X] = 0$
\end{coll}

\begin{coll}
	If $X$ is a positive supermartingale and $T<\infty$ a.s. then
	$$\mathbb{E}[X_T] \leq \mathbb{E} \qty[X_0]$$
	\begin{proof}
		Fatou lemma
	\end{proof}
\end{coll}
\end{theorem}

\begin{lemma}
	Let $T$ be a stop time such that $\exists k > 0$ so that $\exists \epsilon>0$ $\forall n>0$
	$$P\qty(T \leq n+k | \mathcal{F}_n) > \epsilon$$
	Then $\mathbb{E} [T]<\infty$
	\begin{proof}
		$$\mathbb{E} \qty[T] = \sum_j P(T\geq j)$$
		
		Consider $J$, 
		$$P(T\geq kJ) = \mathbb{E} \qty\bigg[\mathbb{E} \qty[\mathds{1}_{T\geq kJ}\mathds{1}_{T\geq k(J-1)}|\mathcal{F}_{k(J-1)}]] = \mathbb{E} \qty\bigg[\mathds{1}_{T\geq k(J-1)} \underbrace{\mathbb{E} \qty[\mathds{1}_{T\geq kJ}|\mathcal{F}_{k(J-1)}]}_{\leq 1-P(T\leq kJ|\mathcal{F}_{k(J-1)}) = 1-\epsilon}] \leq (1-\epsilon) P(T\geq k(J-1)) \leq (1-\epsilon)^J$$
		Thus $P(T\geq kJ)$ decays exponentially.
\end{proof}
\end{lemma}
\subsection{Markov chains}

Let $\left\{ X_n \right\}$ be a stochastic process taking values in $\qty(E, \mathcal{E})$ and $\left\{ \mathcal{F}_n\right\}$ be a filtration such that $X_n\in \mathcal{F}_n$. 
\begin{definition}
	$p: E\times \mathcal{E} \to [0,1]$ is a transition kernel on $E$ if 
	\begin{enumerate}
		\item $\forall e\in E$ $P(e,\cdot)$ is a probability measure.
		\item $\forall A\in \mathcal{E}$ $p(\cdot, A) \in \mathcal{E}$
	\end{enumerate}
\end{definition}
\begin{definition}
$\left\{ X_n \right\}$ is a Markov chain with respect to $\mathcal{F}_n$ with transition kernel $p$ if 
$\forall A\in \mathcal{E}$
$$P(X_{n+1} \in A | \mathcal{F}_n) = p(X_n,A)$$

\end{definition}
We can acquire $$\mathbb{E}\qty[h(X_{n+1}|\mathcal{F}_n)] = \int\limits_E p(X_n, \dd{e}) h(e)$$


Given $(E, \mathcal{E}, p)$ does $\exists X_n$, $\mathcal{F}_n$ Markov with kernel $p$? The answer is yes.

Let $\Omega = E^{\mathbb{N}}$, $\mathcal{F} = E^{\bigotimes \mathbb{E}}$ and $\mathcal{F}_n = \sigma\qty(X_i(\omega) : i\leq n)$. We want
$X(\omega) = \omega_n$. 

 We want to define law of $\left\{ X_n \right\}$ by first specifying 
 \begin{align*}
 P(A_0 \times A_1\times \dots A_n \times E \times E \times \dots) = \mathbb{E} \qty[\prod_{i=0}^n \mathds{1}_{X_i\in A_i}] = \mathbb{E} \qty[\prod_{i=0}^{n-1} \mathds{1}_{X_i\in A_i} \mathbb{E} \qty[\mathds{1}_{X_n\in A_n} | \mathcal{F}_{n-1}]] = \mathbb{E} \qty[\prod_{i=0}^{n-1} \mathds{1}_{X_i\in A_i} p(X_{n-1}, A_n)] =\\= \mathbb{E} \qty[\prod_{i=0}^{n-2} \mathds{1}_{X_i\in A_i} \int\limits_{A_{n-1}} p(X_{n-2}, \dd{e})p(e, A_n)] =\dots = \mathbb{E} \qty[\mathds{1}_{X_0\in A_0} \int\limits_{A_1} \dots \int\limits_{A_n} p(X_0, \dd{e_1})p(e_1, \dd{e_2})\dots p(e_{n-1}, \dd{e_n})]
 \end{align*}

Let law of $X_0$ be $\mu$ on $E$. Then
$$P(A_0 \times A_1\times \dots A_n \times E \times E \times \dots) = \int\limits_{A_0} \mu(\dd{e_0}) \int\limits_{A_1}  p(X_0, \dd{e_1})  \int\limits_{A_2}  p(e_1, \dd{e_2}) \dots \int\limits_{A_n}\dots p(e_{n-1}, \dd{e_n}) $$

From now on we'll assume $E$ is either finite or countable. In this case, Markov condition is $\exists p(i,j)$ such that
$$P(X_{n+1} =j | \mathcal{F}_n) = p(X_n,j)$$

And then
$$\mathbb{E} [h(X_{n+1})|\mathcal{F}_n] = \sum_{j\in E} P(X_n, j)h(j) = p \vdot j$$
where $p$ is matrix and $h$ is a vector.

\begin{definition}
	$h$ is called $p$-superharmonic if $p\vdot h \leq h$ or alternatively, if
	$Y_n = h(X_n)$ is a p-supermartingale.
\end{definition}
\begin{definition}
	Let $T_i = \inf \left\{ n\geq 1: X_n=i \right\}$.
\end{definition}
\begin{definition}
	We say $X_n$ is irreducible if $P_i(T_j<\infty) > 0$ where $P_i$ is a law of $X_n$ started with $X_0=i$ with probability $i$.
\end{definition}
\begin{definition}
We say $X_n$ is irreducible recurrent if $\forall i,j \in E \quad P_i(T_j<\infty) = 1$.
\end{definition}
\begin{theorem}
	$\left\{X_n\right\}$ is irreducible recurrent on $E$ iff all positive superharmonic are constant.
	\begin{proof}
		$\Rightarrow:$
		Let $\left\{X_n\right\}$ is irreducible recurrent on $E$ and $h$ a positive $p$-superharmonic function. Consider $$\mathbb{E}_i \qty[h\qty(X_n^{T_j})] \leq h(i)$$
		Thus by Fatou
		$$\mathbb{E}_i \qty[\liminf h\qty(X_n^{T_j})] \leq h(i)$$
		and now $h(i) \leq h(j)$. By symmetry, $h(j) = h(i)$.
	\end{proof}
\end{theorem}

How to produce $p$-harmonic functions? Let $A\subseteq E$ be some set and $g: A\to \mathbb{R}$ be a bounded function. Assume $\forall i P_i(T_A <\infty) = 1$ and let $h(i) = \mathbb{E}_i \qty[g(X_{T_A})]$.

\begin{lemma}
	$h$ is $p$-harmonic on $A^c$.
	\begin{proof}
		For $i\notin A$, $T_A\geq 1$.
		
		$$P(X_{n+1}\in A | \mathcal{F}_n) = p(X_n, A)$$
		
		$$\forall i \quad P_i(X_{n+1}\in A_i, X_{n+2}\in A_2 | \mathcal{F}_n)  =\int\limits_{A_1} p(X_n, \dd{e_1})\int\limits_{A_2} p(X_n, \dd{e_2}) = P(\hat{X}_1\in A_1, \hat{X}_2\in A_2)$$
		where $\hat{X}_i = X_{n+i}$.
		
		Moreover 
		$$P(X_{n+i} \in F|\mathcal{F}_i) = P_{X_n} (\hat{X}_i \in F)$$
		
		Thus
		$$\mathbb{E} \qty[g(X_{T_A})|\mathcal{F}_1] = \mathbb{E}_{X_1} \qty[g(X_{T_A})] = h(X_1)$$
		and
		
		$$h(i) = \mathbb{E}_i \qty[\mathbb{E} \qty[g(X_{T_A})|\mathcal{F}_1]] = \mathbb{E}_i [h(X_1)] = \sum_j p_{ij} h_j = p\cdot h$$
	\end{proof}
\end{lemma}

\begin{theorem}[Martingale convegrence] \label{mart_conv}
	$X$ is a supermartingale, $\sup\limits_n \mathbb{E} [X_n] <\infty$ then $\lim_{n\to\infty} X_n = X_\infty$ exists a.s. (but not neccesary in $L^1$)
\end{theorem}
\begin{definition}[Uniform integrability]
	A collection $C$ is uniformly integrable if $\forall \epsilon>0$ $\exists K$ such that $\forall X\in C$ $\mathbb{E} [\abs{X} ; \abs{X}>K] <\epsilon$.
\end{definition}

Example
$$X \in L^1 \Rightarrow C = \left\{ \mathbb{E} [X|\mathcal{J}] : \mathcal{J}\subseteq \mathcal{F} \text{ subalgebra} \right\}$$
\begin{lemma}
	$X\in L^1$ then $\forall \epsilon > 0$ $\exists \delta$ such that $P(F)<\delta$ then $\mathbb{E} \qty[\abs{X} ; F] < \epsilon$. 

\end{lemma}

\begin{theorem}
	$X_n \substack{L^1}{\to} X \iff$
	\begin{enumerate}
		\item $\forall \epsilon>0$ $P(\abs{X}_n-X >\epsilon) \to 0$
		\item $\left\{  X_n \right\}, X$ are uniformly integrable.
	\end{enumerate}

\begin{proof}
	
	$\Rightarrow$:
	
	From Markov we get
	$$\epsilon P(\abs{X_n-X}<\epsilon) \leq \mathbb{E} \qty[\abs{X_n-X}]$$

	Now we want to show uniform integrability:
	$$\mathbb{E} \qty[\abs{X_n -X+X }; \abs{X}_n> K] \leq \mathbb{E} \qty[\abs{X_n -X}; \abs{X_n}>K]   +\mathbb{E} \qty[\abs{X}; \abs{X_n}> K]  \leq \epsilon +  \mathbb{E} \qty[\abs{X}; \abs{X_n}> K] $$
	
	Choose $K(\delta)$ such that $\sup\limits_n \mathbb{E} \abs{X_n} < \delta(\epsilon)$ use lemma for $X$ to say $\mathbb{E} \qty[\abs{X} ; \abs{X_n}>K]<\epsilon$
	
	Says $\forall \epsilon>0$ $\exists n(\epsilon), K(\epsilon) $ such that $\mathbb{E} \qty[\abs{X_n} ; \abs{X_n}>K] <\epsilon$.
	$\Leftarrow$:
\end{proof}
\end{theorem}

\begin{theorem}
	Let $M_n$ be u.i. martingale. Then $\lim_{n\to\infty} M_n = M_\infty$ a.s. and in $L_1$.
	\begin{proof}
		A.s. convergence follows from \ref{mart_conv}. Since convergence in probability follows from a.s. convergence, we get $L_1$ convergence.
	\end{proof}
\end{theorem}

\begin{theorem}[Levy's upwards theorem]
	Let $\mathcal{F}_n$ filtration $\eta \in L^1$ and $\mathcal{F}_\infty = \sigma(\bigcup_n \mathcal{F}_n)$.
	
	$M_n = \mathbb{E} \qty[\eta | \mathcal{F}_n]$ is a u.i. martingale and 
	$M_n \to M_\infty$ a.s. and $L^1$ and moreover, $M_\infty = \mathbb{E} \qty[\eta|\mathcal{F}_\infty ]$
	
	\begin{proof}
		If $F\in \mathcal{F}_n$ $\forall r>n$:
		$$\mathbb{E} \qty[M_r ; F ] = \mathbb{E} \qty[ \mathbb{E} \qty[M_r |\mathcal{F}_n] ; F] = \mathbb{E} \qty[M_n; F]$$
		
		Thus $$\mathbb{E} [M_\infty ; F] = \mathbb{E} [M_n; F] = \mathbb{E} [\mathbb{E} \qty[\eta | \mathcal{F}_\infty]; F] $$
		
		So 
		$$F\mapsto \mathbb{E} [M_\infty ; F]$$
		$$ F\mapsto\mathbb{E} [\mathbb{E} \qty[\eta | \mathcal{F}_\infty]; F]$$
		agree on $\bigcup \mathcal{F})n$ by $\pi$-$\lambda$ they agree on $\mathcal{F}_\infty$,
	\end{proof}
\end{theorem}

\begin{definition}[Backward filtration]
	$\mathcal{F}_n$ is a backward filtration if $\mathcal{F}_n \supset \mathcal{F}_{n+1}$
\end{definition}
\begin{definition}[Backward martingale]
$$\mathbb{E} \qty[M_n | \mathcal{F}_{n+1}] = M_{n+1}$$
\end{definition}

\begin{theorem}[Levy's downwards theorem]
	Let $\mathcal{F}_n$ backward filtration $\eta \in L^1$ and $\mathcal{F}_\infty = \bigcap_n \mathcal{F}_n$
	$M_n = \mathbb{E} \qty[\eta | \mathcal{F}_n]$ is a u.i. martingale and 
	$M_n \to M_\infty$ a.s. and $L^1$ and moreover, $M_\infty = \mathbb{E} \qty[\eta|\mathcal{F}_\infty ]$
	
	\begin{proof}
		Recheck upcrossing inequality works for backward martingale.
	\end{proof}
\end{theorem}

\begin{theorem}[Kolmogorov's zero-one law]
	Let $\left\{X_i\right\}$ independent and $\mathcal{T} = \bigcap_n \sigma(X_n,\dots)$ 
	$$\forall F\in \mathcal{T} \quad P(F) \in \left\{0,1\right\}$$
	\begin{proof}
		Let $\mathcal{F}_n = \sigma(X_1, \dots, X_n)$ and $\eta = 1_F$ for $F\in \mathcal{T}$.
		$$P(F) = \mathbb{E} \qty[\eta | \mathcal{F}_n]$$
		From Levy upward $$\lim_{n\to\infty}  \mathbb{E} \qty[\eta | \mathcal{F}_n] = \mathbb{E} \qty[\eta | \mathcal{F}_\infty]= \eta = \mathbb{1}_F$$
		i.e. it's either $0$ or $1$.
	\end{proof}
\end{theorem}


\begin{theorem}[Strong law of large numbers]
	$X_i$ i.i.d. with $\mathbb{E} [\abs{X_n}] <\infty$.
	$$\frac{S_n}{n} \to \mathbb{E} \qty[X_1]$$
		\begin{proof}	
	Let $\mathcal{F}_n = \sigma(S_n, S_{n+1}, \dots)$.
	$$\frac{S_n}{n} = \mathbb{E} \qty[\frac{S_n}{n} | \mathcal{F}_n] = \frac{\sum \mathbb{E} \qty[X_i |\mathcal{F}_n] }{n} =  \mathbb{E} \qty[X_1 | \mathcal{F}_n]$$
	We've shown that $\mathbb{E} \qty[X_1 | \mathcal{F}_n] = \frac{S_n}{n}$. Therefore by Levy downward
	$$\frac{S_n}{n} \stackrel{\text{a.s.}, L^1}{\to} A_\infty =  \mathbb{E} \qty[X_1 | \mathcal{F}_\infty]$$
	
	Since both $\limsup \frac{S_n}{n} \in \mathcal{T}$, $\liminf \frac{S_n}{n} \in \mathcal{T}$ we get $\mathbb{E} \qty[X|\mathcal{F}_\infty] \in \mathcal{T}$. By Kolmogorov 0-1 $\mathbb{E} \qty[X|\mathcal{F}_\infty]$ is constant a.s. and thus equals $\mathbb{E} [X_1]$.
	
\end{proof}
\end{theorem}


Let $M_n$ be a martingale with respect to $\mathcal{F}_n$, $X_n=M_n-_{n+1}$ and $q_n =\mathbb{E}\qty[X^2 | \mathcal{F}_{n-1}]$.
\begin{lemma}
	$$\forall r<s\leq t< u \quad \mathbb{E} \qty[(M_u-M_t)(M_s-M_r)] = 0$$
	$$\mathbb{E}\qty[(M_u-M_t)^2] = \sum_{k=t+1}^u \mathbb{E}[X^2_k]$$
	\begin{proof}
		$$\mathbb{E} \qty[M_u-M_t|\mathcal{F}_t] = 0$$
		$$\mathbb{E} \qty[(M_u-M_t)(M_s-M_r)|\mathcal{F}_t] = (M_s-M_r)\mathbb{E} \qty[(M_u-M_t)|\mathcal{F}_t]$$
		and
		$$(M_u-M_t)^2 = \qty(\sum_{k=t+1}^u X_k)^2 = \sum_{k,l=t+1}^u X_kX_l$$
	\end{proof}
\end{lemma}

\begin{prop}
	Let $N_t = M_t^2 - \sum_{i\leq t} q_i$ then $N_t$ is a martingale.
\begin{proof}
	$$N_{t+1} -N_t = M_{t+1}^2-M_{t}^2 - q_{t+1}$$
	$$\expval{N_{t+1}-N_t|\mathcal{F}_t} = \expval{M_{t+1}^2-M_t^2|\mathcal{F}_t} =  - q_{t+1} $$
	But
	$$M_{t+1}^2 = M_t^2 +(M_{t+1}-M_t)^2+2M_t(M_{t+1}-M_t)$$
\end{proof}
\end{prop}

\begin{coll}
	$$\sup\limits_n \mathbb{E} [M^2_n] < \infty \iff \sum_k \mathbb{E} \qty[\qty(M_{t+1} - M_t)^2] <\infty$$
	\begin{proof}
		$$\mathbb{E} (M_{n+k}-M_n)^2 = \sum_{j=n}^{n+k} \mathbb{E} [\qty(M_{j+1}-M_j)^2] \leq \sum_{j=n}^{\infty} \mathbb{E} [\qty(M_{j+1}-M_j)^2]$$
		By Fatou
		$$\mathbb{E} (M_{\infty} - M_n)^2 \leq \sum_{j=n}^\infty \mathbb{E} [\qty(M_{j+1}-M_j)^2]$$
		
		Moreover by the equality at start of the proof we get
		$$\mathbb{E} (M_\infty - M_n)^2 = \sum_{j=n}^\infty \mathbb{E} [\qty(M_{j+1}-M_j)^2] $$
		Also if $M-0=0$
		
		$$\mathbb{E} M_\infty^2 = \sum_{j=n}^\infty \mathbb{E} [X)^2] $$
	\end{proof}
\end{coll}

\begin{theorem}
	Let $X_k$ be independent random variable such that $\mathbb{E} [X_k] = 0$ and $\sigma^2_k < \infty$. Let $M_n = \sum X_k$
	\begin{enumerate}
		\item If $\sum \sigma_k^2 <\infty $, $M_n\to M_\infty$ a.s. and $L^2$.
		\item If $\abs{X_i} \leq K < \infty $, then $\sum X_i $ converges a.s.
	\end{enumerate} 

\begin{proof}
	Since $\mathbb{E} [X_i] = 0$
	$$\mathbb{E} [M_n|\mathcal{F}_{n-1}] = M_{n-1} + \mathbb{E}[X_n | \mathcal{F}_{n-1}]$$
	By independence, $M_n$ is $\mathcal{F}_n$ martingale.
	
	
	Recall $N_t$ and note $q_{k+1} = \mathbb{E} [X_{k+1}] <\infty$.
	
	Define
	$T_c = \inf \left\{  i: \abs{M_i} \geq c \right\}$
	$$0=\mathbb{E} \qty[N_{\min \left\{ T_c, n \right\}}] = \mathbb{E} [M^2_{\min \left\{ T_c, n \right\}}] - \mathbb{E} \qty[\sum_{k=1}^{\min \left\{ T_c, n \right\}} \sigma_k^2] $$
	Thus
	$$\mathbb{E} \qty[\sum_{k=1}^{\min \left\{ T_c, n \right\}} \sigma_k^2] = \mathbb{E} [M^2_{\min \left\{ T_c, n \right\}}] \leq (c+k)^2 $$
	By Fatou
	$$\mathbb{E} \qty[\sum_{k=1}^{T_c} \sigma_k^2]\leq (c+k)^2 $$
	If $\exists c<\infty$ such that $P(T_c=\infty) = 0$ we are done.
	
	By assumption of a.s. convergence, $M_i \to  M_\infty $ thus $\exists c $ such that $P(\abs{M_i} <c) > 0$.
\end{proof}

\end{theorem}


\begin{lemma}
	If $\abs{X_i} \leq K$, $\sum X_i $ converges, then $\sum \mathbb{E} [X_i]$ converges and $\sum \sigma_i^2<\infty$
	\begin{proof}
		Let $X^*_i=X_i$, then both series converges, and we define $Y_i = X_i - X_i^*$ for which $\mathbb{E} [Y_i] = 0$.
		
		Thus, $\sum \sigma^2(Y_i) <\infty$ and thus $2\sum \sigma^2(X_i) <\infty$. Now from part one, $Z_i = X_i -\mathbb{E} [X_i]$, we get
		$\sum X_i -\mathbb{E} [X_i] $ converges a.s. and so does $\sum \mathbb{E} [X_i]$
	\end{proof}
\end{lemma}



\begin{lemma}[Cesaro's lemma]
	Let $b_n$ be strictly increasing and positive and $b_0 = 0$. Let $\left\{ V_n\right\}$ be a convergent sequence $V_n\to V_\infty$. Then
	$$\frac{1}{b_n} \sum_{k=1}^n (b_k - b_{k-1})V_k \to V_\infty$$
	\begin{proof}
		$$1 = \sum_{k=1}^n \frac{b_k-b_{k-1}}{b_n} $$
		$$V_\infty = \sum_{k=1}^n \frac{b_k-b_{k-1}}{b_n} V_\infty $$
		$$\abs{V_\infty - \frac{1}{b_n} \sum_{k=1}^n (b_k-b_{k-1}) V_k} \leq \frac{1}{b_n} \sum_{k=1}^n (b_k-b_{k-1}) \abs{V_k-V_\infty}$$
	\end{proof}
For $\epsilon>0$ choose $N$ such that $\abs{V_n-V_\infty} < \epsilon$, let $V_k, V_\infty < M$:
$$\frac{1}{b_n} \sum_{k=1}^n (b_k-b_{k-1}) \abs{V_k-V_\infty} \leq 2M\cdot \frac{b_N}{b_n} + \epsilon \leq 2 \epsilon$$
\end{lemma}

\begin{lemma}[Kroniker's lemma]
	Let $\left\{ b_n\right\}$ be a sequence increasing to $\infty$ and $S-n = \sum_{i=1}^n x_i$. Then if
	$$\sum \frac{x_n}{b_n}$$
	converges, $\frac{S_n}{b_n} \to 0$.
	
	\begin{proof}
		Let $u_n=\sum_{k=1}^n \frac{x_k}{b_k}$, then $u_n-u_{n-1} = \frac{x_n}{b_n}$ and $u_n \to u_\infty = \sum_{k=1}^\infty \frac{x_k}{b_k}$. Then
		$x_n = b_n(u_n-u_{n-1})$ and
		$$S_n = \sum_{k=1}^n x_k = \sum_{k=1}^\infty b_k(u_k-u_{k-1}) = b_nu_n - \sum_{k=1}^n (b_k-b_{k-1} ) u_{k-1}$$
		$$\frac{S_n}{b_n} = u_n - \sum_{k=1}^n \qty(\frac{b_k-b_{k-1}}{b_n})u_{k-1}$$
		By Cesaro we get the required.
	\end{proof}
\end{lemma}
\begin{lemma}
	Let $\mathbb{E}[W_i] = 0$ $\sum_k \frac{\mathbb{E}[W_k^2]}{k^2} <\infty$.
Then $\frac{\sum_{k=1}^n W_i}{n} \to 0$
\begin{proof}
	By Kroniker's lemma it's enough to show that $\frac{W_k}{k}$ converges. By previous discussion of random series this follows from $\sum_i \frac{\mathbb{E} W_k^2}{k^2} < \infty$.
\end{proof}
\end{lemma}

\begin{theorem}[Kolmogorov 3 series theorem]
	Let $\left\{ X_i \right\}_{i=1}^\infty$ be independent random variables. Then $\sum_{i=1}^\infty X_i$ converges iff exists $K$ such that
	\begin{enumerate}
		\item $\sum P(\abs{X_i}\geq K) <\infty$
		\item $\sum_i \mathbb{E} \qty[X_i^K]$ convergent
		\item $\sum_i \sigma^2( \qty[X_i^K])$ convergent
		
	\end{enumerate}

\begin{proof}
	If $\sum X_i$ converges, $\lim \abs{X_i} = 0$ a.s., thus $\forall i>I(k)$ sufficiently large $\sum_i X_i^k$ converges a.s. 
	Since $\sum \sigma^2(X_i^k) <\infty$ by a previous lemma $\sum_i (X_i^k - \mathbb{E} X_i^K)$ converges a.s. But also by second part   $\sum_i \mathbb{E} \qty[X_i^K]$ convergent thus  $\sum_{i=1}^\infty X_i$ converges
	
	
	$$Y_i = X \cdot \mathds{1}_{\abs{X} \leq i}$$
	$$\mathbb{E}[Y_i] = \mathbb{E}[X\mathds{1}_{\abs{X} \leq i}]$$
	$$\lim \mathbb{E} [Y_i] = \mathbb{E} [X]$$
	
	
	$$P(X_i\neq Y_i) = P(\abs{X_i}\geq i) = P(\abs{X}\geq i) $$
	So $$\sum P(X_i \neq Y_i) = \sum_i P(\abs{X} \geq i) \leq \mathbb{E} [X ] < \infty$$
	
	$$\frac{\sigma^2(Y_i)}{i^2} = \frac{\sigma^2(Y_i)}{i^2} \leq \frac{\mathbb{E} [X^2 \leq i]}{i^2}$$
	By MCT
	$$\sum_i \frac{\sigma^2(Y_i)}{i} = \mathbb{E} \qty[X^2 \sum \frac{\mathds{1}_{\abs{X} \leq i}}{i^2}]$$
	$$ \sum \frac{\mathds{1}_{\abs{X} \leq i}}{i^2} \leq C \frac{1}{1+\abs{X}}$$
	So
	
	$$\sum_i \frac{\sigma^2(Y_i)}{i} =C \mathbb{E} \qty[ \frac{X^2 }{1+\abs{X}}] < \infty$$
\end{proof}
\end{theorem}


\begin{theorem}[SLLN]
	Let $\left\{  X_i\right\}_{i=1}^\infty$ be iid $X\in L^1$. Let $S_n = \sum_{i=1}^n X_i$ then $\frac{S_n}{n} \stackrel{a.s.}{\to} \mu = \mathbb{E}[X]$
	\begin{proof}
		Let $Y_i = X \cdot \mathds{1}_{\abs{X} \leq i}$. Then we have $\frac{1}{n} \sum Y_i   \stackrel{a.s.}{\to} \mu$
	\end{proof}
\end{theorem}


\section{Weak convergence and CLT}
\paragraph{Modes of convergence}
\begin{enumerate}
	\item $X_n\stackrel{a.s.}{\to} X$
	\item $X_n\stackrel{prob.}{\to} X$
	\item $X_n\stackrel{dist.}{\to} X$
\end{enumerate}
\begin{definition}
	$\mathcal{S}$ is called Polish space if it is complete, separable metric space.
\end{definition}
\begin{definition}
	Let $\mu_n$ and $\mu$ be measures on $(\mathcal{S}, \mathcal{B})$. We say $\mu_n \stackrel{d}{\to} \mu$ if $\forall f \in C_b(\mathcal{S})$:
	$$\int\limits_{\mathcal{S}} f(s) \dd{\mu_n(s)} \to \int\limits_{\mathcal{S}} f(s) \dd{\mu(s)}$$
\end{definition}

\paragraph{Reminder}
$C_b(\mathcal{S})$ is a Banach space with $\norm{f}_\infty = \sup_{s\in \mathcal{S}}$.
$C_b^*(S)$ is dual space of signed measures and $P(S) \subset C_b^*(S)$ is a closed subspace.
\begin{prop}
	If $X_n\sim \mu_n$ and $X_n \to X$ a.s., then $\mu_n \to \mu$ weakly.
	\begin{proof}
		Let $f\in C_b(\mathbb{R})$
		$$f(X_n) \to f(X) $$
		a.s. and by BCT
		$$\mathbb{E} \qty[f(X_n)] = \mathbb{E} [f(X)]$$
	\end{proof}

Also the proposition works for convergence in probability.
\end{prop}

We might guess that $F_n(x) \to F(x)$ pointwise is equivalent to weak convergence.

\begin{theorem}
	If $\mu_n \stackrel{w}{\to} \mu$ then $\forall x \: F_n(x) \to F(x) $ in continuity points of $F$.
	
	\begin{proof}
		We want
		$$\mathbb{E}_{\mu_n} \mathds{1}_{(-\infty, x) } \to \mathbb{E}_{\mu} \mathds{1}_{(-\infty, x) }$$
		$$\forall \delta \in \mathbb{R} \: \mu_n(f_\delta) \to \mu(f_\delta)$$
		$$\mu(f_\delta(x)) \geq \limsup \mu_n(\mathds{1}_{(-\infty, x)})$$
		also $\mu(f_\delta ) \leq \mu_n(\mathds{1}_{(-\infty, x+\delta)}) = F(x+\delta)$
		$$\limsup F_n(x) \leq \liminf F(x+\delta)=F(x)$$
		we can do the same with negative $\delta$ if $F$ is continuous, and thus
		$$\mu_n \stackrel{w}{\to} \mu$$
	\end{proof}
\end{theorem}

\begin{definition}
	Given $\qty(\mathcal{S}_1, \mathcal{B}_1, P_1)$ and $\qty(\mathcal{S}_2, \mathcal{B}_2, P_2)$ a measure $Q$ on $\mathbb{Q}$ is called a coupling of $P_1$, $P_2$ if letting $X_i(S_1,S_2) = S_i$ the distribution of $X_i$ under $Q$ is $P_i$.
\end{definition}
\begin{theorem}[Skorohod representation theorem]
	
	Let $\mu_n$, $n \in \mathbb{N}$ be a sequence of probability measures on a metric space $S$ such that $\mu_n$ converges weakly to some probability measure $\mu_\infty$ on $S$ as $n \to \infty$.  Suppose also that the support of $\mu_\infty$ is separable. Then there exist random variables $X_n$ defined on a common probability space $(\Omega,\mathcal{F},\mathbf{P})$ such that the law of $X_n$ is $\mu_n$ for all $n$ (including $n=\infty$) and such that $X_n$ converges to $X_\infty$, $\mathbf{P}$-almost surely.
\end{theorem}

\begin{theorem}[Helly's selection theorem]
\end{theorem}

\subsection{Characteristic function}

\begin{definition}
	Given $\mu \sim X$ then $\phi_\mu(\theta) = \int e^{i\theta x} \dd{\mu(x)} = \mathbb{E} [e^{i\theta X}]$.
\end{definition}


\begin{itemize}
	\item $\phi(0)=1$
	\item $\abs{\phi(\theta)} \leq 1$
	\item $\phi(\theta)$ is continuous in $\theta$
	\item $\phi_X(-\theta)  = \overline{\phi_X(\theta)}$
	\item $\phi_{aX+b}(\theta)  = e^{i\theta b} \phi_X(a\theta)$
\end{itemize}

\begin{theorem}
	$$\phi_{X+Y}(\theta) = \phi_X(\theta)\phi_Y(\theta)$$
\end{theorem}
\begin{theorem}
	$X\sim \mu \mapsto \phi_{X}(\theta)$ is one-to-one and invertible.
\end{theorem}
\begin{theorem}
$X_n \stackrel{w}{\to} X$ then $\phi_{X_n} \to \phi_X(\theta)$.
\end{theorem}
\begin{theorem}[L\'{e}vy Inversion]
	Let $a<b \in \mathbb{R}$. Then
	$$\lim_{T\to \infty} \frac{1}{2\pi}\int_{-T}^{T}  \frac{e^{-i\theta a}-e^{-i\theta b}}{i\theta} \phi_X(\theta) \dd{\theta} = \frac{\mu(\left\{a\right\})}{2} + \mu((a,b)) + \frac{\mu(\left\{b\right\})}{2} = \frac{1}{2} [F(b) + F(b^-)] - \frac{1}{2} [F(a) + F(a^-)]$$ 
	\begin{proof}
		$$\frac{e^{-i\theta a} - e^{-i\theta b}}{i\theta} = \int_a^b e^{-i\theta \lambda} \dd{\lambda}$$
		\begin{align*}
		 \frac{1}{2\pi}\int_{-T}^{T}  \frac{e^{-i\theta a}-e^{-i\theta b}}{i\theta} \phi_X(\theta) \dd{\theta} =  \frac{1}{2\pi}\int_{-T}^{T}  \qty[\int_a^b e^{-i\theta \lambda} \dd{\lambda}] \phi_X(\theta) \dd{\theta}= \int_a^b \mathbb{E} \qty[ \frac{1}{2\pi}\int_{-T}^{T}  e^{i\theta (X-\lambda)}  \dd{\theta}] \dd{\lambda} =\\=
		  \mathbb{E} \qty[ \frac{1}{2\pi} \int_a^b \int_{-T}^{T}  \frac{e^{iT (X-\lambda)} - e^{-iT (X-\lambda)} }{2i(X-\lambda)T}  \dd{\lambda T}]  = \mathbb{E} \qty[\frac{1}{\pi} \int_{aT}^{bT} \frac{\sin(XT-u)}{XT-u} \dd{u}] = \mathbb{E} \qty[\frac{1}{\pi} \int_{(a-X)T}^{(b-X)T} \frac{\sin(v)}{v} \dd{v}]
		\end{align*}
		$$\lim_{T\to \infty} \mathbb{E} \qty[\frac{1}{\pi} \int_{(a-X)T}^{(b-X)T} \frac{\sin(v)}{v} \dd{v}]  = \frac{1}{2} \chi_{a-X=0}+ \frac{1}{2} \chi_{a-X<0<b-X}}+ \frac{1}{2} \chi_{0=b-X}$$
		By DCT it converges.	
\end{proof}
\end{theorem}



\begin{theorem}[CLT]
	Let $X_i$ be iid with mean $0$ and variance $1$. Let $S_n = \sum_n X_i$ then
	$$\frac{S_n}{\sqrt{n}} \stackrel{w.}{\to} N(0,1)$$
	\begin{proof}
		$$\phi_{\frac{S_n}{\sqrt{n}}} = \qty(\phi_X\qty(\frac{\theta}{\sqrt{N}}))^n $$
		We claim $\qty(\phi_X\qty(\frac{\theta}{\sqrt{N}}))^n \to e^{-\frac{\theta^2}{2}}$.
	\end{proof}
\end{theorem}
\end{document}
