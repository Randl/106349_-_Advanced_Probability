\documentclass[]{article}
\usepackage{amsmath}
\usepackage{amsfonts}
\usepackage{amssymb}
\usepackage{hyperref}
\usepackage{gensymb}
\usepackage{graphicx}
\usepackage{svg}
\usepackage{bbding}
\usepackage{mathtools}
\usepackage{centernot} % not parallel, etc.
\usepackage{lmodern}
\usepackage{morewrites}
\usepackage{xcolor,sectsty} % colorful sections
\usepackage[left=10mm, top=10mm, right=10mm, bottom=20mm, nohead]{geometry}
%\usepackage{bigints}
\usepackage{dsfont} %mathbb 1
\usepackage{esint} % beatiful integrals
\usepackage[arrowdel]{physics}
\usepackage{amsthm} % theorems

\usepackage[T1]{fontenc}
% Nicer default font (+ math font) than Computer Modern for most use cases
% \usepackage{mathpazo} % problems with greek vectors
\usepackage[utf8x]{inputenc} % Allow utf-8 characters in the tex document
% Prevent overflowing lines due to hard-to-break entities
\sloppy 
% Colors for the hyperref package
\definecolor{urlcolor}{rgb}{0,.145,.698}
\definecolor{linkcolor}{rgb}{.71,0.21,0.01}
\definecolor{citecolor}{rgb}{.12,.54,.11}
% Setup hyperref package
\hypersetup{
	breaklinks=true,  % so long urls are correctly broken across lines
	colorlinks=true,
	urlcolor=urlcolor,
	linkcolor=linkcolor,
	citecolor=citecolor,
}


\DeclareFontFamily{OMX}{lmex}{}
\DeclareFontShape{OMX}{lmex}{m}{n}{<-> lmex10}{}


%colors of sections
\definecolor{secfont}{RGB}{46,116,181}
\definecolor{subfont}{RGB}{146,23,57}
\definecolor{parfont}{RGB}{19,127,43}
\definecolor{subparfont}{RGB}{7,11,100}

\subsectionfont{\color{subfont}}
\sectionfont{\color{secfont}}
\paragraphfont{\color{parfont}}
\subparagraphfont{\color{subparfont}}


% declare a new theorem style
\newtheoremstyle{bluestyle}%
{3pt}% Space above
{3pt}% Space below 
{}% Body font
{}% Indent amount
{\bfseries\color{blue}}% Theorem head font
{.}% Punctuation after theorem head
{.5em}% Space after theorem head
{}% Theorem head spec (can be left empty, meaning ‘normal’)
% declare a new theorem style
\newtheoremstyle{redstyle}{3pt}{3pt}{}{}{\bfseries\color{red}}{.}{.5em}{}
\newtheoremstyle{olivestyle}{3pt}{3pt}{}{}{\bfseries\color{olive}}{.}{.5em}{}
\newtheoremstyle{orangestyle}{3pt}{3pt}{}{}{\bfseries\color{orange}}{.}{.5em}{}
\newtheoremstyle{magentastyle}{3pt}{3pt}{}{}{\bfseries\color{magenta}}{.}{.5em}{}

\theoremstyle{bluestyle}
\newtheorem{theorem}{Theorem}[section]
\theoremstyle{redstyle}
\newtheorem{definition}{Definition}[section]
\theoremstyle{magentastyle}
\newtheorem{coll}{Collary}[section]
\theoremstyle{olivestyle}
\newtheorem{lemma}{Lemma}[section]
\theoremstyle{olivestyle}
\newtheorem{prop}[theorem]{Proposition}

%\usepackage{babel}[english]
%opening
\title{106349 - Advanced probability}
\author{Nick Crawford}
% njc860@gmail.com
% Amado 707
% https://sites.google.com/site/njcrawfordacademic/
% David Williams Probability with Martingales
% Rick Derret Probability Theory and examples


\parindent=0em
\begin{document}


\maketitle

\begin{abstract}

\end{abstract}

%\tableofcontents
\section{Introduction. Summary of course through an example. Branching process}
We have an individual that gives a birth to a random number of offsprings -- random variable $X$.
$X$ define a distribution, i.e., $P: \mathbb{Z}^+ \to [0,1]$, i.e., $P(X=k) \in [0,1]$, and $\sum_{k=0}^\infty P(X=k)  = 1$.

\begin{definition}
	$f_X(\theta) = \sum_{k=0}^\infty \theta^k P(X=k)$ -- moment-generating function.
\end{definition}

 The series is absolutely convergent for $\theta\in [-1,1]$ since $k$ sums to $1$. For $\theta \in (-1,1)$, $f_x$ is analytic, thus we can differentiate it term-by-term:
$$f'_X(\theta) = \sum_{k\geq1} \theta^{k-1} P(X=k)$$
Since, $f_X$ is analytic, knowing it means knowing $P(X=k)$ and vice versa.

Note that $f_X(0) = P(X=0)$ and $f_X(1)=1$. Also
$$f'_X(1) = \sum_{k\geq 0 }^\infty k P(X=k) = \mathbb{E} X = \mu$$
$$\lim_{\theta \to 1} \frac{f_X(1)-f_X(\theta)}{1-\theta}=\lim_{\theta \to 1} \frac{1-f_X(\theta)}{1-\theta}$$

Note also that $f_X$ is convex, since second derivative is positive.
\paragraph{Size of $n^{th}$ generation}
Let $\qty(X_r^{(n)})_{n,r=1^\infty}$, where $n$ is generation and $r$ is offspring number (index) in $n^{th}$ generation.

Assume $X_r^{(n)}$ are i.i.d.  (independent, identically distributed) random variables.

Identically distributed means $$P(X_n^r = k) = P(X=k)$$.

Independence means $$P\qty(\forall i<J \: X_{r_i}^{n_i} =k) = \prod_{i=1}^J P\qty( X_{r_i}^{n_i} =k)$$.

Define $z_1=X_1^1$. $z_2 = \sum_{r=1}^{z_1} X_r^2$ an so on:
$$z_{n+1} = \sum_{r=1}^{z_n} X_{r}^{n}$$

We want to study asymptotics of $z_n$.

Given $U$ and $V$ taking values in $ \mathbb{Z}^+$,
$$\mathbb{E}] \qty[ U|V=k] = \sum_{j=0}^\infty j P(U=j | V=k)$$, where $$P(U=j|V=k) = \frac{P(U=j, V=k)}{P(V=k)}$$

\paragraph{} If $U$, $V$ are independent, $P(U=j|V=k) = P(U=j)$ and thus $\mathbb{E}\qty[ U|V=k] = \mathbb{E} U$.
\begin{definition}
Define random variable $\mathbb{E}\qty[U|V]$ such that
$$\mathbb{E}\qty[U|V] = \mathbb{E}\qty[U|V=k] $$
if $V=k$.
\end{definition}

\begin{definition}[Tower property]
$$\mathbb{E} \qty\big[\mathbb{E}\qty[U|V]] = \mathbb{E}U$$

Define $$f_{n} = \sum_{k=0}^\infty \sum_{k=0}^\infty \theta^{k} P(z_n=k) = \mathbb{E} \theta^{z_n}$$. 

\end{definition}
\begin{theorem}
$$f_{n+1} (\theta) = f_n (f_X(\theta))$$
or
$$f_{n} (\theta) = \underbrace{f\circ f\circ \dots \circ f}_{n \text{ times}}(\theta))$$
\begin{proof}
	
	Use tower property with $U^{z_{n+1}} $ and $V=\theta^{z_n}$.
	By tower property
	$$\mathbb{E} \qty[\theta^{z_{n+1}}] = \mathbb{E} \qty\big[\mathbb{E} \qty[\theta^{z_{n+1}}| \theta^{z_{n}}]]$$
	$$ \mathbb{E} \qty\big[\mathbb{E} \qty[\theta^{z_{n+1}}| \theta^{z_{n}}]] = \sum_{k=0}^\infty P(z_n=k) \mathbb{E} \qty[\theta^{z_{n+1}}| \theta^{z_{n}}=k]$$
What is $\mathbb{E} \qty[\theta^{z_{n+1}}| \theta^{z_{n}}=k]$?
$$\mathbb{E} \qty[\theta^{z_{n+1}}| \theta^{z_{n}}=k] =\mathbb{E} \qty[\theta^{\sum_{j=1}^{k} X_j^{n+1}}| \theta^{z_{n}}=k] \stackrel{\text{independence}}{=}\mathbb{E} \qty[\theta^{\sum_{j=1}^{z_n} X_j^{n+1}}]\stackrel{\text{independence}}{=} \prod_{j=1}^k \mathbb{E} \qty[\theta^{ X_j^{n+1}}] \stackrel{\text{i.d.}}{=} (f_X(\theta))^k $$
Thus
$$ \mathbb{E} \qty\big[\mathbb{E} \qty[\theta^{z_{n+1}}| \theta^{z_{n}}]] = \sum_{k=0}^\infty P(z_n=k) (f_X(\theta))^k = f_n(f(\theta))$$

Also we can say
$$\mathbb{E} \qty[\theta^{z_{n+1}}|z_n] = \qty(f_X(\theta))^{z_n}$$
\end{proof}
\end{theorem}

\paragraph{Study of $z_n$}
What is $\pi_n= P(z_n=0) = f_n(0) = f(\pi_{n-1})$, probability that population is extinguished. Since $z_{n-1} =0 \Rightarrow z_n=0$, i.e. $\pi_n$ is non-decreasing.

Let $P\qty(z_n=0 \text{ for some n}) = \pi$.

We hope that $\left\{ z_n=0 \right\}$ such that
$$\bigcup_n \left\{ z_n=0 \right\} = \left\{ z_n=0 \text{ for some n}\right\}$$
i.e., $\pi = \lim_{n\to \infty} \pi_n$. We call $\pi$ the extinction probability.

\begin{theorem}
	If $\mu=\mathbb{E} > 1$ then $\pi$ is a unique root of $\pi=f(\pi)$ and $\pi \in [0,1)$. If $\mu\leq 1$, $\pi=1$.
	
	If we look at $f(\pi)$ and $\pi$, they intersect in $1$, and they can intersect in two points since $f(x)$ is convex. There is second intersection iff $f'(1) = \mu > 1$.
\end{theorem}
\paragraph{Construction of $X_n^r$}
Construct set $\Omega$, $f_{n,r}: \Omega \to \mathbb{Z}^+$ and $\mathcal{F}$ a collection of subsets of $\Omega$ with $P: \mathcal{F} \to [0,1]$.

Let $\Omega = \mathbb{Z}^+ \times \mathbb{Z}^+$, $\mathcal{F} = \left\{ 0,1 \right\}^\Omega$. 

The problem is when we have infinitely number of variables.



\paragraph{Example}
Example of not well-behaved triple $(\Omega, \mathcal{F}, P)$. $\Omega =\mathbb{N}$. Now $\mathcal{F} = \left\{  C\subset \mathbb{N} : C \text{ has density} \right\}$.

$C$ has density means 
$$\frac{\abs{C \cap \mathbb{N}}}{n} \stackrel{n\to \infty}{\to} \rho(C)$$

However, for $C(m) = \left\{  1,2,\dots, m \right\}$, $\forall m \quad \rho(c_m)$, and 
$$\rho\qty(\bigcup C_m ) = 1$$

Thus $(\mathbb{N}, \mathcal{F}, \rho)$ is not a good probability space, since it doesn't fulfills this
$\pi_n\to \pi$ property. Note we can define other probabilities on naturals, for example
$$P\qty(\left\{ i \right\}) = 2^{-i}$$


\paragraph{Asymptotics of $z_i$ }
Assuming $\pi \in (0,1)$, what is behavior of $z_n$?
\begin{definition}
	$z_n$ is a Markov chain if 
	$$P(z_{n+1}=j|z_i=k_i \quad \forall i\leq n) = P(z_{n+1} = j | z_n=k_n)$$
	We can use to compute expectation:
	$$\mathbb{E}[z_{n+1}|z_i=k_i \quad \forall i<n] = E\qty[z_{n+1}|z_n=k_n]$$
\end{definition}

Then, since $E\qty[\sum_{i=1}^J X_i^n] = J\mu$
$$E\qty[z_{n+1}|z_n] = \mu z_n$$

Let $M_n = \frac{z_n}{\mu^n}$ then $\mathbb{E}[M_n] = 1$. Also
$$\mathbb{E}[M_{n+1}|z_0,\dots, z_n] = M_n$$
This is a definition of martingale with respect to $z_0, \dots, z_n$.

Let $\qty(\Omega, \mathcal{F}, P)$ we say $S$ happens almost surely (a.s.) if $$P\qty(\left\{ w\in \Omega : S \text{ is true for w} \right\}) = 1$$
\begin{theorem}[Martingale convergence theorem]
	If $M_n$ is a positive martingale then $\lim_{n\to \infty} M_N=M_\infty$ exists a.s. and 
	\begin{itemize}
		\item $\mu \leq 1$. $M_\infty =0 $ a.s. That means $\mathbb{E}M_\infty = 0$ but $\mathbb{E} M_ = 1$, i.e., 
		$$\mathbb{E}\qty[\liminf_{n\to \infty} M_n] < [\liminf_{n\to \infty} \mathbb{E}\qty[M_n]$$ 
		\item $\mu>1$. If $M_\infty > 0$ with positive probability then $z_n \sim \mu^n M_\infty$.  
	\end{itemize}
\end{theorem}


\begin{lemma}[Fatou's lemma] 
	$$\mathbb{E}\qty[\liminf_{n\to \infty} M_n] \leq \liminf_{n\to \infty} \mathbb{E}\qty[M_n]$$ 
\end{lemma}

\begin{theorem}
	$$\mathbb{E} \qty[M_\infty] = 1 \iff \mu>1 \quad \text{ and } \mathbb{E}\qty[X \log(X)] < \infty$$
\end{theorem}

\section{Overview of measure theory}
\paragraph{Notation}
\begin{itemize}
	\item $S$ is a set.
	\item $\mathcal{A}$ is algebra of subsets of $S$
	\begin{enumerate}
		\item $S\in \mathcal{A}$
		\item $$E\in \mathcal{A} \Rightarrow E^C\in \mathcal{A}$$, where $E^C = S \setminus E$
		\item $$E_1, E_2 \in \mathcal{A} \Rightarrow E_1 \cup E_2 \in \mathcal{A}$$
		meaning
		$$E_1, E_2 \in \mathcal{A} \Rightarrow E_1 \cap E_2 \in \mathcal{A}$$
	\end{enumerate}
\item $\mathcal{F}$ is a $\sigma$-algebra if lest item works for countable union.
\item $E\Delta F = E\setminus F \cup F \setminus E$ 
\end{itemize}

\begin{definition}
	A measurable space is a pair $\left\{S,\mathcal{F}\right\}$.
\end{definition}

\begin{prop}
	If we have $\qty(\mathcal{F}_i)_{i\in I}$, then $\bigcap_{i\in I} \mathcal{F} $ is also a $\sigma$-algebra.
\end{prop}

\begin{definition}
	Let $C$ be a collection of subsets of $S$. $\sigma(C)$ is a smallest $\sigma$-algebra containing $C$ ($\sigma$-algebra generated by $C$).
	
	It is easy to construct one
	$$I = \left\{ \mathcal{F} :\mathcal{F} \supset C \right\}$$
	and then
	$$\sigma(C) = \bigcap_{\mathcal{F}\in I} \mathcal{F}$$
\end{definition}

\begin{definition}
	Let $\left\{S,\mathcal{F}\right\}$ be a topological space. $\mathcal{B}(X)$ (Borel $\sigma$-algebra) is defined as $\sigma$-algebra generated by open sets. We denote
	$\mathcal{B} = \mathcal{B}(\mathbb{R})$.
\end{definition}
\paragraph{Exercise}
$$\pi(\mathbb{R}) = \left\{ (-\infty, x], \: x\in \mathbb{R} \right\}$$
Show that $\sigma(\pi(\mathbb{R})) = B$

\begin{definition}
	Additive set function on a collection of sets $\mathcal{F}$ is
	$$\mu: \mathcal{F} \to [0, \infty)$$
	$$\forall E,F \in \mathcal{F} \: E\cap F = \emptyset \quad \mu(E\cup F) = \mu(E) +\mu(F) $$
	
	We say $\mu$ is $\sigma$-additive if same holds of countable infinite sets
	$$\forall \left\{ E_i \right\}_{i=1}^\infty \: E_i \cap E_j = \emptyset \quad \mu\qty(E\cup F) = \sum_{i=1}^\infty \mu(E_i)  $$
	
\end{definition}


\begin{definition}
	A triple $\qty(S, \mathcal{F}, \mu)$ is a measure space if $\mathcal{F}$ is a $\sigma$-algebra on $S$ and $\mu$ is $\sigma$-additive on $\mathcal{F}$.
	
\end{definition}


\begin{definition}
 $\qty(S, \mathcal{F}, \mu)$ is finite if $\mu(S) < \infty$
 
  $\qty(S, \mathcal{F}, \mu)$ is $\sigma$-finite if
	$$\exists \left\{ E_i, \: \mu(E_i)<\infty \right\}_{i=1}^\infty \quad S=\bigcup_{i=1}^\infty E_i$$
\end{definition}
\begin{definition}
If $\mu(S) =1$, $\qty(S, \mathcal{F}, \mu)$ is probability space.
\end{definition}

\begin{definition}
	$E$ is null if $\mu(E)=0$.
\end{definition}
\begin{definition}
	$\phi$ is said to be true almost everywhere with respect of $\mu$ if 
	$$\mu(\left\{ X : \phi(X) = \text{False} \right\}) = 0$$
\end{definition}

\subsection{Results from measure theory}
\begin{definition}
	A collection of sets $\mathcal{D}$ is called a $\pi$-system if $E,F \in D \: \Rightarrow \: E\cap F \in \mathcal{D}$
\end{definition}
\begin{theorem}[Uniqness]
	Let $\mathcal{D}$ be a $\pi$-system generating a $\sigma$-algebra $\mathcal{F}$. Let $\mu_1$ and $\mu_2$ be two finite measures on $\mathcal{F}$ which agree on $\mathcal{D}$. Then $\mu_1=\mu_2$.
	
	\begin{coll}
		$(S,\mathcal{F}, P_1)$, $(S,\mathcal{F}, P_2)$ probability spaces, $P1=P2$ on $\pi$-system $\mathcal{D}$, then $P_1=P_2$.
	\end{coll}
\end{theorem}

\begin{theorem}[Carath\'{e}odory's extension theorem]
	Let $\mathcal{A}$ be an algebra of sets. $\mu_0: \mathcal{A} \to \mathbb{R}^+$ $\sigma$-additive set function on $\mathcal{A}$. Then exists unique extension $\bar{\mu} : \sigma(\mathcal{A}) \to \mathbb{R}^+ $ such that $\bar{mu} = \mu_0$.
\end{theorem}

\paragraph{Homework}
Lebesgue on $\mathbb{R}$.  $\mathcal{A} = \left\{ \text{open set} \right\}$. If we have
$$O = \bigcup_{i=1}^\infty (a_i, b_i)$$
then
$$\mu_0(O) = \sum_{i=1}^\infty b_i-a_i$$

Check that $\mu_0$ is well defined and $\sigma$-additive.

\begin{lemma}
	
	$(S,\mathcal{F}, \mu)$ measure space. $A,B \in \mathcal{F}$, then
	$$\mu(A\cup B) \leq \mu(A) + \mu(B) $$
	$$\mu\qty(\bigcup_{i=1}^\infty F_i) \leq \sum_{i=1}^\infty \mu(F_i) $$
	
	If $\mu(S) < \infty$
	$$\mu(A\cup B) = \mu(A) + \mu(B) - \mu(A\cap B)$$
	
	From that we get inclusion-exclusion:
	$$\mu\qty(\bigcup_{i=1}^n A_i) = \sum_{i=1}^n \mu(A)_i - \sum_{i\neq j}\mu(A_i\cap A_j) + \dots + (-1)^{n-1} \mu\qty(\bigcap_{i=1}^n A_i)$$
\end{lemma}
\paragraph{Exercise}
Proof the lemma

\begin{lemma}
	If $F_n \subseteq F_{n+1}$ then
	$$\mu\qty(\bigcup_{i=1}^\infty F_i) = \lim_{n\to \infty} \mu(F_n)$$
	
	
	If $\mu(S) < \infty$ and $F_n \supseteq F_{n+1}$ then
	$$\mu\qty(\bigcap_{i=1}^\infty F_i) = \lim_{n\to \infty} \mu(F_n)$$
\end{lemma}
\begin{proof}
	Assume $\mu(S) < \infty$.  Define 
	$F_\infty = \bigcup_{i=1}^\infty F_i$.
	Let $G_n = F_{n}\setminus F_{n+1}$. Then
	$$F_\infty = \bigcup_{i=1}^\infty G_i$$
	Meaning 
	$$\mu(F_\infty) =\sum_{i=1}^\infty G_i$$
	$$\mu(F_n) =\sum_{k=1}^n G_k$$
	
	Since measure is finite, the tail of series tends to $0$, thus 
	$$\mu(F_\infty) - \mu(F_n) = \sum_{k=n}^\infty G_k \to 0$$
	
	Then we can take complements and get the second statement.
\end{proof}

\paragraph{Exercise}
Proof unconditionally
\section{Recasting measure theory as probability}
\begin{definition}
	A probability space is a $(\Omega, \mathcal{F}, P)$ is a measure space such that $P(\Omega) = 1$.
	
	We call $\omega \in \Omega$ an outcome. $E\in \mathcal{F}$ is an event. $P(E)$ is probability of the event.
\end{definition}

\paragraph{Example}
Tossing finite or infinite sequence of coins.


\subparagraph{Tossing 4 coins}
$$\Omega = \left\{ HHHH, HHHT, HHTH, \dots, TTTH, TTTT \right\}$$
$$\mathcal{F} = 2^\Omega$$
$$P(\omega \in \Omega) = \frac{1}{\abs{\Omega}}$$
	
\subparagraph{Tossing infinite number coins}
$$\Omega = \left\{ 0,1 \right\}^{\mathbb{N}}$$


$\Omega$ has a natural topology which is called a product topology. It is coarsest topology such that $\pi_i: \Omega \to \left\{ 0,1 \right\}$ $\pi_i(\omega) =\omega_i$ is continuous. 

Let $\mathcal{F} = \mathcal{B}(\Omega)$.

Smallest $\sigma$-algebra such that
$$\pi_i^{-1}(0) \subset \Omega \in \mathcal{F} $$
$$\pi_i^{-1}(1) \subset \Omega \in \mathcal{F} $$

$$\pi_i (\Omega, \mathcal{F} ) \to \qty(\left\{ 0,1 \right\}, \left\{ 0,1 \right\}^{\left\{ 0,1 \right\}})$$

Natural $\pi$-system $\mathcal{F}_n$ smallest $\sigma$-algebra making $\pi_1, \dots, \pi_n$ measurable.

Note that
\begin{prop}
	
	$$\bigcup_n \mathcal{F}_n \neq \mathcal{F} $$
	\begin{proof}
		
		Define $S_n(\omega) = \sum_{i=1}^n \omega_n$. 
		$$X_n = \frac{S_n(\omega)}{n}$$
		Define
		$$Y(\omega) = \limsup X_n(\omega)$$
		$$E = \left\{ \omega : Y(\omega) \geq \frac{1}{3} \right\}$$
		$$E \in \mathcal{F} \setminus \bigcup_n \mathcal{F}_n$$
	\end{proof}
\end{prop}


\paragraph{What $\mathcal{F}_n$ looks like?}
For example, $\mathcal{F}_2$ has 4 outcomes, deciding only first two tosses.
\paragraph{Note} If we take $(\Omega_4, \mathcal{F}^{(4)}, P_4)$, restricting to $(\Omega_3, \mathcal{F}^{(3)}, P_3)$
$$P_4\qty(\left\{ (0,0,0,\omega_4)\right\}) = P_3\qty(\left\{ (0,0,0)\right\})  $$


Thus we want $P_{fair}$ defined on $\Omega$ to  fulfill same property:
$$P_{fair}(E) = P_n(\tilde{E})$$
where $E \in \mathcal{F}_n$ and $\tilde{E} \in F^{(n)}$.

\begin{definition}
	$E\subset \mathcal{F}$ occures almost surely (a.s.) if $P(E)=1$.
\end{definition}


\begin{definition}[$\limsup$ and $\liminf$]
 Let	$\left\{ E_n \right\}$ be a sequence of events.
 $$\limsup E_n = \bigcap_{m}\bigcup_{n\geq m} E_n = \left\{ E_n \text{ occurs infenetely often (i.o.)} \right\} = \left\{ \omega \in \Omega  : \: \forall m \: \exists n(\omega)>m \quad \omega \in E_n(\omega) \right\}$$
 
 Alternatively, $(\Omega, \mathcal{F})$ and $\left\{ E_n \right\}$  there is a natural map 
 $$I: \Omega \to \left\{ 0,1\right\}^N$$
 $$\omega \mapsto \left\{ 1_{E_n} (\omega) \right\}$$
 
 where $$1_{E}(\omega) = \begin{cases}
 0 & \omega \notin E\\
 1 & \omega \in E\\
 \end{cases}$$
 
 Now
 
 $$\liminf E_n = \bigcup_{m}\bigcap_{n\geq m} E_n = \left\{ E_n \text{ occurs eventually} \right\} = \left\{ \omega \in \Omega  : \: \exists m(\omega) \: \forall n\geq m(\omega) \quad \omega \in E_n(\omega) \right\}$$
\end{definition}

\paragraph{Remark}
Since everything is countable, if $E_n \in \mathcal{F}$, then $\limsup E_n,\liminf E_n\in \mathcal{F}$ 

We can write
$$\left\{ \frac{S_n}{n} \to \frac{1}{2} \right\} = \left\{ \limsup \frac{S_n}{n} \leq \frac{1}{2} \right\} \cap \left\{ \liminf \frac{S_n}{n} \geq \frac{1}{2} \right\}$$.

Choose $q\in \mathbb{Q}^+$ and take a look at $$\left\{ \liminf \frac{S_n}{n} > q \right\} = \liminf E_n(q)$$
where $E_n = \left\{ \omega  : \frac{S_n}{n} > q \right\}$.

In addition
$$\left\{ \limsup \frac{S_n}{n} < q \right\} = \liminf F_n(q)$$
where $F_n = \left\{ \omega  : \frac{S_n}{n} < q \right\}$.

Therefore
$\left\{ \liminf \frac{S_n}{n} > q \right\} \in \mathcal{F}$.


Finally,
$$\left\{ \liminf \frac{S_n}{n} \geq \alpha \right\} = \bigcap_{q<\alpha} \left\{ \liminf \frac{S_n}{n} > q \right\} $$



\begin{lemma}[Fatou's lemma]
	$$P\qty[\liminf_{n\to \infty} E_n] \leq \liminf_{n\to \infty} p\qty(E_n)$$
	
	\begin{proof}
		$$\liminf_{n\to \infty} E_n = \bigcup_m \bigcap_{n\geq m} E_n $$
		
		Sets $F_m= \bigcap_{n\geq m} E_n $ are increasing and $F_n \subseteq E_n$, thus
		$$P\qty[\liminf_{n\to \infty} E_n] = \lim_{n\to \infty} P(F_n) \leq \liminf_{n\to \infty} P(E_n)  $$
	\end{proof} 
\end{lemma}


\begin{lemma}[Fatou's lemma] \label{fatou}
	$$P\qty[\limsup_{n\to \infty} E_n] \geq \limsup_{n\to \infty} p\qty(E_n)$$
	
	\begin{proof}
		Note that $\qty(\limsup E_n)^C = \liminf E_n^C$, thus this is straightforward form previous lemma.
	\end{proof} 
\end{lemma}

\begin{lemma}[First Borel-Cantelli lemma] \label{bc1}
	Let $\left\{ E_n\right\} \subseteq \mathcal{F}$ be a sequence of events s.t. $\sum_n P(E_n) < \infty$, then
	$$P(E_n \text{ happens i.o.}) =0$$
	
	\begin{proof}
		$$P(E_n \text{ i.o.}) = P\qty(\bigcap_m \bigcup_{n\geq m} E_n) \leq  P\qty( \bigcup_{n\geq m} E_n) \leq \sum_{n=m}^\infty P(E_n) \stackrel{m\to \infty}{\to} 0$$
		Since $P(E_n \text{ i.o.})$ is independent on $m$, it got to be $0$.
	\end{proof}
\end{lemma}

\paragraph{Example}
Fix $\epsilon>0$. Look at $P\qty(\abs{\frac{S_n(\omega)}{n} -\frac{1}{2}}>\epsilon)$.

\paragraph{Claim} $$P\qty(\abs{\frac{S_n(\omega)}{n} -\frac{1}{2}}>\epsilon) \leq \frac{12}{\epsilon^4} \frac{1}{n^2}$$

By \ref{bc1} $P\qty(\abs{\frac{S_n(\omega)}{n} -\frac{1}{2}}>\epsilon \text{ i.o.}) = 0$ thus
$$\left\{ \frac{S_n}{n} \to \frac{1}{2} \right\} = \bigcap_{\epsilon>0} \left\{ \abs{\frac{S_n(\omega)}{n} -\frac{1}{2}}<\epsilon \text{ eventually}  \right\} = 1$$
		
		



\begin{definition}
	Let $\qty(S,\mathcal{F})$, $\qty(\Omega, \mathcal{B})$ be measurable spaces.
	$$\phi: S \to \Omega$$
	
	$\phi$ is $\qty\big((\mathcal{F}, \mathcal{B}))$-measurable if 
	$\forall B\in \mathcal{B} \quad \phi^{-1}(B) \in \mathcal{F} $.
\end{definition}

\paragraph{Remark}
$\mathcal{C}$ is collection of sets in $\Omega$. $\phi^{-1}(\mathcal{C}) = \left\{ \phi^{-1}(C) : C \in \mathcal{C} \right\}$.
\begin{itemize}
	\item $$\phi^{-1}\qty(\bigcap_{i\in I} B_i) =\bigcap \phi^{-1} (B_i)$$
	\item $$\phi^{-1}\qty(\bigcup_{i\in I} B_i) =\bigcup \phi^{-1} (B_i)$$
	\item $$\phi^{-1}\qty(B^C) =\qty[\phi^{-1} (B)]^c$$
\end{itemize}

\begin{lemma}
	Let $\sigma(\mathcal{C}) = \mathcal{B}$. $\phi$ is measurable iff $\phi^{-1}(\mathcal{C}) \subseteq \mathcal{F}$.
	
	\begin{coll}
		$\Omega = \mathbb{R}$, $\mathcal{B}(\mathbb{R})$ then $\phi$ is measurable iff
		$$\forall x \: \phi^{-1}((-\infty, x]) \subseteq \mathcal{F}$$
	\end{coll}
\end{lemma}
\begin{lemma}
	Let $\qty(S,\mathcal{F})$, $\qty(T, \mathcal{T})$, $\qty(\Omega, \mathcal{B})$ be measurable spaces. Let $\phi_1: S \to T$ and $\phi_2 T \to \Omega$ measurable. Then $\phi_2 \circ \phi_1$ is measurable.
	\begin{proof}
		Let $B\in \mathcal{B}$. Then $\phi_2^{-1}(B) \in \mathcal{T}$, and thus $\phi_1^{-1}\qty(\phi_2^{-1}(B)) \in \mathcal{F}$, meaning $\qty(\phi_2 \circ \phi_1)^{-1}\qty(B) \in \mathcal{F}$.
	\end{proof}
\end{lemma}
\begin{lemma}
	$\Omega = \mathbb{R}$. Then $\left\{ \phi| \phi \text{ is } \mathcal{F}, \mathcal{B} \text{-measurable} \right\}$ is an algebra over $\mathbb{R}$.
	\begin{proof}
		Using previous lemma and the fact $+$ is continuous, and thus measurable, we define $\Psi(s) = \qty(\phi_1(s), \phi_2(s))$.
		
		$\Psi$ is measurable. Take a look at
		$$\Psi^{-1}\big((-\infty, x_1] \cross (-\infty, x_2] \big) = \left\{  s: \phi_1(s) \in (-\infty, x_1] ,\: \phi_2(s) \in (-\infty, x_2]  \right\}$$
		
	\end{proof}
\end{lemma}

\paragraph{Notation} 
$$\phi: (S,\mathcal{F}) \to (\Omega, \mathcal{B})$$
We write $\phi\in \mathcal{F}$ for $\phi$ is $\mathcal{F},\mathcal{B}$ measurable.
\paragraph{Constructions preserved by measurability}
\begin{prop}
	If $\left\{ \phi_n \right\}_{n=1}^\infty$ measurable maps $(S,\mathcal{F}) \to (\Omega, \mathcal{B})$, then $\liminf \phi_n$, $\limsup \phi_n$, $\inf \phi_n$, $\sup_n$ are also measurable.
	\begin{proof}
		For example, fpr infimum, we need to show that
		$$\left\{s|\: \inf\limits_n \phi_n(s) \leq c  \right\} \in \mathcal{F}$$
		or alternatively,
		$$\left\{s|\: \inf\limits_n \phi_n(s) > c  \right\} \in \mathcal{F}$$
		which is just countable intersection:
		$$\bigcap_n \left\{ s: \phi_n(s) > c \right\}$$
		
		Same for $\limsup$, which is just infimum of supremum:
		$$\limsup \phi_n = \inf\limits_m \qty(\sup\limits_{n\geq m} \phi_n)$$
	\end{proof}
\end{prop}

\paragraph{Recall}
$$S_n = \text{number of 1's until n}$$
We can view $s_n$ as a composition of projection and sum:
$$\omega \mapsto \qty\big(\pi_1(\omega),\dots \pi_n(\omega)) \mapsto \sum_{i=1}^n \pi_i(\omega)$$
Both are continuous (projection from the definition of product topology) and thus measurable, and so is $\frac{S_n}{n}$.
\section{Random variables}
\begin{definition}
	Let $\qty(\Omega, \mathcal{F}, P)$ be a probability space. 
	$X: \Omega \to (S, \mathcal{S}) $ measurable is called a random variable. 
\end{definition}

\paragraph{Basic constructions with random variables}
\begin{definition}
	Given a probability space $\qty(\Omega, \mathcal{F}, P)$  and measurable $ (S, \mathcal{S})$, $X$ induces measure $\mathcal{L}_X$ on $ (S, \mathcal{S})$ via
	$$\mathcal{L}_X(E) = P(X \in E)$$

$\mathcal{L}_X$ is called marginal distribution of $X$ or law of $X$.
\end{definition}
\begin{prop}
	$\mathcal{L}_X$ is countably additive set function.
\end{prop}

If $ (S, \mathcal{S})$ is $\mathbb{R}, \mathcal{B}$. By uniqueness theorem, $\mathcal{L}_X$ if defined by
$$F_X(x) = \mathcal{L}_X\big((-\infty, x]\big) = P\big(X\in (-\infty, x]\big)$$ 

$\mathcal{L}_X \mapsto F_X$ is 1-1 (onto). $F_X$ is cumulative distribution function (CDF) or distribution function of $X$.

We ask the question: what are set properties distinguish $F_X$? 

\begin{prop}[Properties of CDF]
\begin{enumerate}
	\item $F_X$ is non-decreasing
	\item $F_X$ is right continuous
	\item $F_X(-\infty) = 0$ 
\end{enumerate}
\begin{proof}
	
	\begin{enumerate}
		\item If $x<y$, $(-\infty, x] \subseteq (-\infty, y]$, thus, from monotonicity of measure $F_X(x)\leq F_X(y)$  
		\item we want to show
		$$\lim_{x\downarrow x_0} F(x) = F(x_0)$$
		
		Since if $E_n \downarrow E$, then $\mu(E_n) \to \mu(E)$
		
		We can look on sequence $\left\{ x_n\right\}$:
		$$\omega \in \bigcap_n \left\{ X \in (-\infty, x_n) \right\} \Rightarrow \forall n \quad X(\omega) \leq x_n \Rightarrow X(\omega) \leq x \Rightarrow \bigcap_n \left\{ X \in (-\infty, x_n) \right\} \subseteq \left\{ X \in (-\infty, x) \right\}$$
		The other direction is obvious.
		\item Since 
		$$\bigcap_x X^{-1}\big((-\infty,x]\big) = \emptyset$$
	\end{enumerate}
\end{proof}
\end{prop}

\subsection{Independence}
$(\Omega, \mathcal{F}, P)$ probability space.
Let $\left\{ \mathcal{J}_i \right\}_{i\in I}$ be a collection of sub-$\sigma$-algebras of $\mathcal{F}$.
\begin{definition}[Independence of $\sigma$-algebras]
	Say $\left\{ \mathcal{J}_i \right\}_{i\in I}$ are independent if
	$$\forall i_1, \dots i_k \: \forall j \quad G_{ij} \in  \mathcal{J}_i \quad P\qty(\bigcap_{j=1}^k G_{ij}) = \prod_{j=1}^k P(G_{ij})$$
\end{definition}
\begin{definition}[Independence of random variables]
	Say $\left\{ X_i \right\}_{i\in I}$ are independent if $\sigma(X_i)$ are independent.
\end{definition}
\begin{definition}[Independence of sets]
Say $\left\{ E_i \right\}_{i\in I}$ are independent if random variables $\mathds{1}_{E_i}$ are independent.
\end{definition}
\begin{lemma}[Checking independence]
	Let $\mathcal{F}_1$, $\mathcal{F}_2$ be $\sigma$-algebras, $\mathcal{A}_1$, $\mathcal{A}_2$ $\pi$-systems such that $\sigma(\mathcal{A_1})=\mathcal{F}_i$.
	
	Then $\mathcal{F}_1$, $\mathcal{F}_2$ are independent iff 
	$$\forall A_1\in \mathcal{A}_1, \: A_2\in \mathcal{A}_2 \quad P(A_1\cap A_2) P(A_1)P(A_2)$$ 
	\begin{proof}
		Given $E \in \mathcal{F}$ let $P_E(A) = P(A\cap E)$, a measure on $(\Omega, \mathcal{F})$.
		
		Given $A_1\in \mathcal{A}_1$ consider $\eval{P_{A_1}}_{\mathcal{F}_2}$.
			
		$$\forall \: A_2\in \mathcal{A}_2 \: P_{A_1}(A_2) = P(A_1)P(A_2)$$
		
		Thus $P_{A_1}$ and $P(A_2)\times P$ are measures on $\mathcal{F}_2$ agreeing on $\mathcal{A}_2$.
		
		$$\forall A_1 \in \mathcal{A}_1 , E_2\in \mathcal{F}_2  \quad P(A_1\cap E_2) = P(A_1)P(E_2)$$
		
		Next iterate argument argument for  $\mathcal{F}_1 $
		$$\forall \: E_2\in \mathcal{F}_2 \: P_{E_2}= P(E_2)P$$
		
		By uniqueness 
		$$\forall E_i \in \mathcal{F}_i \quad P(E_1\cap E_2) = P(E_1)P(E_2)$$  
	\end{proof}

\begin{coll}
	To check $X_1,\dots, X_k$ are independent random variables it suffices 
	$$\forall x\in \mathbb{R}^k  \quad P(X_i\leq x_i) = \prod_{i=1}^k P(X_1\leq x_i)$$ 
\end{coll}
\end{lemma}

\begin{lemma}[Second Borel-Cantelli lemma]
	 \label{bc2}
	 If $\sum_i P(E_i) =\infty$ and $\left\{ E_i \right\}_{i=1}^\infty $ are independent, then 
	 
	 $$P(E_n \text{ i.o.}) =0$$
	 \begin{proof}
	 	$$\left\{ E_i  \text{ i.o.} \right\}^C =\left\{ E_i^C  \text{ eventually} \right\} $$
	 	It's enough to show
	 	$$P\qty(\bigcap_{i\geq n} E_i^C) = 0 $$
	 	or, by truncating
	 	$$P\qty(\bigcap_{i\geq n}^k E_i^C) = 0 $$
	 	
	 	By independence
	 	$$P\qty(\bigcap_{i\geq n} E_i^C) \prod_{i=n}^k P(E_i^C) = \prod_{i=n}^k \qty[1-P(E_i)] \leq e^{-\sum_{i=n}^k P(E_i)}$$
	 	
	 	Since the sum tends to infinity, the exponent tends to $0$.
	 \end{proof}
\end{lemma}

\paragraph{Example}
Let $\left\{ X_i \right\}_{i=1}^\infty$ be i.i.d. $Exp(1)$ random variables, i.e. 
$$P(X_i>x) = e^{-x}$$

We are interested in growth rate of $X_n \leq f(n)$.

If $f(n) = \alpha \log(n)$
$$P(X_n > f(n)) = e^{-f(n)} = n^{-\alpha}$$
Thus, from Lemmas \ref{bc1}, \ref{bc2}
$$P(X_n \geq \alpha \log(n) \text{ i.o.}) = \begin{cases}
0 & \alpha > 1\\
1 & \alpha \leq 1
\end{cases}$$
Define $L = \limsup_{n\to \infty} \frac{X_n}{\log(n)} $, then
$$P(L\geq 1) = P(X_n \geq \log(n) \text{ i.o.}) = 1$$

Finally, if we look at $E = \bigcup_{k=1}^\infty \left\{ L \geq 1+ \frac{1}{k} \right\}$,
$$P(E)  \leq \sum_{k=1}^\infty P\qty( L \geq 1+ \frac{1}{k}) \leq \sum_{k=1}^\infty P\qty( X_n \geq \qty(1+ \frac{1}{2k})\log(n)) = 0 $$
thus $P(L\leq 1)=1$.

\paragraph{Method of generation of i.i.d. uniform $[a,b]$ variables}

We write $\omega = \sum_{i=1}^\infty \frac{\omega_i}{2^i}$
$$\begin{cases}
w^{(1)} = \omega_1\omega_3\omega_6\omega_{10}\omega_{15}\dots\\
w^{(2)} = \omega_2\omega_5\omega_{9}\omega_{14}\dots\\
w^{(3)} = \omega_4\omega_8\omega_{13}\omega_{19}\dots\\
\vdots
\end{cases}$$
\end{document}
