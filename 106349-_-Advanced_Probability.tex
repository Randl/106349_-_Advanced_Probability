\documentclass[]{article}
\usepackage{amsmath}
\usepackage{amsfonts}
\usepackage{amssymb}
\usepackage{hyperref}
\usepackage{gensymb}
\usepackage{graphicx}
\usepackage{svg}
\usepackage{bbding}
\usepackage{mathtools}
\usepackage{centernot} % not parallel, etc.
\usepackage{lmodern}
\usepackage{morewrites}
\usepackage{xcolor,sectsty} % colorful sections
\usepackage[left=10mm, top=10mm, right=10mm, bottom=20mm, nohead]{geometry}
%\usepackage{bigints}
\usepackage{dsfont} %mathbb 1
\usepackage{esint} % beatiful integrals
\usepackage[arrowdel]{physics}
\usepackage{amsthm} % theorems

\usepackage[T1]{fontenc}
% Nicer default font (+ math font) than Computer Modern for most use cases
% \usepackage{mathpazo} % problems with greek vectors
\usepackage[utf8x]{inputenc} % Allow utf-8 characters in the tex document
% Prevent overflowing lines due to hard-to-break entities
\sloppy 
% Colors for the hyperref package
\definecolor{urlcolor}{rgb}{0,.145,.698}
\definecolor{linkcolor}{rgb}{.71,0.21,0.01}
\definecolor{citecolor}{rgb}{.12,.54,.11}
% Setup hyperref package
\hypersetup{
	breaklinks=true,  % so long urls are correctly broken across lines
	colorlinks=true,
	urlcolor=urlcolor,
	linkcolor=linkcolor,
	citecolor=citecolor,
}


\DeclareFontFamily{OMX}{lmex}{}
\DeclareFontShape{OMX}{lmex}{m}{n}{<-> lmex10}{}


%colors of sections
\definecolor{secfont}{RGB}{46,116,181}
\definecolor{subfont}{RGB}{146,23,57}
\definecolor{parfont}{RGB}{19,127,43}
\definecolor{subparfont}{RGB}{7,11,100}

\subsectionfont{\color{subfont}}
\sectionfont{\color{secfont}}
\paragraphfont{\color{parfont}}
\subparagraphfont{\color{subparfont}}


% declare a new theorem style
\newtheoremstyle{bluestyle}%
{3pt}% Space above
{3pt}% Space below 
{}% Body font
{}% Indent amount
{\bfseries\color{blue}}% Theorem head font
{.}% Punctuation after theorem head
{.5em}% Space after theorem head
{}% Theorem head spec (can be left empty, meaning ‘normal’)
% declare a new theorem style
\newtheoremstyle{redstyle}{3pt}{3pt}{}{}{\bfseries\color{red}}{.}{.5em}{}
\newtheoremstyle{olivestyle}{3pt}{3pt}{}{}{\bfseries\color{olive}}{.}{.5em}{}
\newtheoremstyle{orangestyle}{3pt}{3pt}{}{}{\bfseries\color{orange}}{.}{.5em}{}
\newtheoremstyle{magentastyle}{3pt}{3pt}{}{}{\bfseries\color{magenta}}{.}{.5em}{}

\theoremstyle{bluestyle}
\newtheorem{theorem}{Theorem}[section]
\theoremstyle{redstyle}
\newtheorem{definition}{Definition}[section]
\theoremstyle{magentastyle}
\newtheorem{coll}{Collary}[section]
\theoremstyle{olivestyle}
\newtheorem{lemma}{Lemma}[section]
\theoremstyle{olivestyle}
\newtheorem{prop}[theorem]{Proposition}

%\usepackage{babel}[english]
%opening
\title{106349 - Advanced probability}
\author{Nick Crawford}
% njc860@gmail.com
% Amado 707
% https://sites.google.com/site/njcrawfordacademic/
% David Williams Probability with Martingales
% Rick Derret Probability Theory and examples


\parindent=0em
\begin{document}


\maketitle

\begin{abstract}

\end{abstract}

%\tableofcontents
\section{Introduction. Summary of course through an example. Branching process}
We have an individual that gives a birth to a random number of offsprings -- random variable $X$.
$X$ define a distribution, i.e., $P: \mathbb{Z}^+ \to [0,1]$, i.e., $P(X=k) \in [0,1]$, and $\sum_{k=0}^\infty P(X=k)  = 1$.

\begin{definition}
	$f_X(\theta) = \sum_{k=0}^\infty \theta^k P(X=k)$ -- moment-generating function.
\end{definition}

 The series is absolutely convergent for $\theta\in [-1,1]$ since $k$ sums to $1$. For $\theta \in (-1,1)$, $f_x$ is analytic, thus we can differentiate it term-by-term:
$$f'_X(\theta) = \sum_{k\geq1} \theta^{k-1} P(X=k)$$
Since, $f_X$ is analytic, knowing it means knowing $P(X=k)$ and vice versa.

Note that $f_X(0) = P(X=0)$ and $f_X(1)=1$. Also
$$f'_X(1) = \sum_{k\geq 0 }^\infty k P(X=k) = \mathbb{E} X = \mu$$
$$\lim_{\theta \to 1} \frac{f_X(1)-f_X(\theta)}{1-\theta}=\lim_{\theta \to 1} \frac{1-f_X(\theta)}{1-\theta}$$

Note also that $f_X$ is convex, since second derivative is positive.
\paragraph{Size of $n^{th}$ generation}
Let $\qty(X_r^{(n)})_{n,r=1^\infty}$, where $n$ is generation and $r$ is offspring number (index) in $n^{th}$ generation.

Assume $X_r^{(n)}$ are i.i.d.  (independent, identically distributed) random variables.

Identically distributed means $$P(X_n^r = k) = P(X=k)$$.

Independence means $$P\qty(\forall i<J \: X_{r_i}^{n_i} =k) = \prod_{i=1}^J P\qty( X_{r_i}^{n_i} =k)$$.

Define $z_1=X_1^1$. $z_2 = \sum_{r=1}^{z_1} X_r^2$ an so on:
$$z_{n+1} = \sum_{r=1}^{z_n} X_{r}^{n}$$

We want to study asymptotics of $z_n$.

Given $U$ and $V$ taking values in $ \mathbb{Z}^+$,
$$\mathbb{E}] \qty[ U|V=k] = \sum_{j=0}^\infty j P(U=j | V=k)$$, where $$P(U=j|V=k) = \frac{P(U=j, V=k)}{P(V=k)}$$

\paragraph{} If $U$, $V$ are independent, $P(U=j|V=k) = P(U=j)$ and thus $\mathbb{E}\qty[ U|V=k] = \mathbb{E} U$.
\begin{definition}
Define random variable $\mathbb{E}\qty[U|V]$ such that
$$\mathbb{E}\qty[U|V] = \mathbb{E}\qty[U|V=k] $$
if $V=k$.
\end{definition}

\begin{definition}[Tower property]
$$\mathbb{E} \qty\big[\mathbb{E}\qty[U|V]] = \mathbb{E}U$$

Define $$f_{n} = \sum_{k=0}^\infty \sum_{k=0}^\infty \theta^{k} P(z_n=k) = \mathbb{E} \theta^{z_n}$$. 

\end{definition}
\begin{theorem}
$$f_{n+1} (\theta) = f_n (f_X(\theta))$$
or
$$f_{n} (\theta) = \underbrace{f\circ f\circ \dots \circ f}_{n \text{ times}}(\theta))$$
\begin{proof}
	
	Use tower property with $U^{z_{n+1}} $ and $V=\theta^{z_n}$.
	By tower property
	$$\mathbb{E} \qty[\theta^{z_{n+1}}] = \mathbb{E} \qty\big[\mathbb{E} \qty[\theta^{z_{n+1}}| \theta^{z_{n}}]]$$
	$$ \mathbb{E} \qty\big[\mathbb{E} \qty[\theta^{z_{n+1}}| \theta^{z_{n}}]] = \sum_{k=0}^\infty P(z_n=k) \mathbb{E} \qty[\theta^{z_{n+1}}| \theta^{z_{n}}=k]$$
What is $\mathbb{E} \qty[\theta^{z_{n+1}}| \theta^{z_{n}}=k]$?
$$\mathbb{E} \qty[\theta^{z_{n+1}}| \theta^{z_{n}}=k] =\mathbb{E} \qty[\theta^{\sum_{j=1}^{k} X_j^{n+1}}| \theta^{z_{n}}=k] \stackrel{\text{independence}}{=}\mathbb{E} \qty[\theta^{\sum_{j=1}^{z_n} X_j^{n+1}}]\stackrel{\text{independence}}{=} \prod_{j=1}^k \mathbb{E} \qty[\theta^{ X_j^{n+1}}] \stackrel{\text{i.d.}}{=} (f_X(\theta))^k $$
Thus
$$ \mathbb{E} \qty\big[\mathbb{E} \qty[\theta^{z_{n+1}}| \theta^{z_{n}}]] = \sum_{k=0}^\infty P(z_n=k) (f_X(\theta))^k = f_n(f(\theta))$$

Also we can say
$$\mathbb{E} \qty[\theta^{z_{n+1}}|z_n] = \qty(f_X(\theta))^{z_n}$$
\end{proof}
\end{theorem}

\paragraph{Study of $z_n$}
What is $\pi_n= P(z_n=0) = f_n(0) = f(\pi_{n-1})$, probability that population is extinguished. Since $z_{n-1} =0 \Rightarrow z_n=0$, i.e. $\pi_n$ is non-decreasing.

Let $P\qty(z_n=0 \text{ for some n}) = \pi$.

We hope that $\left\{ z_n=0 \right\}$ such that
$$\bigcup_n \left\{ z_n=0 \right\} = \left\{ z_n=0 \text{ for some n}\right\}$$
i.e., $\pi = \lim_{n\to \infty} \pi_n$. We call $\pi$ the extinction probability.

\begin{theorem}
	If $\mu=\mathbb{E} > 1$ then $\pi$ is a unique root of $\pi=f(\pi)$ and $\pi \in [0,1)$. If $\mu\leq 1$, $\pi=1$.
	
	If we look at $f(\pi)$ and $\pi$, they intersect in $1$, and they can intersect in two points since $f(x)$ is convex. There is second intersection iff $f'(1) = \mu > 1$.
\end{theorem}
\paragraph{Construction of $X_n^r$}
Construct set $\Omega$, $f_{n,r}: \Omega \to \mathbb{Z}^+$ and $\mathcal{F}$ a collection of subsets of $\Omega$ with $P: \mathcal{F} \to [0,1]$.

Let $\Omega = \mathbb{Z}^+ \times \mathbb{Z}^+$, $\mathcal{F} = \left\{ 0,1 \right\}^\Omega$. 

The problem is when we have infinitely number of variables.



\paragraph{Example}
Example of not well-behaved triple $(\Omega, \mathcal{F}, P)$. $\Omega =\mathbb{N}$. Now $\mathcal{F} = \left\{  C\subset \mathbb{N} : C \text{ has density} \right\}$.

$C$ has density means 
$$\frac{\abs{C \cap \mathbb{N}}}{n} \stackrel{n\to \infty}{\to} \rho(C)$$

However, for $C(m) = \left\{  1,2,\dots, m \right\}$, $\forall m \quad \rho(c_m)$, and 
$$\rho\qty(\bigcup C_m ) = 1$$

Thus $(\mathbb{N}, \mathcal{F}, \rho)$ is not a good probability space, since it doesn't fulfills this
$\pi_n\to \pi$ property. Note we can define other probabilities on naturals, for example
$$P\qty(\left\{ i \right\}) = 2^{-i}$$


\paragraph{Asymptotics of $z_i$ }
Assuming $\pi \in (0,1)$, what is behavior of $z_n$?
\begin{definition}
	$z_n$ is a Markov chain if 
	$$P(z_{n+1}=j|z_i=k_i \quad \forall i\leq n) = P(z_{n+1} = j | z_n=k_n)$$
	We can use to compute expectation:
	$$\mathbb{E}[z_{n+1}|z_i=k_i \quad \forall i<n] = E\qty[z_{n+1}|z_n=k_n]$$
\end{definition}

Then, since $E\qty[\sum_{i=1}^J X_i^n] = J\mu$
$$E\qty[z_{n+1}|z_n] = \mu z_n$$

Let $M_n = \frac{z_n}{\mu^n}$ then $\mathbb{E}[M_n] = 1$. Also
$$\mathbb{E}[M_{n+1}|z_0,\dots, z_n] = M_n$$
This is a definition of martingale with respect to $z_0, \dots, z_n$.

Let $\qty(\Omega, \mathcal{F}, P)$ we say $S$ happens almost surely (a.s.) if $$P\qty(\left\{ w\in \Omega : S \text{ is true for w} \right\}) = 1$$
\begin{theorem}[Martingale convergence theorem]
	If $M_n$ is a positive martingale then $\lim_{n\to \infty} M_N=M_\infty$ exists a.s. and 
	\begin{itemize}
		\item $\mu \leq 1$. $M_\infty =0 $ a.s. That means $\mathbb{E}M_\infty = 0$ but $\mathbb{E} M_ = 1$, i.e., 
		$$\mathbb{E}\qty[\liminf_{n\to \infty} M_n] < [\liminf_{n\to \infty} \mathbb{E}\qty[M_n]$$ 
		\item $\mu>1$. If $M_\infty > 0$ with positive probability then $z_n \sim \mu^n M_\infty$.  
	\end{itemize}
\end{theorem}


\begin{lemma}[Fatou's lemma] 
	$$\mathbb{E}\qty[\liminf_{n\to \infty} M_n] \leq \liminf_{n\to \infty} \mathbb{E}\qty[M_n]$$ 
\end{lemma}

\begin{theorem}
	$$\mathbb{E} \qty[M_\infty] = 1 \iff \mu>1 \quad \text{ and } \mathbb{E}\qty[X \log(X)] < \infty$$
\end{theorem}

\section{Overview of measure theory}
\paragraph{Notation}
\begin{itemize}
	\item $S$ is a set.
	\item $\mathcal{A}$ is algebra of subsets of $S$
	\begin{enumerate}
		\item $S\in \mathcal{A}$
		\item $$E\in \mathcal{A} \Rightarrow E^C\in \mathcal{A}$$, where $E^C = S \setminus E$
		\item $$E_1, E_2 \in \mathcal{A} \Rightarrow E_1 \cup E_2 \in \mathcal{A}$$
		meaning
		$$E_1, E_2 \in \mathcal{A} \Rightarrow E_1 \cap E_2 \in \mathcal{A}$$
	\end{enumerate}
\item $\mathcal{F}$ is a $\sigma$-algebra if lest item works for countable union.
\item $E\Delta F = E\setminus F \cup F \setminus E$ 
\end{itemize}

\begin{definition}
	A measurable space is a pair $\left\{S,\mathcal{F}\right\}$.
\end{definition}

\begin{prop}
	If we have $\qty(\mathcal{F}_i)_{i\in I}$, then $\bigcap_{i\in I} \mathcal{F} $ is also a $\sigma$-algebra.
\end{prop}

\begin{definition}
	Let $C$ be a collection of subsets of $S$. $\sigma(C)$ is a smallest $\sigma$-algebra containing $C$ ($\sigma$-algebra generated by $C$).
	
	It is easy to construct one
	$$I = \left\{ \mathcal{F} :\mathcal{F} \supset C \right\}$$
	and then
	$$\sigma(C) = \bigcap_{\mathcal{F}\in I} \mathcal{F}$$
\end{definition}

\begin{definition}
	Let $\left\{S,\mathcal{F}\right\}$ be a topological space. $\mathcal{B}(X)$ (Borel $\sigma$-algebra) is defined as $\sigma$-algebra generated by open sets. We denote
	$\mathcal{B} = \mathcal{B}(\mathbb{R})$.
\end{definition}
\paragraph{Exercise}
$$\pi(\mathbb{R}) = \left\{ (-\infty, x], \: x\in \mathbb{R} \right\}$$
Show that $\sigma(\pi(\mathbb{R})) = B$

\begin{definition}
	Additive set function on a collection of sets $\mathcal{F}$ is
	$$\mu: \mathcal{F} \to [0, \infty)$$
	$$\forall E,F \in \mathcal{F} \: E\cap F = \emptyset \quad \mu(E\cup F) = \mu(E) +\mu(F) $$
	
	We say $\mu$ is $\sigma$-additive if same holds of countable infinite sets
	$$\forall \left\{ E_i \right\}_{i=1}^\infty \: E_i \cap E_j = \emptyset \quad \mu\qty(E\cup F) = \sum_{i=1}^\infty \mu(E_i)  $$
	
\end{definition}


\begin{definition}
	A triple $\qty(S, \mathcal{F}, \mu)$ is a measure space if $\mathcal{F}$ is a $\sigma$-algebra on $S$ and $\mu$ is $\sigma$-additive on $\mathcal{F}$.
	
\end{definition}


\begin{definition}
 $\qty(S, \mathcal{F}, \mu)$ is finite if $\mu(S) < \infty$
 
  $\qty(S, \mathcal{F}, \mu)$ is $\sigma$-finite if
	$$\exists \left\{ E_i, \: \mu(E_i)<\infty \right\}_{i=1}^\infty \quad S=\bigcup_{i=1}^\infty E_i$$
\end{definition}
\begin{definition}
If $\mu(S) =1$, $\qty(S, \mathcal{F}, \mu)$ is probability space.
\end{definition}

\begin{definition}
	$E$ is null if $\mu(E)=0$.
\end{definition}
\begin{definition}
	$\phi$ is said to be true almost everywhere with respect of $\mu$ if 
	$$\mu(\left\{ X : \phi(X) = \text{False} \right\}) = 0$$
\end{definition}

\subsection{Results from measure theory}
\begin{definition}
	A collection of sets $\mathcal{D}$ is called a $\pi$-system if $E,F \in D \: \Rightarrow \: E\cap F \in \mathcal{D}$
\end{definition}
\begin{theorem}[Uniqness]
	Let $\mathcal{D}$ be a $\pi$-system generating a $\sigma$-algebra $\mathcal{F}$. Let $\mu_1$ and $\mu_2$ be two finite measures on $\mathcal{F}$ which agree on $\mathcal{D}$. Then $\mu_1=\mu_2$.
	
	\begin{coll}
		$(S,\mathcal{F}, P_1)$, $(S,\mathcal{F}, P_2)$ probability spaces, $P1=P2$ on $\pi$-system $\mathcal{D}$, then $P_1=P_2$.
	\end{coll}
\end{theorem}

\begin{theorem}[Carath\'{e}odory's extension theorem]
	Let $\mathcal{A}$ be an algebra of sets. $\mu_0: \mathcal{A} \to \mathbb{R}^+$ $\sigma$-additive set function on $\mathcal{A}$. Then exists unique extension $\bar{\mu} : \sigma(\mathcal{A}) \to \mathbb{R}^+ $ such that $\bar{mu} = \mu_0$.
\end{theorem}

\paragraph{Homework}
Lebesgue on $\mathbb{R}$.  $\mathcal{A} = \left\{ \text{open set} \right\}$. If we have
$$O = \bigcup_{i=1}^\infty (a_i, b_i)$$
then
$$\mu_0(O) = \sum_{i=1}^\infty b_i-a_i$$

Check that $\mu_0$ is well defined and $\sigma$-additive.

\begin{lemma}
	
	$(S,\mathcal{F}, \mu)$ measure space. $A,B \in \mathcal{F}$, then
	$$\mu(A\cup B) \leq \mu(A) + \mu(B) $$
	$$\mu\qty(\bigcup_{i=1}^\infty F_i) \leq \sum_{i=1}^\infty \mu(F_i) $$
	
	If $\mu(S) < \infty$
	$$\mu(A\cup B) = \mu(A) + \mu(B) - \mu(A\cap B)$$
	
	From that we get inclusion-exclusion:
	$$\mu\qty(\bigcup_{i=1}^n A_i) = \sum_{i=1}^n \mu(A)_i - \sum_{i\neq j}\mu(A_i\cap A_j) + \dots + (-1)^{n-1} \mu\qty(\bigcap_{i=1}^n A_i)$$
\end{lemma}
\paragraph{Exercise}
Proof the lemma

\begin{lemma}
	If $F_n \subseteq F_{n+1}$ then
	$$\mu\qty(\bigcup_{i=1}^\infty F_i) = \lim_{n\to \infty} \mu(F_n)$$
	
	
	If $\mu(S) < \infty$ and $F_n \supseteq F_{n+1}$ then
	$$\mu\qty(\bigcap_{i=1}^\infty F_i) = \lim_{n\to \infty} \mu(F_n)$$
\end{lemma}
\begin{proof}
	Assume $\mu(S) < \infty$.  Define 
	$F_\infty = \bigcup_{i=1}^\infty F_i$.
	Let $G_n = F_{n}\setminus F_{n+1}$. Then
	$$F_\infty = \bigcup_{i=1}^\infty G_i$$
	Meaning 
	$$\mu(F_\infty) =\sum_{i=1}^\infty G_i$$
	$$\mu(F_n) =\sum_{k=1}^n G_k$$
	
	Since measure is finite, the tail of series tends to $0$, thus 
	$$\mu(F_\infty) - \mu(F_n) = \sum_{k=n}^\infty G_k \to 0$$
	
	Then we can take complements and get the second statement.
\end{proof}

\paragraph{Exercise}
Proof unconditionally
\end{document}
