\documentclass[]{article}
\usepackage{amsmath}
\usepackage{amsfonts}
\usepackage{amssymb}
\usepackage{hyperref}
\usepackage{gensymb}
\usepackage{graphicx}
\usepackage{svg}
\usepackage{bbding}
\usepackage{mathtools}
\usepackage{centernot} % not parallel, etc.
\usepackage{lmodern}
\usepackage{morewrites}
\usepackage{xcolor,sectsty} % colorful sections
\usepackage[left=10mm, top=10mm, right=10mm, bottom=20mm, nohead]{geometry}
%\usepackage{bigints}
\usepackage{dsfont} %mathbb 1
\usepackage{esint} % beatiful integrals
\usepackage[arrowdel]{physics}

\usepackage[T1]{fontenc}
% Nicer default font (+ math font) than Computer Modern for most use cases
% \usepackage{mathpazo} % problems with greek vectors
\usepackage[utf8x]{inputenc} % Allow utf-8 characters in the tex document
% Prevent overflowing lines due to hard-to-break entities
\sloppy 
% Colors for the hyperref package
\definecolor{urlcolor}{rgb}{0,.145,.698}
\definecolor{linkcolor}{rgb}{.71,0.21,0.01}
\definecolor{citecolor}{rgb}{.12,.54,.11}
% Setup hyperref package
\hypersetup{
	breaklinks=true,  % so long urls are correctly broken across lines
	colorlinks=true,
	urlcolor=urlcolor,
	linkcolor=linkcolor,
	citecolor=citecolor,
}


\DeclareFontFamily{OMX}{lmex}{}
\DeclareFontShape{OMX}{lmex}{m}{n}{<-> lmex10}{}


%colors of sections
\definecolor{secfont}{RGB}{46,116,181}
\definecolor{subfont}{RGB}{146,23,57}
\definecolor{parfont}{RGB}{19,127,43}
\definecolor{subparfont}{RGB}{7,11,100}

\subsectionfont{\color{subfont}}
\sectionfont{\color{secfont}}
\paragraphfont{\color{parfont}}
\subparagraphfont{\color{subparfont}}

%\usepackage{babel}[english]
%opening
\title{106349 - Advanced probability}
\author{Nick Crawford}
% njc860@gmail.com
% Amado 707
% https://sites.google.com/site/njcrawfordacademic/
% David Williams Probability with Martingales
% Rick Derret Probability Theory and examples

\parindent=0em
\begin{document}


\maketitle

\begin{abstract}

\end{abstract}

%\tableofcontents
\section{Introduction. Summary of course through an example. Branching process}
We have an individual that gives a birth to a random number of offsprings -- random variable $X$.
$X$ define a distribution, i.e., $P: \mathbb{Z}^+ \to [0,1]$, i.e., $P(X=k) \in [0,1]$, and $\sum_{k=0}^\infty P(X=k)  = 1$.

\begin{definition}
	$f_X(\theta) = \sum_{k=0}^\infty \theta^k P(X=k)$ -- moment-generating function.
\end{definition}

 The series is absolutely convergent for $\theta\in [-1,1]$ since $k$ sums to $1$. For $\theta \in (-1,1)$, $f_x$ is analytic, thus we can differentiate it term-by-term:
$$f'_X(\theta) = \sum_{k\geq1} \theta^{k-1} P(X=k)$$
Since, $f_X$ is analytic, knowing it means knowing $P(X=k)$ and vice versa.

Note that $f_X(0) = P(X=0)$ and $f_X(1)=1$. Also
$$f'_X(1) = \sum_{k\geq 0 }^\infty k P(X=k) = \mathbb{E} X = \mu$$
$$\lim_{\theta \to 1} \frac{f_X(1)-f_X(\theta)}{1-\theta}=\lim_{\theta \to 1} \frac{1-f_X(\theta)}{1-\theta}$$

Note also that $f_X$ is convex, since second derivative is positive.
\paragraph{Size of $n^{th}$ generation}
Let $\qty(X_r^{(n)})_{n,r=1^\infty}$, where $n$ is generation and $r$ is offspring number (index) in $n^{th}$ generation.

Assume $X_r^{(n)}$ are i.i.d.  (independent, identically distributed) random variables.

Identically distributed means $$P(X_n^r = k) = P(X=k)$$.

Independence means $$P\qty(\forall i<J \: X_{r_i}^{n_i} =k) = \prod_{i=1}^J P\qty( X_{r_i}^{n_i} =k)$$.

Define $z_1=X_1^1$. $z_2 = \sum_{r=1}^{z_1} X_r^2$ an so on:
$$z_{n+1} = \sum_{r=1}^{z_n} X_{r}^{n}$$

We want to study asymptotics of $z_n$.

Given $U$ and $V$ taking values in $ \mathbb{Z}^+$,
$$\mathbb{E}] \qty[ U|V=k] = \sum_{j=0}^\infty j P(U=j | V=k)$$, where $$P(U=j|V=k) = \frac{P(U=j, V=k)}{P(V=k)}$$

\paragraph{} If $U$, $V$ are independent, $P(U=j|V=k) = P(U=j)$ and thus $\mathbb{E}\qty[ U|V=k] = \mathbb{E} U$.
\begin{definition}
Define random variable $\mathbb{E}\qty[U|V]$ such that
$$\mathbb{E}\qty[U|V] = \mathbb{E}\qty[U|V=k] $$
if $V=k$.
\end{definition}

\begin{definition}[Tower property]
$$\mathbb{E} \qty\big[\mathbb{E}\qty[U|V]] = \mathbb{E}U$$

Define $$f_{n} = \sum_{k=0}^\infty \sum_{k=0}^\infty \theta^{k} P(z_n=k) = \mathbb{E} \theta^{z_n}$$. 

\end{definition}
\begin{theorem}
$$f_{n+1} (\theta) = f_n (f_X(\theta))$$
or
$$f_{n} (\theta) = \underbrace{f\circ f\circ \dots \circ f}_{n \text{ times}}(\theta))$$
\begin{proof}
	
	Use tower property with $U^{z_{n+1}} $ and $V=\theta^{z_n}$.
	By tower property
	$$\mathbb{E} \qty[\theta^{z_{n+1}}] = \mathbb{E} \qty\big[\mathbb{E} \qty[\theta^{z_{n+1}}| \theta^{z_{n}}]]$$
	$$ \mathbb{E} \qty\big[\mathbb{E} \qty[\theta^{z_{n+1}}| \theta^{z_{n}}]] = \sum_{k=0}^\infty P(z_n=k) \mathbb{E} \qty[\theta^{z_{n+1}}| \theta^{z_{n}}=k]$$
What is $\mathbb{E} \qty[\theta^{z_{n+1}}| \theta^{z_{n}}=k]$?
$$\mathbb{E} \qty[\theta^{z_{n+1}}| \theta^{z_{n}}=k] =\mathbb{E} \qty[\theta^{\sum_{j=1}^{k} X_j^{n+1}}| \theta^{z_{n}}=k] \stackrel{\text{independence}}{=}\mathbb{E} \qty[\theta^{\sum_{j=1}^{z_n} X_j^{n+1}}]\stackrel{\text{independence}}{=} \prod_{j=1}^k \mathbb{E} \qty[\theta^{ X_j^{n+1}}] \stackrel{\text{i.d.}}{=} (f_X(\theta))^k $$
Thus
$$ \mathbb{E} \qty\big[\mathbb{E} \qty[\theta^{z_{n+1}}| \theta^{z_{n}}]] = \sum_{k=0}^\infty P(z_n=k) (f_X(\theta))^k = f_n(f(\theta))$$

Also we can say
$$\mathbb{E} \qty[\theta^{z_{n+1}}|z_n] = \qty(f_X(\theta))^{z_n}$$
\end{proof}
\end{theorem}

\paragraph{Study of $z_n$}
What is $\pi_n= P(z_n=0) = f_n(0) = f(\pi_{n-1})$, probability that population is extinguished. Since $z_{n-1} =0 \Rightarrow z_n=0$, i.e. $\pi_n$ is non-decreasing.

Let $P\qty(z_n=0 \text{ for some n}) = \pi$.

We hope that $\left\{ z_n=0 \right\}$ such that
$$\bigcup_n \left\{ z_n=0 \right\} = \left\{ z_n=0 \text{ for some n}\right\}$$
i.e., $\pi = \lim_{n\to \infty} \pi_n$. We call $\pi$ the extinction probability.

\begin{theorem}
	If $\mu=\mathbb{E} > 1$ then $\pi$ is a unique root of $\pi=f(\pi)$ and $\pi \in [0,1)$. If $\mu\leq 1$, $\pi=1$.
	
	If we look at $f(\pi)$ and $\pi$, they intersect in $1$, and they can intersect in two points since $f(x)$ is convex. There is second intersection iff $f'(1) = \mu > 1$.
\end{theorem}
\paragraph{Construction of $X_n^r$}
Construct set $\Omega$, $f_{n,r}: \Omega \to \mathbb{Z}^+$ and $\mathcal{F}$ a collection of subsets of $\Omega$ with $P: \mathcal{F} \to [0,1]$.

Let $\Omega = \mathbb{Z}^+ \times \mathbb{Z}^+$, $\mathcal{F} = \left\{ 0,1 \right\}^\Omega$. 

The problem is when we have infinitely number of variables.
\end{document}
