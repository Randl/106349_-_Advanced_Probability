\subsection{Markov chains}

Let $\left\{ X_n \right\}$ be a stochastic process taking values in $\qty(E, \mathcal{E})$ and $\left\{ \mathcal{F}_n\right\}$ be a filtration such that $X_n\in \mathcal{F}_n$. 
\begin{definition}
	$p: E\times \mathcal{E} \to [0,1]$ is a transition kernel on $E$ if 
	\begin{enumerate}
		\item $\forall e\in E$ $P(e,\cdot)$ is a probability measure.
		\item $\forall A\in \mathcal{E}$ $p(\cdot, A) \in \mathcal{E}$
	\end{enumerate}
\end{definition}
\begin{definition}
$\left\{ X_n \right\}$ is a Markov chain with respect to $\mathcal{F}_n$ with transition kernel $p$ if 
$\forall A\in \mathcal{E}$
$$P(X_{n+1} \in A | \mathcal{F}_n) = p(X_n,A)$$

\end{definition}
We can acquire $$\mathbb{E}\qty[h(X_{n+1}|\mathcal{F}_n)] = \int\limits_E p(X_n, \dd{e}) h(e)$$


Given $(E, \mathcal{E}, p)$ does $\exists X_n$, $\mathcal{F}_n$ Markov with kernel $p$? The answer is yes.

Let $\Omega = E^{\mathbb{N}}$, $\mathcal{F} = E^{\bigotimes \mathbb{E}}$ and $\mathcal{F}_n = \sigma\qty(X_i(\omega) : i\leq n)$. We want
$X(\omega) = \omega_n$. 

 We want to define law of $\left\{ X_n \right\}$ by first specifying 
 \begin{align*}
 P(A_0 \times A_1\times \dots A_n \times E \times E \times \dots) = \mathbb{E} \qty[\prod_{i=0}^n \mathds{1}_{X_i\in A_i}] = \mathbb{E} \qty[\prod_{i=0}^{n-1} \mathds{1}_{X_i\in A_i} \mathbb{E} \qty[\mathds{1}_{X_n\in A_n} | \mathcal{F}_{n-1}]] = \mathbb{E} \qty[\prod_{i=0}^{n-1} \mathds{1}_{X_i\in A_i} p(X_{n-1}, A_n)] =\\= \mathbb{E} \qty[\prod_{i=0}^{n-2} \mathds{1}_{X_i\in A_i} \int\limits_{A_{n-1}} p(X_{n-2}, \dd{e})p(e, A_n)] =\dots = \mathbb{E} \qty[\mathds{1}_{X_0\in A_0} \int\limits_{A_1} \dots \int\limits_{A_n} p(X_0, \dd{e_1})p(e_1, \dd{e_2})\dots p(e_{n-1}, \dd{e_n})]
 \end{align*}

Let law of $X_0$ be $\mu$ on $E$. Then
$$P(A_0 \times A_1\times \dots A_n \times E \times E \times \dots) = \int\limits_{A_0} \mu(\dd{e_0}) \int\limits_{A_1}  p(X_0, \dd{e_1})  \int\limits_{A_2}  p(e_1, \dd{e_2}) \dots \int\limits_{A_n}\dots p(e_{n-1}, \dd{e_n}) $$

From now on we'll assume $E$ is either finite or countable. In this case, Markov condition is $\exists p(i,j)$ such that
$$P(X_{n+1} =j | \mathcal{F}_n) = p(X_n,j)$$

And then
$$\mathbb{E} [h(X_{n+1})|\mathcal{F}_n] = \sum_{j\in E} P(X_n, j)h(j) = p \vdot j$$
where $p$ is matrix and $h$ is a vector.

\begin{definition}
	$h$ is called $p$-superharmonic if $p\vdot h \leq h$ or alternatively, if
	$Y_n = h(X_n)$ is a p-supermartingale.
\end{definition}
\begin{definition}
	Let $T_i = \inf \left\{ n\geq 1: X_n=i \right\}$.
\end{definition}
\begin{definition}
	We say $X_n$ is irreducible if $P_i(T_j<\infty) > 0$ where $P_i$ is a law of $X_n$ started with $X_0=i$ with probability $i$.
\end{definition}
\begin{definition}
We say $X_n$ is irreducible recurrent if $\forall i,j \in E \quad P_i(T_j<\infty) = 1$.
\end{definition}
\begin{theorem}
	$\left\{X_n\right\}$ is irreducible recurrent on $E$ iff all positive superharmonic are constant.
	\begin{proof}
		$\Rightarrow:$
		Let $\left\{X_n\right\}$ is irreducible recurrent on $E$ and $h$ a positive $p$-superharmonic function. Consider $$\mathbb{E}_i \qty[h\qty(X_n^{T_j})] \leq h(i)$$
		Thus by Fatou
		$$\mathbb{E}_i \qty[\liminf h\qty(X_n^{T_j})] \leq h(i)$$
		and now $h(i) \leq h(j)$. By symmetry, $h(j) = h(i)$.
	\end{proof}
\end{theorem}

How to produce $p$-harmonic functions? Let $A\subseteq E$ be some set and $g: A\to \mathbb{R}$ be a bounded function. Assume $\forall i P_i(T_A <\infty) = 1$ and let $h(i) = \mathbb{E}_i \qty[g(X_{T_A})]$.

\begin{lemma}
	$h$ is $p$-harmonic on $A^c$.
	\begin{proof}
		For $i\notin A$, $T_A\geq 1$.
		
		$$P(X_{n+1}\in A | \mathcal{F}_n) = p(X_n, A)$$
		
		$$\forall i \quad P_i(X_{n+1}\in A_i, X_{n+2}\in A_2 | \mathcal{F}_n)  =\int\limits_{A_1} p(X_n, \dd{e_1})\int\limits_{A_2} p(X_n, \dd{e_2}) = P(\hat{X}_1\in A_1, \hat{X}_2\in A_2)$$
		where $\hat{X}_i = X_{n+i}$.
		
		Moreover 
		$$P(X_{n+i} \in F|\mathcal{F}_i) = P_{X_n} (\hat{X}_i \in F)$$
		
		Thus
		$$\mathbb{E} \qty[g(X_{T_A})|\mathcal{F}_1] = \mathbb{E}_{X_1} \qty[g(X_{T_A})] = h(X_1)$$
		and
		
		$$h(i) = \mathbb{E}_i \qty[\mathbb{E} \qty[g(X_{T_A})|\mathcal{F}_1]] = \mathbb{E}_i [h(X_1)] = \sum_j p_{ij} h_j = p\cdot h$$
	\end{proof}
\end{lemma}
