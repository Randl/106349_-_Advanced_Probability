\section{Recasting measure theory as probability}
\begin{definition}
	A probability space is a $(\Omega, \mathcal{F}, P)$ is a measure space such that $P(\Omega) = 1$.
	
	We call $\omega \in \Omega$ an outcome. $E\in \mathcal{F}$ is an event. $P(E)$ is probability of the event.
\end{definition}

\paragraph{Example}
Tossing finite or infinite sequence of coins.


\subparagraph{Tossing 4 coins}
$$\Omega = \left\{ HHHH, HHHT, HHTH, \dots, TTTH, TTTT \right\}$$
$$\mathcal{F} = 2^\Omega$$
$$P(\omega \in \Omega) = \frac{1}{\abs{\Omega}}$$
	
\subparagraph{Tossing infinite number coins}
$$\Omega = \left\{ 0,1 \right\}^{\mathbb{N}}$$


$\Omega$ has a natural topology which is called a product topology. It is coarsest topology such that $\pi_i: \Omega \to \left\{ 0,1 \right\}$ $\pi_i(\omega) =\omega_i$ is continuous. 

Let $\mathcal{F} = \mathcal{B}(\Omega)$.

Smallest $\sigma$-algebra such that
$$\pi_i^{-1}(0) \subset \Omega \in \mathcal{F} $$
$$\pi_i^{-1}(1) \subset \Omega \in \mathcal{F} $$

$$\pi_i (\Omega, \mathcal{F} ) \to \qty(\left\{ 0,1 \right\}, \left\{ 0,1 \right\}^{\left\{ 0,1 \right\}})$$

Natural $\pi$-system $\mathcal{F}_n$ smallest $\sigma$-algebra making $\pi_1, \dots, \pi_n$ measurable.

Note that
\begin{prop}
	
	$$\bigcup_n \mathcal{F}_n \neq \mathcal{F} $$
	\begin{proof}
		
		Define $S_n(\omega) = \sum_{i=1}^n \omega_n$. 
		$$X_n = \frac{S_n(\omega)}{n}$$
		Define
		$$Y(\omega) = \limsup X_n(\omega)$$
		$$E = \left\{ \omega : Y(\omega) \geq \frac{1}{3} \right\}$$
		$$E \in \mathcal{F} \setminus \bigcup_n \mathcal{F}_n$$
	\end{proof}
\end{prop}


\paragraph{What $\mathcal{F}_n$ looks like?}
For example, $\mathcal{F}_2$ has 4 outcomes, deciding only first two tosses.
\paragraph{Note} If we take $(\Omega_4, \mathcal{F}^{(4)}, P_4)$, restricting to $(\Omega_3, \mathcal{F}^{(3)}, P_3)$
$$P_4\qty(\left\{ (0,0,0,\omega_4)\right\}) = P_3\qty(\left\{ (0,0,0)\right\})  $$


Thus we want $P_{fair}$ defined on $\Omega$ to  fulfill same property:
$$P_{fair}(E) = P_n(\tilde{E})$$
where $E \in \mathcal{F}_n$ and $\tilde{E} \in F^{(n)}$.

\begin{definition}
	$E\subset \mathcal{F}$ occures almost surely (a.s.) if $P(E)=1$.
\end{definition}


\begin{definition}[$\limsup$ and $\liminf$]
 Let	$\left\{ E_n \right\}$ be a sequence of events.
 $$\limsup E_n = \bigcap_{m}\bigcup_{n\geq m} E_n = \left\{ E_n \text{ occurs infenetely often (i.o.)} \right\} = \left\{ \omega \in \Omega  : \: \forall m \: \exists n(\omega)>m \quad \omega \in E_n(\omega) \right\}$$
 
 Alternatively, $(\Omega, \mathcal{F})$ and $\left\{ E_n \right\}$  there is a natural map 
 $$I: \Omega \to \left\{ 0,1\right\}^N$$
 $$\omega \mapsto \left\{ 1_{E_n} (\omega) \right\}$$
 
 where $$1_{E}(\omega) = \begin{cases}
 0 & \omega \notin E\\
 1 & \omega \in E\\
 \end{cases}$$
 
 Now
 
 $$\liminf E_n = \bigcup_{m}\bigcap_{n\geq m} E_n = \left\{ E_n \text{ occurs eventually} \right\} = \left\{ \omega \in \Omega  : \: \exists m(\omega) \: \forall n\geq m(\omega) \quad \omega \in E_n(\omega) \right\}$$
\end{definition}

\paragraph{Remark}
Since everything is countable, if $E_n \in \mathcal{F}$, then $\limsup E_n,\liminf E_n\in \mathcal{F}$ 

We can write
$$\left\{ \frac{S_n}{n} \to \frac{1}{2} \right\} = \left\{ \limsup \frac{S_n}{n} \leq \frac{1}{2} \right\} \cap \left\{ \liminf \frac{S_n}{n} \geq \frac{1}{2} \right\}$$.

Choose $q\in \mathbb{Q}^+$ and take a look at $$\left\{ \liminf \frac{S_n}{n} > q \right\} = \liminf E_n(q)$$
where $E_n = \left\{ \omega  : \frac{S_n}{n} > q \right\}$.

In addition
$$\left\{ \limsup \frac{S_n}{n} < q \right\} = \liminf F_n(q)$$
where $F_n = \left\{ \omega  : \frac{S_n}{n} < q \right\}$.

Therefore
$\left\{ \liminf \frac{S_n}{n} > q \right\} \in \mathcal{F}$.


Finally,
$$\left\{ \liminf \frac{S_n}{n} \geq \alpha \right\} = \bigcap_{q<\alpha} \left\{ \liminf \frac{S_n}{n} > q \right\} $$



\begin{lemma}[Fatou's lemma]
	$$P\qty[\liminf_{n\to \infty} E_n] \leq \liminf_{n\to \infty} p\qty(E_n)$$
	
	\begin{proof}
		$$\liminf_{n\to \infty} E_n = \bigcup_m \bigcap_{n\geq m} E_n $$
		
		Sets $F_m= \bigcap_{n\geq m} E_n $ are increasing and $F_n \subseteq E_n$, thus
		$$P\qty[\liminf_{n\to \infty} E_n] = \lim_{n\to \infty} P(F_n) \leq \liminf_{n\to \infty} P(E_n)  $$
	\end{proof} 
\end{lemma}


\begin{lemma}[Fatou's lemma]
	$$P\qty[\limsup_{n\to \infty} E_n] \geq \limsup_{n\to \infty} p\qty(E_n)$$
	
	\begin{proof}
		Note that $\qty(\limsup E_n)^C = \liminf E_n^C$, thus this is straightforward form previous lemma.
	\end{proof} 
\end{lemma}

\begin{lemma}[First Borel-Cantelli lemma] \label{bc1}
	Let $\left\{ E_n\right\} \subseteq \mathcal{F}$ be a sequence of events s.t. $\sum_n P(E_n) < \infty$, then
	$$P(E_n \text{ happens i.o.}) =0$$
	
	\begin{proof}
		$$P(E_n \text{ i.o.}) = P\qty(\bigcap_m \bigcup_{n\geq m} E_n) \leq  P\qty( \bigcup_{n\geq m} E_n) \leq \sum_{n=m}^\infty P(E_n) \stackrel{m\to \infty}{\to} 0$$
		Since $P(E_n \text{ i.o.})$ is independent on $m$, it got to be $0$.
	\end{proof}
\end{lemma}

\paragraph{Example}
Fix $\epsilon>0$. Look at $P\qty(\abs{\frac{S_n(\omega)}{n} -\frac{1}{2}}>\epsilon)$.

\paragraph{Claim} $$P\qty(\abs{\frac{S_n(\omega)}{n} -\frac{1}{2}}>\epsilon) \leq \frac{12}{\epsilon^4} \frac{1}{n^2}$$

By \ref{bc1} $P\qty(\abs{\frac{S_n(\omega)}{n} -\frac{1}{2}}>\epsilon \text{ i.o.}) = 0$ thus
$$\left\{ \frac{S_n}{n} \to \frac{1}{2} \right\} = \bigcap_{\epsilon>0} \left\{ \abs{\frac{S_n(\omega)}{n} -\frac{1}{2}}<\epsilon \text{ eventually}  \right\} = 1$$
		
		
